\documentclass[12pt, titlepage]{article}

\usepackage{amsmath, mathtools}

\usepackage[round]{natbib}
\usepackage{amsfonts}
\usepackage{amssymb}
\usepackage{graphicx}
\usepackage{colortbl}
\usepackage{xr}
\usepackage{hyperref}
\usepackage{longtable}
\usepackage{xfrac}
\usepackage{tabularx}
\usepackage{float}
\usepackage{siunitx}
\usepackage{booktabs}
\usepackage{multirow}
\usepackage[section]{placeins}
\usepackage{caption}
\usepackage{fullpage} 
\usepackage{amsmath}
\hypersetup{
    colorlinks,
    citecolor=black,
    filecolor=black,
    linkcolor=red,
    urlcolor=blue
}
\usepackage[round]{natbib}

\input{../../Comments}
%% Common Parts

\newcommand{\progname}{SpecSearch} % PUT YOUR PROGRAM NAME HERE %Every program
                                % should have a name


\begin{document}

\title{Unit Verification and Validation Plan for \progname{}} 
\author{Robert E. White}
\date{\today}
	
\maketitle

\pagenumbering{roman}

\section{Revision History}

\begin{tabularx}{\textwidth}{p{3cm}p{2cm}X}
\toprule {\bf Date} & {\bf Version} & {\bf Notes}\\
\midrule
2018-11-25 & 1.0 & Creation of first draft.\\
2018-11-29 & 1.1 & Update of 1.0. Completed tracebility, added references, 
deleted list of tables and updated section 2. \\ 
2018-12-02 & 1.2& Minor edits before submission. \\ 
2018-12-06 & 1.3& First edit for final documentation. More details were added 
to the plotting tests. The "qualities of a good plot" list was added to the 
appendix. \\ 
2018-12-09 & 1.4& Creation of final draft for final documentation. \\
\bottomrule
\end{tabularx}

~\newpage

\tableofcontents

\listoftables


\listoffigures


\newpage

\section{Symbols, Abbreviations and Acronyms}

\renewcommand{\arraystretch}{1.2}
\begin{tabular}{l l} 
  \toprule		
  \textbf{symbol} & \textbf{description}\\
  \midrule 
  T & Test\\ 
  SRS & System Requirements Specfication\\ 
  MG & Module Guide\\ 
  MIS & Module Interface Specification\\ 
  VnV & Verification and Validation\\ 
  M & Module\\ 
  R & Requirement\\
  \bottomrule
\end{tabular}\\

See SRS (\cite{SRS}), MG (\cite{MG}) and MIS (\cite{MIS}) documentation at 
\url{https://github.com/whitere123/CAS741_REW} for a more complete table. 

\newpage

\pagenumbering{arabic}

Units are small components of source code. Unit testing involves testing 
the individual units of the software package.  A unit Verification and 
Validation plan consists of outlining test cases for particular units of a 
software's modules. These tests are created to ensure that the units satisfy 
the software's functional and nonfunctional requirements. The tests can be 
traced to a particular module. The module should be traced to a particular 
requirement. 

\section{General Information}

\subsection{Purpose}

The software that is being unit tested is called \progname. \progname \ is 
scientific computing software that finds and plots the spectrum of a matrix 
operator that appears in a LAX pair compatible with solutions of the Non-Linear 
Schr\"{o}dinger equation for three different numerical algorithms and two 
boundary solutions. The modules are listed and described in the MG (\cite{MG}) 
and 
MIS (\cite{MIS}). The requirements are outlined in the SRS (\cite{SRS}). All of 
these documents can be 
found in \url{https://github.com/whitere123/CAS741_REW}. 

\subsection{Scope}

A lot of the modules in \progname \ are built-in MATLAB functions. These 
modules 
are outside of the scope for the Unit Verification and Validation Plan. They 
include: Plotting, Eigenvalue and Eigenvector solver, Matrix, Elliptic 
Functions, Elliptic Integrals, Linspace and Toeplitz.\\  
Control will not be tested because it simply coordinates the other modules.

%\wss{What modules are outside of the scope.  If there are modules that are
%  developed by someone else, then you would say here if you aren't planning on
 % verifying them.  There may also be modules that are part of your software, 
 %but
  %have a lower priority for verification than others.  If this is the case,
%  explain your rationale for the ranking of module importance.}

\section{Plan}
	
\subsection{Verification and Validation Team}

The Verification and Validation team consists of my supervisor, Dr. Dmitry 
Pelinovsky, and I, Robert White. 

\subsection{Automated Testing and Verification Tools}

Matlab provides a unit testing framework. I will write unit-based scripts 
for testing the modules in \progname. The functions and classes that automate 
this process 
are detailed in the following link: 
\url{https://www.mathworks.com/help/matlab/script-based-unit-tests.html}.\\

The functions of particular importance are: throw, catch and assert. The throw 
and catch will also be used for the error tests in the SystVnVPlan. 

%\wss{What tools are you using for automated testing.  Likely a unit testing
%  framework and maybe a profiling tool, like ValGrind.  Other possible tools
 % include a static analyzer, make, continuous integration tools, test coverage
%  tools, etc.  Explain your plans for summarizing code coverage metrics.}

\subsection{Non-Testing Based Verification} 

The non-testing based verfication will involve a code inspection by my 
thesis supervisor. He will inspect each of the modules to guarentee that the 
physical equations have been implemented succesfully, verify that the software 
is designed to be maintainable and manageable and complete the 
two surveys in the appendix \ref{app}.

%\wss{List any approaches like code inspection, code walkthrough, symbolic
%  execution etc.  Enter not applicable if you do not plan on any non-testing
%  based verification.}

\section{Unit Test Description} \label{testDescription} 

The modules that are being unit tested are specificed in the MIS 
(\url{https://github.com/whitere123/CAS741_REW}). This plan was created to 
ensure that each module is behaving accurately and that each module is 
satisfying the appropriate requirements.  \\

All of the inputs for the automated white box unit tests are listed in Table 4 
of the System VnV Plan \url{https://github.com/whitere123/CAS741_REW}. From now 
on this table will be denoted by ``Input table". An automated boundary value 
analysis will be performed for two of the modules. Values of k approaching the 
two boundaries of the (0,1) constraint will be considered. We will see how 
these modules behave in the limit. The prescribed inputs for these tests are in 
\ref{app}. From now on this table will be denoted by ``Boundary Table". 

\subsection{Tests for Functional Requirements}

%\wss{Most of the verification will be through automated unit testing.  If
% appropriate specific modules can be verified by a non-testing based
%  technique.  That can also be documented in this section.}

\subsubsection{Input Parameters Module}

The following tests were created to ensure that the input parameters module is 
storing the correct values specified by the user. This first step is critical 
as all of the other modules rely on these quantities. This is also the first 
requirement outlined by the SRS 
(\url{https://github.com/whitere123/CAS741_REW}). \\
The exception tests are covered in the System 
VnV Plan and the module is explained mathematical in the 
MIS (\url{https://github.com/whitere123/CAS741_REW}). \\

\begin{enumerate}

\item{test-InParams-N\\} \label{INPARAMS}

Type: Automaic
					
Initial State: 
					
Input: Input Table.
					
Output: assert=True

Test Case Derivation: The environment variable N inputted by the user should 
match the state variable from InParams.N .

How test will be performed: Natural number values ranging from 50 to 500 will 
be 
inputed in the environment. This value should be exactly equal to the state 
variable InParams.N . An assert statement will return true if these are equal. 
				
\item{test-InParams-P\\}

Type: Automaic

Initial State: 

Input: Input Table.

Output: assert=True

Test Case Derivation: The environment variable P inputted by the user should 
match the state variable meant to store the value of P.

How test will be performed: Integer numbers between 2 and 4 
will be inputed in the environment for k. This value should be exactly equal to 
the state variable InParams.P . An assert statement will return true if these 
are equal.

\item{test-InParams-k\\}

Type: Automaic

Initial State: 

Input: Input Table.

Output: assert=True

Test Case Derivation: The environment variable k inputted by the user should 
match the state variable meant to store the value of k.

How test will be performed: Real numbers between 0 and 1 
will be inputed in the environment for k. This value should be exactly equal to 
the state variable InParams.k . An assert statement will return true if these 
are equal.
    
\end{enumerate}

\subsubsection{Numerical Parameters Module}

The following tests were created to build confidence in the Numpar state 
variables. The exaxt values of each component of the state variables will not 
be checked individually. This is because they are created with the input 
parameters and built-in matlab functions. The Matlab 
functions are outside of the scope 
of this plan. The input parameters have already been tested in the previously 
mentioned tests \ref{INPARAMS}. We only want to ensure that the input variables 
were correctly entered into the MATLAB functions. \\
These tests enforce the requirement of finding the spectrum. This is because 
the numerical 
parameters will build the matrix whose spectrum we are interested in. \\ 
\begin{enumerate}
\item{test-NumParams-Dom\\}

Type: Automaic

Initial State: 

Input: Input Table. 

Output: assert=True

Test Case Derivation: The domain is created using the linspace matlab function. 
This function accepts two endpoints and a step size as input. The outputted 
vector should have endpoints equal to the inputted endpoints. This test will 
ensure that the domain has the desired endpoints, $+/- xend$. 

How test will be performed: The variable NumParams.Dom(1) and NumPars.Dom(end) 
should be equal to -NumParams.xend and +NumParams.xend respectively. We are 
assuming that linspace correctly fills in the gaps.  

\item{test-NumParams-Dom-Bound\\}

Type: Automaic

Initial State: 

Input: Boundary Table. 

Output: assert=True

Test Case Derivation: Is the same as test-NumParams-Dom. The only exception is 
that the integrand in the integral that defines NumParams.xend diverges near 
one of 
the end points 
of integration when k is close to 1. We are doing a boundary value analysis of 
NumParams against k. 

How test will be performed: The variable NumParams.Dom(1) and NumPars.Dom(end) 
should be equal to -NumParams.xend and +NumParams.xend respectively. The module 
should behave normally in the limit. 

\item{test-NumParams-EllipMat\\}

Type: Automaic

Initial State: 

Input: Input Table.

Output: assert=True

Test Case Derivation: The $Numpars.ellipjMat$ state variable is created by 
applying the built in matlab diag function to $Numpars.ellipjdn$. To ensure 
that the diag function was successfully implemented with $Numpars.ellipjdn$ I 
will check that at least one of the ellipjdn values corresponds to the 
appropriate diagonal element. The other elements should follow since the diag 
function is out of the scope of testing. 

How test will be performed: For the tested points in input tables we will 
check that assert 
($Numpars.ellipjdn(2)==Numpars.EllipMat(2,2)$ ) returns true. 
\end{enumerate} 

\subsubsection{Spectrum Matrix Module}

The following tests were created to ensure that the spectrum matrix state 
variables are of the correct form. We are not concerned with the specific 
values as we are having faith in matlabs built in functions. We are only 
checking that we correctly glued the numerical parameters toegether to form the 
spectrum matrix. \\ 
These tests were built to enforce the requirement of finding the spectrum. This 
is because we are calculating the spectrum of the state variables in this 
module. \\ 

\begin{enumerate}				

	\item{test-SpecMAT-Tr1\\}
	
	Type: Automaic
	
	Initial State: 
	
	Input: Input Table.
	
	Output: assert=True
	
	Test Case Derivation: From the specification of the Spectrum Matrix Module 
	in the MIS 
	(\url{https://github.com/whitere123/CAS741_REW}) it follows that 
	the numerical derivative approximation matrices $NUM_{x}$ for 
	$x:(O(8),O(10),O(12))$ are equal to negative of their own transpose.
	
	How test will be performed: We will check to see if each matrix is 
	equal to neagtive its own transpose. This will be done by checking that 
	$assert( 
	SpecMat.NUM_{x} == -SpecMat.NUM_{x}^{T} ) $ is true for $x: ( O(8), 
	O(10), O(12) )$.  
	\wrw{see module guide for proper notation and references}
		
\end{enumerate} 

\subsubsection{Plotting Module} 

The following tests will involve inspection of the spectral plots by my thesis 
supervisor, Dr. Dmitry Pelinvosky. He will ensure that the plots fit the 
theoretical results, are useful for our research purposes and presented in a 
clear fashion. The first plot test involves inspecting the spectrum for 
moderate input values (ie, not close to their bounds). The second plot test 
will involve inspecting the behaviour of the plots near the bounds of the input 
parameter k. The list of qualities of an adequate plot is presented in 
\ref{specplotlist}. 

\begin{enumerate}				
	\item{test-Plotting-Inspect\\}
	
	Type: Manual
	
	Initial State: 
	
	Input: Output Parameters
	
	Output: 6 spectral plots on the complex plane
	
	Test Case Derivation: There are a total of six spectral matrices. Each 
	corresponds to one of two boundary solutions to the Non-Linear Schr\"{o} 
	dinger equation and one of three numerical algorithms. 
	
	How test will be performed: My supervisor will inspect the plots to see if 
	they resmble the theory from \cite{SegaletAl} He will also check if the 
	theoretical 
	values superimposed on the plot overlap with the appropriate numerical 
	eigenvalues.  

			
	\item{test-Plotting-Inspect-Bound\\}
	
	Type: Manual
	
	Initial State: 
	
	Input: Boundary Input
	
	Output: 6 spectral plots on the complex plane
	
	Test Case Derivation: Is the same as test-Plotting-Inspect. The only 
	exception is that the domain will have a very large bound for k close to 1.
	
	How test will be performed: My supervisor will inspect the plots to see if 
	they resmble the theory from \cite{SegaletAl} He will check if the 
	theoretical 
	values superimposed on the plot overlap with the appropriate numerical 
	eigenvalues. 
\end{enumerate}

\subsection{Tests for Nonfunctional Requirements}

Planning for nonfunctional tests of units will not be relevant to \progname. 
For system tests related to Nonfunctional requirements please see the System 
Verification and Validation Plan at 
\url{https://github.com/whitere123/CAS741_REW}. 

\subsection{Traceability Between Test Cases and Requirements}

This section shows the traceability between the modules and the test cases.

\label{traceability}
\begin{table}[h]
	\centering
	\begin{tabular}{|c|c|c|c|c|c|c|c|c|c|c|c|c|c|c|c|c|c|c|c|c|c|c|c|}
		\hline        
		& Rin& Rfind & Rplt  \\
		\hline
		test-InParams-N     &X & &   \\ \hline
		test-InParams-P    &X& &    \\ \hline
		test-InParams-k    &X & &   \\ \hline 
		test-NumParams-Dom    & &X &     \\ \hline 
		test-NumParams-Dom-Bound    & &X &    \\ \hline 
		test-NumParams-EllipMat    & &X &     \\ \hline 
		test-SpecMAT-Tr1  & &X &    \\ \hline 
		test-Plotting-Inspect & & &X     \\ \hline 
		test-Plotting-Inspect-Bound  & & &X   \\
		\hline
	\end{tabular}\\
	\caption{Traceability Between Test }
	\label{Table:D_1}
\end{table} 

\clearpage
\bibliographystyle {plainnat}
\bibliography {../../../ReferenceMaterial/References}

\newpage

\section{Appendix} 
\label{app}

\subsection{Software Verification Checklist} 
\label{softwarevercheck}
\begin{itemize}
	\item Did any of the inputs you entered provide suprising results? If yes, 
	what were they?
	\item Were you able to identify which numerical algorithm and wave solution 
	the plot represented? 
	\item Were all of the plots legible? 
	\item Was the output useful for your research? 
	\item Was it clear how to input the variables? 
\end{itemize}  

\subsection{Usability Survey Questions}
\label{UsabilitySurvey}
\begin{itemize}
	\item How long did it take before you could run the software? How many 
	attempts at running SpecSeach did it take before you understood how to 
	properly use it and interpret the output?
	\item Was this program useful for your research and were you able to 
	interpret the results? 
	\item What aspects of this software do you feel need improvement?
	\item How does this program compare with other software that finds this 
	particular spectrum? 
	\item Was it clear how and where to input the variables? 
	\item Were the plots and stability results clear? 
\end{itemize}  

\clearpage
\subsection{Boundary Value Analysis Table}
\begin{table}[h!]
	\centering
	\begin{tabular}{|c|c|c|c|c|c|c|c|c|c|c|c|c|c|c|c|c|c|c|c|c|c|c|c|}
		\hline        
		Input ID& k& N & P & Result \\
		\hline
		I1      &0.9 &100 &2 & Pass \\ \hline
		I2     &0.99 &120 &2 & Pass \\ \hline
		I3     &0.9999 &500 &2 & Pass\\ \hline 
		I4     &0.99999 &550 &2 & Pass \\ \hline 
		I5      &0.999999 &2 &4 & Pass \\ \hline
		I6     &0.01 &700 & 2& Pass \\ \hline
		I7    &0.0001 &100 & 2& Pass\\ \hline 
		I8     &0.000001 &400 &4& Pass \\ \hline
		I9      &0.0000001 &500 &4 & Pass \\ 
		\hline
	\end{tabular}\\
	\caption{Boundary Value Inputs}
	\label{Table:D_13}
\end{table} 

\subsection{Qualities of an adequate spectral plot} \label{specplotlist}

\begin{itemize}
	\item Most of the eigenvalues should be purely imaginary. Most of the 
	points should be on the vertical axis. 
	\item The non-purely imaginary eigenvalues for the dn-wave solutions should 
	be purely real.
	\item The non-purely imaginary eigenvalues for the cn-wave solutions should 
	have real and imaginary component. 
	\item All of the eigen values should be a complex conjugate pair.
	\item The non purely imaginary eigenvalues (from literature) are plotted 
	with an asterix. These asterixes should lie within one of the numerically 
	approximated eigenvalues, denoted by a hollow circle. 
\end{itemize}

\end{document}
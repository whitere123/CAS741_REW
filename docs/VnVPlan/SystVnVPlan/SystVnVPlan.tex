\documentclass[12pt, titlepage]{article}
\usepackage{amsmath}
\usepackage{booktabs}
\usepackage{tabularx}
\usepackage{hyperref}
\hypersetup{
    colorlinks,
    citecolor=black,
    filecolor=black,
    linkcolor=red,
    urlcolor=blue
}
\usepackage[round]{natbib}

%% Comments

\usepackage{color}

\newif\ifcomments\commentstrue

\ifcomments
\newcommand{\authornote}[3]{\textcolor{#1}{[#3 ---#2]}}
\newcommand{\todo}[1]{\textcolor{red}{[TODO: #1]}}
\else
\newcommand{\authornote}[3]{}
\newcommand{\todo}[1]{}
\fi

\newcommand{\wss}[1]{\authornote{blue}{SS}{#1}}
\newcommand{\an}[1]{\authornote{magenta}{Author}{#1}}



\begin{document}

\title{Project Title: System Verification and Validation Plan} 
\author{Robert White}
\date{\today}
	
\maketitle

\pagenumbering{roman}

\section{Revision History}

\begin{tabularx}{\textwidth}{p{3cm}p{2cm}X}
\toprule {\bf Date} & {\bf Version} & {\bf Notes}\\
\midrule
2018-10-13 & 1.0 & Creation of first draft for VnV plan presentation.\\
2018-10-20 & 1.1 & Edit of 1.0. I considered all of the feedback from my 
presentation and made necessary changes. I also prepared the document for 
submission. \\ 
2018-10-22 & 1.2 & Edit of 1.1. Removed comments, updated test cases, added 
references, improved writing and added test-NFR2c. \\ 
2018-10-23 & 1.3 & Edit of 1.2. Minor corrections and revisions. \\
\bottomrule
\end{tabularx}

~\newpage

\section{Symbols, Abbreviations and Acronyms}

\renewcommand{\arraystretch}{1.2}
\begin{tabular}{l l} 
  \toprule		
  \textbf{symbol} & \textbf{description}\\
  \midrule 
  T & Test\\
  R & Requirement\\ 
  NFR & Non-functional Requirement\\ 
  \bottomrule
\end{tabular}\\

\newpage

\tableofcontents

\listoftables

\listoffigures

\newpage

\pagenumbering{arabic}

This document discusses the verification and validation requirements for 
SpecSearch. The Project Management Body of Knowledge (PMBOK) guide provides 
unambigious definitions for verification and validation. PMBOK defines 
verification as ``the evaluation of whether or not a product, service, or 
system 
complies with a regulation, requirement, specification, or imposed condition. 
It is often an internal process." \\

PMBOK defines validation as ``the assurance that 
a product, service, or system meets the needs of the customer and other 
identified stakeholders. It often involves acceptance and suitability with 
external customers" [Project Management Institute, 2017]. After reading this 
document one should be able to create and run test cases to verify and validate 
SpecSearch. 

\section{General Information}

\subsection{Summary}

The software that is going to be tested is called SpecSearch. SpeacSearch will 
search for the 
spectrum (set of eigenvalues) of a particular lax equation from a lax pair that 
is compatbilible 
with solutions to the Non-Linear Schrodinger (NLS) Equation. SpecSearch will 
also use 
the spectral information to determine the stability of the solutions. Refer to 
the 
instance models in the SRS for more details.
\subsection{Objectives}

	The qualities that are most important for SpecSearch are highlighted in 
	this section. Many of these qualities are also presented in the SRS. \\
	
	My supervisor and I intend to use this program as a tool to search for the 
	continuous spectrum of the previously mentioned spectral problem. 
	SpecSearch 
	will only provide the approximate 
	location of the spectrum and a finite number of elements in the spectrum. 
	We will 
	use this output as a guide in analytically solving for the 
	entire continuous spectrum.  Therefore, the 
	code should be reliable and accurate within my supervisor's standards. \\
	% An unreliable and inaccurate output will provide a misleading direction 
	%for the next stage of our research. 
	
	There are various numerical methods that can be used to approximate the 
	derivatives of the eigenfunctions [Graselli and Pelinovksy, 2007]. My 
	supervisor and I intend to experiment with these different 
	methods in an attempt to create a more accurate picture of the spectrum. 
	Therefore, it is important that SpecSeach is maintainable and 
	manageable. \\
	% This will allow for easy implementation of new algorithms.
	
	 SpecSearch will also be used by a team of researchers studying modulated 
	 wave packets and rogue waves. These users may not have a strong software 
	 developing background. Therefore, it is important that the code is easy to 
	 use and has a simple user interface. \\
	 
	 The objectives are summarized in the following points for easy reference. 
	 The objectives are to:
	
\begin{itemize}
	\item Build confidence in software correctness.
	\item Ensure maintainability and manageability 
	\item Satisfy the requirements of my thesis supervisor and those outlined 
	in the SRS. 
	\item Verify usability.	
\end{itemize}

\subsection{References}

\begin{itemize} 

	\item 	Bernard Deconinck and Benjamin L.Segal. 
	The stability spectrum for elliptic solutions to the focusing NLS equation. 
	PhysicaD, 2017.  
	\item Robert White. System Requirements Specification for SpecSearch. 
	Github, 2018.
	\item Matheus Grasselli and Dmitry Pelinovsky. Numerical Mathematics. Jones 
	and Bartlett Learning, 2007.
	
\end{itemize}

\section{Plan}
	
\subsection{Verification and Validation Team}

The verification and validation team consists of my thesis supervisor, Dr. 
Dmitry Pelinovsky, and I. \\

\subsection{SRS Verification Plan}

The SRS verification plan consists of feedback from Dr. Dmitry Pelinovsky 
(supervisor), Dr. Spencer Smith and CAS 741 classmates. My supervisor will 
provide feedback regarding mathematical theory, model assumptions, constraints 
and research goals. Feedback from classmates and Dr. Smith will criticize the 
document outline, readability and requirements. 

\subsection{Design Verification Plan}

The design verification plan will simply involve inspection of the software by 
my thesis supervisor. 

\subsection{Implementation Verification Plan}

The implementation verification plan consists of two parts. The first part 
is a software verification checklist. The checklist will be completed by 
researchers , Dr. Pelinovksy and I. The 
checklist verifies that basic software features have been implemented 
successfully. 
For example, it checks if the user was able to fufill their responsibilities 
outlined in the SRS. 
This checklist can be found in the appendix (7.4).\\

 The 
second part involves running the software tests 
outlined in sections 5 and 6. Unit testing will also be performed. 

\subsection{Software Validation Plan}

This section does not apply to SpecSearch. 

\section{System Test Description}
	
\subsection{Tests for Functional Requirements}

\begin{enumerate}
\item{test-Rin1\\}	
				
Initial State: -
					
Input: Table 2 in Appendix 7.3.
					
Output: Error or pass message.
					
How test will be performed: Combinations of inputing non-numerical values as 
input  (such as letters), or numerical values outside of their respective 
constraints, will be considered. A successful test in these instances will be 
an error message. \\
I will also test cases with each variable in the input having an 
acceptable numerical value. A successful test in these cases will be a pass 
message. 
					
\item{test-Rfind1} 

Initial State: - 

Input: Table 2 in Appendix 7.3 (Similar to test-Rin1).

Output: Size of spectrum array 

How the test will be performed: Passing inputs from test-Rin1 should produce 
$4n$ eigenvalues. Unsuccessful cases 
from test-Rin1 should produce no eigenvalue array 
or an empty eigenvalue array. We are not testing for accuracy in this test.

\item{test-Rcon1} 

Initial State: -

Input: Table 2 in Appendix 7.3 (Similar to test-Rin1).

Output: Approximated Spectrum (Connected or disconnected)

How test will be performed: This test will check to see if there is a 
sufficient amount of points between the tagged portions of the spectrum. Tagged 
portions are the explicitly calculated eigenvalues from the previous test. The 
spectrum array should have $2*m-1$ approximated values between these "tagged 
portions", where m is the scaling parameter. 
Failed test cases from test-Rin1 should have no data for this section. 
We are not testing for accuracy in this test.\\
 The scaling parameter has not 
been studied yet. It will be incorporated in the final draft of the documents.

\item{test-Rplt} 

Initial State: -

Input: Table 2 in Appendix 7.3 (Similar to test-Rin1).

Output: A plot of the spectrum.

How test will be performed: This will involve visual inspection by my 
supervisor to see if the spectrum plot is adequate. \\
There should be no plot for the failed test cases in test-Rin1. 

\item{test-Rstl} 

Initial State: -

Input: Table 2 in Appendix 7.3 (Similar to test-Rin1).

Output: Binary Variable (Verification of stability) 

How test will be performed: The stability results will be compared with the 
stability analysis in [Deconinck and Segal, 2017].  My supervisor 
and I are still working on test cases for Rstl.Test-Rstl will be updated in 
the final draft of the documents. 

\end{enumerate}

\subsection{Tests for Nonfunctional Requirements}

\begin{enumerate}

\item{test-NFR1\\}

Type: Static
					
Initial State: -
					
Input/Condition: SpecSearch MATLAB code
					
How test will be performed: The software will be manually read by Dr. 
Pelinovksy and I to see if there is a more effective code structure to allow 
implementation of new numerical algorithms. 

\item{test-NFR2a\\} 

Type: Manual 

Initial State: -

Input: Integer n (same as test-Rin1).

Output: Operator Matrix.

How test will be performed: Test-NFR2a will check if the operator matrix is of 
the correct form. See Appendix 7.5 for more detail. The values from table 2 
will be used to inspect A1.   \\

\item{test-NFR2b\\} 

Type: Manual 

Initial State: -

Input: Table 2 in Appendix 7.3 (Similar to test-Rin1).

Output: Spectrum.

How test will be performed: The output will be tested against the boundary 
value eigen-values derived analytically in Deconinck and Segal. See details 
below table 2 of appendix 7.3. \\ 

\item{test-NFR2c\\} 

Type: Usability Survey (see appendix 7.2) 

How test will be performed: The survey will be administered to the future users 
of SpecSearch. Feedback from the survey will be considered when updating 
SpecSearch. The survey is in appendix 7.2. 

\end{enumerate}

\newpage
\subsection{Traceability Between Test Cases and Requirements}

\begin{table}[h]
	\centering
	\begin{tabular}{|c|c|c|c|c|c|c|c|c|c|c|c|c|c|c|c|c|c|c|c|c|c|c|c|}
		\hline        
		& Rin& Rfind & Rcon & Rplt & Rstl & NFR1 & NFR2 \\
		\hline
		test-Rin     &X & & & & & &  \\ \hline
		test-Rfind    & &X & & & & &  \\ \hline
		test-Rcon    & & &X & & & & \\ \hline 
		test-Rplt    & & & &X & & &   \\ \hline 
		test-Rstl    & & & & &X & &   \\ \hline 
		test-NFR1    & & & & & &X &   \\ \hline 
		test-NFR2a  & & & & & & &X   \\ \hline 
		test-NFR2b  & & & & & & &X   \\ \hline 
		test-NFR2c  & & & & & & &X \\
		\hline
	\end{tabular}\\
	\caption{Traceability Between Test }
	\label{Table:D_1}
\end{table} 


\section{Static Verification Techniques}

\begin{itemize}
	\item Code inspection : I will 
	go through the code to see if each part is correct with respect to the
	mathematical theory. In particular I will ensure that: 
	\begin{itemize}
		\item Variables are being used in the right context. 
		\item Functions from other packages are being used in the right 
		context. Some packages have different standards for 
		constants. For instance, one of the complete elliptic integral 
		functions in matlab does not square the inputted variable. Theoretical 
		convention dictates that this constant should be squared. 
		\item The dimensions of the vectors and matrices are appropriate. For 
		example, multiplication of row with column versus column with row.
		\item Equations are translated correctly into matlab syntax.  
	\end{itemize} 
	\item Code walkthrough: My supervisor and I will go through the code 
	together to ensure that: 
	\begin{itemize}
		\item I correctly implemented the mathematical theory and numerical 
		algorithms.
		\item I made the code manageable and maintainable for future use.
	\end{itemize}
\end{itemize}
				
\bibliographystyle{plainnat}

\newpage 
\begin{thebibliography}{9} 
	\bibitem{latexcompanion} 
	Project Management Institute. 
	A Guide to the Project Management Body of Knowledge (PMBOK Guide)–Sixth 
	Edition. 
	Project Management Institute, 2017. 
	
	\bibitem{latexcompanion} 
	Caswell and Johnston. 
	Design Verification. 
	University of Calary. 2017. 
	
	\bibitem{latexcompanion} 
	Bernard Deconinck and Benjamin L.Segal. 
	The stability spectrum for elliptic solutions to the focusing NLS equation. 
	PhysicaD, 2017. 
	 
	\bibitem{latexcompanio} 
	Robert White. 
	System Requirements Specification for SpecSearch. 
	Github, 2018. 
	
	\bibitem{latexcompanio} 
	Matheus Grasselli and Dmitry Pelinovsky. 
	Numerical Methods. 
	Jones and Bartlett Learning; 1 edition, Aug. 13 2007
	
\end{thebibliography} 

%http://people.ucalgary.ca/~design/engg251/First%20Year%20Files/design_verif.pdf

\newpage

\section{Appendix}

\subsection{Symbolic Parameters}

\subsection{Usability Survey Questions}

\begin{itemize}
	\item How long did it take before you could run the software? How many 
	attempts at running SpecSeach did it take before you understood how to 
	properly use it and interpret the output?
	\item Was this program useful for your research and were you able to 
	interpret the results? 
	\item What aspects of this software do you feel need improvement?
	\item How does this program compare with other software that finds this 
	particular spectrum? 
	\item Was it clear how and where to input the variables? 
	\item Were the plots and stability results clear? 
\end{itemize} 

\newpage
\subsection{Data Input Tables} 

The following data table is for test-Rin1. An "X" indicates no-input or 
incorrect input. Incorrect input is anything that is non-numerical (ie letters 
or symbols). 

\begin{table}[h!]
	\centering
	\begin{tabular}{|c|c|c|c|c|c|c|c|c|c|c|c|c|c|c|c|c|c|c|c|c|c|c|c|}
		\hline        
		Input ID& a& b & k & n& Result \\
		\hline
		I1     &1 &1 &0.9 &100 & Pass \\ \hline
		I2    &X &1 &0.9 &10 & Error \\ \hline
		I3    &1 &X &0.9 &100 & Error\\ \hline 
		I4    &1 &1 &X &100 &Error \\ \hline 
		I5     &1 &1 &0.9 &X &Error \\ \hline
		I6    &1 &1 &1 & 1.2& Error\\ \hline
		I7    &1 &1 &1.2 & 100&Error\\ \hline 
		I8    &1 &-1 &0.9 &100& Error \\ \hline
		I9     &-1 &2 &0.9 &100 & Pass \\ \hline
		I10    &1 &2 &0.99 &100 & Pass\\ \hline
		I11    &1 &3 &0.99 &500 & Pass \\ \hline 
		I13   &1 &1 &0 &150 & Pass  \\ \hline
		I14     &-1 &2 &0.9 &330 & Pass \\ \hline
		I15    &1 &1 &0.99 &170 & Pass\\ \hline
		I16    &1 &5 &0.999 &590 & Pass \\ \hline 
		I17   &1 &1 &0 &200 & Pass  \\ 
		\hline
	\end{tabular}\\
	\caption{Prescribed input for testing Rin}
	\label{Table:D_1}
\end{table} 

The following table outlines the boundary value test cases that can be verified 
from previous literature. A 'V' means that the parameter is free to vary.
\begin{table}[h!]
	\centering
	\begin{tabular}{|c|c|c|c|c|c|c|c|c|c|c|c|c|c|c|c|c|c|c|c|c|c|c|c|}
		\hline        
		Boundary ID& a& b& k  \\
		\hline
		B1     &$\sqrt{b}$ &$a^{2}$ &V    \\ \hline
		B2    &$\sqrt{\frac{b}{k^{2}}}$ &$a^{2}k^{2}$ &V   \\ \hline
		B3    &V &V &0  \\ 
		\hline
	\end{tabular}\\
	\caption{Prescribed input for Rin}
	\label{Table:D_2}
\end{table} 

For these cases all of the eigenvalues, except for four of them, will be purely 
imaginary. The remaing four have the form: $\lambda = \frac{+}{-} \frac{1}{2} a 
(1+\sqrt{1-k^{2}})$ and $\lambda = \frac{+}{-} \frac{1}{2} a 
(1-\sqrt{1-k^{2}})$ for B1 (see table 3). \\ 
As for B2, the eigenvalues will have the form:  $\lambda =  \frac{1}{2} a 
(\frac{+}{-}k+ i \sqrt{1-k^{2}})$ and $\lambda = \frac{1}{2} a 
(\frac{+}{-}k - i \sqrt{1-k^{2}})$. [Deonick and segal 2017] \\ 
The plan is to run B1 with a=b=1 and k=0.999. B2 will be run with the same 
input. The expected eigenvalues were previously mentioned. 

\subsection{Software Verification Checklist} 

\begin{itemize}
	\item Does the software allow the user to fufill their responsibilities 
	(as outlined in the SRS)?
	\item Does the software fufill its intended responsibilities (see SRS)?
	\item Was the intended user able to use the software and interpret the 
	output for their research purposes?
	\item Are any of the system constraints violated (see SRS)?
\end{itemize}  

\newpage	
\subsection{Correct Matrix Form} 

test-NFR2a will ensure that the matrix for which the eigen values are 
calculated, E, is of the correct form. For any given input n, the matrix is 4n 
by 4n. It is of the form $ [A1 A2; A2 (-A1)] $ where $ A1, A2$ is the first 
row 
and $A2 -A1$ is the second row. A1 and A2 are matrices. A2 is a diagonal 2n by 
2n matrix and A1 is 2n by 2n and defined as follows: A1(i,j) = n/2 for 
(1,2),(2,3),... (2n-1,2n) and A1(i,j)=-n/2 for (2,1),(3,2) ... (2n,2n-1). 
A1(1,2n)=-n/2 and A1(2n,1)=n/2. // 

For example for arbitrary n it should have the form: 
\[
A1=
\begin{bmatrix}
0 & \frac{n}{2} & 0 & 0 & 0 &-\frac{n}{2}\\
-\frac{n}{2} & 0 & \frac{n}{2} & 0 & 0 &0\\
0 & -\frac{n}{2} & 0 & \frac{n}{2} & 0 &0\\
0 & 0 & -\frac{n}{2} & 0 & \frac{n}{2} &0\\
0 & 0 & 0 & -\frac{n}{2} & 0 &\frac{n}{2} \\ 
\frac{n}{2} & 0 & 0 & 0 & -\frac{n}{2} & 0 
\end{bmatrix}
\] 

\end{document}
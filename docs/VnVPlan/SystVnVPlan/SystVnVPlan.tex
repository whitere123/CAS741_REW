\documentclass[12pt, titlepage]{article}
\usepackage{amsmath}
\usepackage{booktabs}
\usepackage{tabularx}
\usepackage{hyperref}
\hypersetup{
    colorlinks,
    citecolor=black,
    filecolor=black,
    linkcolor=red,
    urlcolor=blue
}

\usepackage[round]{natbib}

%% Comments

\usepackage{color}

\newif\ifcomments\commentstrue

\ifcomments
\newcommand{\authornote}[3]{\textcolor{#1}{[#3 ---#2]}}
\newcommand{\todo}[1]{\textcolor{red}{[TODO: #1]}}
\else
\newcommand{\authornote}[3]{}
\newcommand{\todo}[1]{}
\fi

\newcommand{\wss}[1]{\authornote{blue}{SS}{#1}}
\newcommand{\an}[1]{\authornote{magenta}{Author}{#1}}


%% Common Parts

\newcommand{\progname}{SpecSearch} % PUT YOUR PROGRAM NAME HERE %Every program
                                % should have a name


\begin{document}

\title{\progname: System Verification and Validation Plan} 
\author{Robert White}
\date{\today}
	
\maketitle

\pagenumbering{roman}

\section{Revision History}

\begin{tabularx}{\textwidth}{p{3cm}p{2cm}X}
\toprule {\bf Date} & {\bf Version} & {\bf Notes}\\
\midrule
2018-10-13 & 1.0 & Creation of first draft for VnV plan presentation.\\
2018-10-20 & 1.1 & Edit of 1.0. I considered all of the feedback from my 
presentation and made necessary changes. I also prepared the document for 
submission. \\ 
2018-10-22 & 1.2 & Edit of 1.1. Removed comments, updated test cases, added 
references, improved writing and added test-NFR2c. \\ 
2018-10-23 & 1.3 & Edit of 1.2. Minor corrections and revisions. \\
2018-11-28 & 1.4 & Edit of 1.3. First wave of corrections for final 
documenation.\\ 
2018-12-09 & 1.5 & Creation of final draft for final documentation. \\
\bottomrule
\end{tabularx}

~\newpage

\section{Symbols, Abbreviations and Acronyms}

\renewcommand{\arraystretch}{1.2}
\begin{tabular}{l l} 
  \toprule		
  \textbf{symbol} & \textbf{description}\\
  \midrule 
  T & Test\\
  R & Requirement\\ 
  NFR & Non-functional Requirement\\
  \bottomrule
\end{tabular}\\

Refer to the SRS Symbols, Abbreviations and Acronyms for a more 
complete list (\cite{SRS}) \url{https://github.com/whitere123/CAS741_REW}. 


\newpage

\tableofcontents

\listoftables

%\listoffigures

\newpage

\pagenumbering{arabic}

This document discusses the verification and validation requirements for 
\progname. The Project Management Body of Knowledge (PMBOK) guide 
provides 
unambigious definitions for verification and validation (\cite{PMBOK}). PMBOK 
defines 
verification as ``the evaluation of whether or not a product, service, or 
system 
complies with a regulation, requirement, specification, or imposed condition. 
It is often an internal process." \\

PMBOK defines validation as ``the assurance that 
a product, service, or system meets the needs of the customer and other 
identified stakeholders. It often involves acceptance and suitability with 
external customers" (\cite{PMBOK}). After reading this 
document one should be able to create and run test cases to verify and validate 
\progname. 

\section{General Information}

\subsection{Summary}
\label{Summary} 

The software that is going to be tested is called \progname. \progname \ will 
search for the 
spectrum (set of eigenvalues) of a particular lax equation from a lax pair that 
is compatible 
with solutions to the Non-Linear Schr\"{o}dinger (NLS) Equation. We are only 
concerned with the spectrum of two particular solutions. Refer to the 
instance model in the SRS (\cite{SRS}) for more details.
\subsection{Objectives}
\label{Objectives}
	The qualities that are most important for \progname \ are highlighted in 
	this section. Many of these qualities are described in the SRS
	(\cite{SRS}). \\
	
	My supervisor, Dr$.$ Dmitry Pelinovsky, and I intend to use this program 
	as 
	a tool to search for the 
	 spectrum of the previously mentioned spectral problem. 
	\progname \
	will only provide the approximate 
	location of the spectra for two particular wave solutions of the NLS 
	equation. In addition, it 
	can only find a finite number of elements in the spectra. 	
	We will 
	use this output as a guide in analytically solving for the 
	entire continuous spectrum of general traveling periodic waves.  Therefore, 
	the 
	code should be reliable and accurate within my supervisor's standards. \\
	% An unreliable and inaccurate output will provide a misleading direction 
	%for the next stage of our research. 
	
	There are various numerical methods that can be used to approximate the 
	derivatives of the eigenfunctions, \cite{graspel}. My 
	supervisor and I intend to experiment with these different 
	methods in an attempt to create a more accurate picture of the spectrum. 
	Therefore, it is important that \progname \ is maintainable and 
	manageable. \\
	% This will allow for easy implementation of new algorithms.
	
	 \progname \ will also be used by a team of researchers studying rogue 
	 waves. These users may not have a strong software 
	 developing background. Therefore, it is important that the code is easy to 
	 use and has a simple user interface. \\
	 
	 The objectives are summarized in the following points for easy reference. 
	 The objectives are to:
	
\begin{itemize}
	\item Build confidence in software correctness.
	\item Ensure maintainability and manageability 
	\item Satisfy the requirements of my thesis supervisor and those outlined 
	in the SRS (\cite{SRS}). 
	\item Verify usability.	
\end{itemize}

\subsection{References}
\label{REF1}
The following reference will be used to build confidence in software 
correctness. This research paper explores the same spectrum that my supervisor 
and I are studying. They were successful at analytically finding the set of 
eigenvalues for particular boundary conditions. These eigenvalues will be 
plotted alongside our numerical spectrum. These theoretical results define the 
standard of our software correctness. Our confidence in software correctness 
will be higher if our spectrum overlaps with these theorectical results. 
\begin{itemize} 
	\item 	Bernard Deconinck and Benjamin L.Segal (\cite{SegaletAl}) 
	The stability spectrum for elliptic solutions to the focusing NLS equation. 
	PhysicaD, 2017.  
\end{itemize} 
The following reference will be frequently mentioned in this document. The 
test cases outlined in this document are meant to test the necessary qualities 
and requirements outlined in the SRS (\cite{SRS}). 
\begin{itemize}
	\item Robert White. System Requirements Specification for SpecSearch 
	(\cite{SRS}). 
	Github, 2018. 
\end{itemize}  

\section{Plan}
	 
\subsection{Verification and Validation Team}
\label{VnVTeam}
The verification and validation team consists of my thesis supervisor, Dr. 
Dmitry Pelinovsky and I, Robert White. \\

\subsection{SRS Verification Plan}
\label{SRSVerPlan}
The SRS verification plan consists of feedback from Dr. Dmitry Pelinovsky 
(supervisor), Dr. Spencer Smith and CAS 741 classmates. My supervisor will 
provide feedback regarding mathematical theory, model assumptions, constraints 
and research goals. Feedback from classmates and Dr. Smith will criticize the 
document outline, readability and requirements. In particular, the contents of 
this document will be reviewed by Jennifer Garner 
\url{https://github.com/PeaWagon/Kaplan}. 

\subsection{Design Verification Plan}
\label{DesignVerificationPlan}
The design verification plan will simply involve inspection of the software by 
my thesis supervisor, Dr$.$ Smith and CAS 741 classmates.  

\subsection{Implementation Verification Plan}
\label{ImplementationVerPlan}
The implementation verification plan consists of three parts. The first part 
is a software verification checklist. The checklist will be completed by 
researchers, Dr$.$ Pelinovsky and I. The checklist verifies that basic 
software features have been implemented successfully. 
For example, it checks if the user was able to fulfill their responsibilities 
outlined in the SRS (\cite{SRS}). 
This checklist can be found in the ~\ref{softwarevercheck}.\\

 The second part involves running the software tests outlined in sections 
 \ref{FRtests} and 
 \ref{app}. Unit testing will also be performed. \\
 
The third part is summarized as follows: 
 \label{staticVertech}
 \begin{itemize}
 	\item Code inspection : I will 
 	go through the code to see if each part is correct with respect to the
 	mathematical theory. In particular I will ensure that: 
 	\begin{itemize}
 		\item Variables are being used in the right context. 
 		\item Functions from other packages are being used in the right 
 		context. Some packages have different standards for 
 		constants. For instance, one of the complete elliptic integral 
 		functions in MATLAB does not square the inputted variable. Theoretical 
 		convention dictates that this constant should be squared. 
 		\item The dimensions of the vectors and matrices are appropriate. For 
 		example, multiplication of row with column versus column with row.
 		\item Equations are translated correctly into MATLAB syntax.  
 	\end{itemize} 
 	\item Supervisor Inspection: My supervisor and me will go through the code 
 	together to ensure that: 
 	\begin{itemize}
 		\item I correctly implemented the mathematical theory and numerical 
 		algorithms.
 		\item I made the code manageable and maintainable for future use.
 	\end{itemize} 
 \end{itemize}

\subsection{Software Validation Plan}
\label{SoftwareValidationPlan}
The first reference in \ref{REF1} will be used to validate \progname. The 
boundary value inputs from this will be checked against the output of 
\progname. This test is outlined in test-find of \ref{Rfind}. 

\newpage
\section{System Test Description}
	
\subsection{Tests for Functional Requirements}
\label{FRtests}
\begin{enumerate}
\item{test-Rin1NonNumeric\\}	
				
Initial State: -
					
Input: Table ~\ref{Table:D_11} in ~\ref{datainput}.
					
Output: Exception (Non-Numeric) 
					
How test will be performed: Combinations of inputing non-numerical values for 
different 
inputs  (such as strings) will be considered. A non-numerical value is denoted 
by 'X' in \ref{Table:D_11}. These tests should catch an exception. 'X' will 
only be a string, character or vector of dimension greater than 2. Other 
data-type mismatches are handled by MATLAB.\\

\item{test-Rin1kBounds\\}	

Initial State: -

Input: I1 to I9 of Table ~\ref{Table:D_12} in ~\ref{datainput}.

Output: Exception (k out of constraint).

How test will be performed: Variations of k being out of its respective bound 
will be considered in each test. I will test negative k values, values on the 
endpoints and values greater than 1. The variable k should be contained in 
$(0,1)$ (see 
SRS for more details (\cite{SRS})). A successful test will involve catching 
an exception. This test is ensuring that Rin is satisfied (\cite{SRS}).\\ 

\item{test-Rin1NBounds\\}	

Initial State: -

Input: I10 to I14, and I19 of Table ~\ref{Table:D_12} in ~\ref{datainput}.

Output: Exception (N out of constraint).

How test will be performed: Variations of N being out of its respective bound 
will be considered in each test. I will test fractions and negative numbers. 
The variable N should be a natural number. A successful test will involve 
catching an exception. This test is ensuring that Rin is satisfied 
(\cite{SRS}). \\  

\item{test-Rin1PBounds\\}	

Initial State: -

Input: I15 to I18 of Table ~\ref{Table:D_12} in ~\ref{datainput}.

Output: Exception (P out of constraint).

How test will be performed: Variations of P being out of its respective bound 
will be considered in each test. I will test 0, unallowable positive numbers, 
and negative numbers. The variable P 
can only be 2 or 4. A successful test will involve catching an exception. This 
test is ensuring that Rin is satisfied (\cite{SRS}). \\
					
\item{test-Rfind-Rin\\} 
\label{Rfind}
Initial State: - 

Input: Table ~\ref{Table:D_13} in ~\ref{datainput}

Output: $4N$

How the test will be performed: $4N$ eigenvalues should be returned. Having 
this many eigenvalues is evidence that Rfind is being satisfied (\cite{SRS}). 
This 
test is also checking whether or not the code ran smoothly when given inputs 
within their constraints. Eigenvalues are only calculated when the other 
modules do not throw exceptions. Therefore, the functional requirement Rin is 
also being validated (\cite{SRS}). \\

\item{test-Rplt} 

Initial State: -

Input: Table ~\ref{Table:D_13} in ~\ref{datainput} .

Output: A plot of the spectrum.

How test will be performed: Numerical parameters within their respective 
constraints will be inputted into \progname. The program should run smoothly 
and generate six plots; a cn and dn spectrum for each of the three numerical 
algorithms. This test will involve visual 
inspection by my supervisor to see if the spectrum plots are adequate. This 
test is ensuring that the functional requirement Rplt is satisfied.  \\
 

\end{enumerate}

\subsection{Tests for Nonfunctional Requirements}
\label{NFRtests}
\begin{enumerate}

\item{test-NFR1\\}

Type: Static
					
Initial State: -
					
Input/Condition: SpecSearch MATLAB code
					
How test will be performed: The software will be manually read by Dr. 
Pelinovsky and I to see if there is a more effective code structure to allow 
implementation of new numerical algorithms. The purpose of this test is to 
ensure that the code is maintainable and manageable. This is related NFR1 in 
the 
SRS (\cite{SRS}). 

\item{test-NFR2\\} 

Type: Manual 

Initial State: -

Input: Table ~\ref{Table:D_13} in ~\ref{datainput} .

Output: Six Spectrum plots.

How test will be performed: The numerically calculated eigenvalues from 
\progname \ will be expressed as unfilled circles on a complex plane. Four 
theoretical eigenvalues from (segal et al) will be plotted their appropriate 
figure. These theoretical eigenvalues will be denoted by an asterix. My 
supervisor will inspect these plots to see if the \progname is accurate. This 
is related to NFR2 in the SRS (\cite{SRS}). The theoretical eigenvalues should 
lie 
within the circles. \\ 

\item{test-UserPerformance\\} 

Type: Usability Survey (see ~\ref{UsabilitySurvey}) 

How test will be performed: The survey will be administered to the future users 
of SpecSearch. Feedback from the survey will be considered when updating 
SpecSearch. The survey is in ~\ref{UsabilitySurvey}. This test is also related 
to NFR2. 

\end{enumerate}

\newpage
\subsection{Traceability Between Test Cases and Requirements}
\label{traceability}
\begin{table}[h]
	\centering
	\begin{tabular}{|c|c|c|c|c|c|c|c|c|c|c|c|c|c|c|c|c|c|c|c|c|c|c|c|}
		\hline        
		& Rin& Rfind & Rplt & NFR1 & NFR2 \\
		\hline
		test-Rin1NonNumeric     &X & & & &  \\ \hline
		test-Rin1kBounds    & X& & & &   \\ \hline
		test-Rin1NBounds    &X & & & &  \\ \hline 
		test-Rin1PBounds    &X & & & &    \\ \hline 
		test-RFind-Rin    &X &X & & &   \\ \hline 
		test-Rplt    & & &X & &    \\ \hline 
		test-NFR1  & & & &X &   \\ \hline 
		test-NFR2  & & & & &X    \\ \hline 
		test-UserPerformance  & & & & &X  \\
		\hline
	\end{tabular}\\
	\caption{Traceability Between Test }
	\label{Table:D_1}
\end{table} 
				

\clearpage
\bibliographystyle {plainnat}
\bibliography {../../../ReferenceMaterial/References} 

%\begin{thebibliography}{9} 
%	\bibitem{latexcompanion} 
%	Project Management Institute. 
%	A Guide to the Project Management Body of Knowledge (PMBOK Guide)–Sixth 
%	Edition. 
%	Project Management Institute, 2017. 
%	
%	\bibitem{latexcompanion} 
%	Caswell and Johnston. 
%%	University of Calary. 2017. 
	
%	\bibitem{latexcompanion} 
%	Bernard Deconinck and Benjamin L.Segal. 
%	The stability spectrum for elliptic solutions to the focusing NLS equation. 
%	PhysicaD, 2017. 
	 
%	\bibitem{latexcompanion} 
%	Robert White. 
%	System Requirements Specification for SpecSearch. 
%	Github, 2018. 
	
%	\bibitem{latexcompanion} 
%	Numerical Methods. 
%	Jones and Bartlett Learning; 1 edition, Aug. 13 2007
	
%\end{thebibliography} 

\newpage

\section{Appendix} \label{app}

\subsection{Usability Survey Questions}
\label{UsabilitySurvey}
\begin{itemize}
	\item How long did it take before you could run the software? How many 
	attempts at running SpecSeach did it take before you understood how to 
	properly use it and interpret the output?
	\item Was this program useful for your research and were you able to 
	interpret the results? 
	\item What aspects of this software do you feel need improvement?
	\item How does this program compare with other software that finds this 
	particular spectrum? 
	\item Was it clear how and where to input the variables? 
	\item Were the plots and stability results clear? 
\end{itemize} 

\newpage
\subsection{Data Input Tables} 
\label{datainput}
The following data tables list the inputs and expected outputs for the 
functional requirements test \ref{FRtests}. Table \ref{Table:D_11} consists of 
inputs with non-numerical values. A non-numerical value is either a string, 
character or 2+ dimensional vector. Table \ref{Table:D_12} lists inputs with 
one of the values being out of their respective constraint set. Table 
\ref{Table:D_13} is a list of inputs for which no exception is thrown. Each of 
these test cases should produce six plots and $4N$ eigenvalues. 

\begin{table}[h!]
	\centering
	\begin{tabular}{|c|c|c|c|c|c|c|c|c|c|c|c|c|c|c|c|c|c|c|c|c|c|c|c|}
		\hline        
		Input ID& k& N & P & Result \\
		\hline
		I1    &X &X &X & Error \\ \hline
		I2    &X &30 &2 & Error\\ \hline 
		I3    &X &X &2 &Error \\ \hline 
		I4     &X &100 &X &Error \\ \hline
		I5     &0.5 &X & 2& Error\\ \hline
		I6    &0.8 &X &X & Error  \\ \hline
		I7      &X &X &4 & Error \\ \hline
		I8     &X &1000 &X & Error \\ \hline
		I9     &0.99 &X &X & Error \\ \hline 
		I10    &0.6 &250 &X & Error \\ 
		\hline
	\end{tabular}\\
	\caption{Combinations of non-numerical input}
	\label{Table:D_11}
\end{table} 

\begin{table}[h!]
	\centering
	\begin{tabular}{|c|c|c|c|c|c|c|c|c|c|c|c|c|c|c|c|c|c|c|c|c|c|c|c|}
		\hline        
		Input ID& k& N & P & Result \\
		\hline
		I1     &1 &100 &2 & Error \\ \hline
		I2     &0 &100 &4 & Error\\ \hline 
		I3     &-10 &50 &2 &Error \\ \hline 
		I4      &-55 &200 &4 &Error \\ \hline
		I5     &2 &150 & 2& Error\\ \hline
		I6     &1.0001 & 100 &2 & Error \\ \hline
		I7     &-0.001 & 200 &4 & Error\\ \hline 
		I8     &1.01 & 200 &2 &Error \\ \hline 
		I9      &-0.01 &100 &4 &Error \\ \hline
		I10     &0.7 & -10 &2 & Error \\ \hline
		I11     &0.9 & 20 &4 & Error\\ \hline 
		I12     &0.9 & 5 &2 &Error \\ \hline 
		I13      &0.7 &0 &4 &Error \\ \hline
		I14     &0.5 & -1 & 2& Error\\ \hline 
		I15    &0.9 & 100 &0 & Error\\ \hline 
		I16     &0.9 & 500 &-2 &Error \\ \hline 
		I17      &0.7 &150 &-4 &Error \\ \hline
		I18     &0.5 & 200 & 100& Error\\ \hline 
		I19     &0.7 & 0.5 & 100& Error \\
		\hline
	\end{tabular}\\
	\caption{One of the inputs is out of their respective bound}
	\label{Table:D_12}
\end{table}  

\begin{table}[h!]
	\centering
	\begin{tabular}{|c|c|c|c|c|c|c|c|c|c|c|c|c|c|c|c|c|c|c|c|c|c|c|c|}
		\hline        
		Input ID& k& N & P & Result \\
		\hline
		I1      &0.6 &100 &2 & Pass \\ \hline
		I2     &0.1 &120 &2 & Pass \\ \hline
		I3     &0.9 &500 &2 & Pass\\ \hline 
		I4     &0.88 &550 &2 & Pass \\ \hline 
		I5      &0.99 &200 &2 & Pass \\ \hline
		I6     &0.65 &700 & 2& Pass \\ \hline
		I7    &0.4 &100 & 2& Pass\\ \hline 
		I8     &0.8 &400 &4& Pass \\ \hline
		I9      &0.9 &500 &4 & Pass \\ \hline
		I10     &0.2 &700 &4 & Pass\\ \hline
		I11     &0.3 &200 &4 & Pass \\ \hline 
		I13    &0.8 &100 &4 & Pass  \\ \hline
		I14      &0.89 &150 &4 & Pass \\ \hline
		I15     &0.69 &500 &4 & Pass\\ \hline
		I16     &0.55 &300 &2 & Pass \\ \hline 
		I17    &0.9 &400 &4 & Pass  \\ 
		\hline
	\end{tabular}\\
	\caption{All inputs within constraints}
	\label{Table:D_13}
\end{table} 

\clearpage
\subsection{Software Verification Checklist} 
\label{softwarevercheck}
\begin{itemize}
	\item Did any of the inputs you entered provide suprising results? If yes, 
	what were they?
	\item Were you able to identify which numerical algorithm and wave solution 
	the plot represented? 
	\item Were all of the plots legible? 
	\item Was the output useful for your research? 
	\item Was it clear how to input the variables? 
\end{itemize}  


\end{document}
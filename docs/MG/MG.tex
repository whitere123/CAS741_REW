\documentclass[12pt, titlepage]{article}

\usepackage[latin1]{inputenc}
\usepackage{tikz}
\usetikzlibrary{shapes,arrows}

\usepackage{fullpage}
\usepackage[round]{natbib}
\usepackage{multirow}
\usepackage{booktabs}
\usepackage{tabularx}
\usepackage{graphicx}
\usepackage{float}
\usepackage{hyperref}
\hypersetup{
    colorlinks,
    citecolor=black,
    filecolor=black,
    linkcolor=red,
    urlcolor=blue
}

\input{../Comments}

\newcounter{acnum}
\newcommand{\actheacnum}{AC\theacnum}
\newcommand{\acref}[1]{AC\ref{#1}}

\newcounter{ucnum}
\newcommand{\uctheucnum}{UC\theucnum}
\newcommand{\uref}[1]{UC\ref{#1}}

\newcounter{mnum}
\newcommand{\mthemnum}{M\themnum}
\newcommand{\mref}[1]{M\ref{#1}}

\newcommand{\progname}{SpecSearch}

\begin{document}

\title{Module Guide: SpecSearch} 
\author{Robert E. White}
\date{\today}

\maketitle
%\pagenumbering{roman}
\section{Revision History}

\begin{tabularx}{\textwidth}{p{3cm}p{2cm}X}
\toprule {\bf Date} & {\bf Version} & {\bf Notes}\\
\midrule
October 26 & 1.0 & Creation of first draft for presentation.\\ 
Nov 4 & 1.1 & Post presentation edits. Feedback from Dr. Smith regarding 
hierarchy and control module.\\
Nov 7 & 1.2 & Peer review edits. \\ 
Nov 9 & 1.3 & Implementation of new MG template. \\ 
\bottomrule
\end{tabularx}

\newpage

\tableofcontents

\listoftables

\listoffigures
\section*{Abbreviations and Acronyms}

\renewcommand{\arraystretch}{1.2}
\begin{tabular}{l l} 
	\toprule		
	\textbf{symbol} & \textbf{description}\\
	\midrule 
	MG & Module Guide\\
	AC & Anticipated Change\\ 
	SRS & System Requirements Specification\\ 
	UC & Unlikely Change\\
	M & Module\\ 
	$cn$ & Elliptic cosine function\\ 
	$sn$ & Elliptic sine function\\ 
	$dn$ & Elliptic delta function \\ 
%	\wss{etc.} & \wss{...}\\
	\bottomrule
\end{tabular}\\

\newpage

\newpage

\pagenumbering{arabic}

\section{Introduction}

Decomposing a system into modules is a commonly accepted approach to developing
software.  A module is a work assignment for a programmer or programming
team~\citep{ParnasEtAl1984}.  We advocate a decomposition
based on the principle of information hiding~\citep{Parnas1972a}.  This
principle supports design for change, because the ``secrets'' that each module
hides represent likely future changes.  Design for change is valuable in SC,
where modifications are frequent, especially during initial development as the
solution space is explored.

Our design follows the rules layed out by \citet{ParnasEtAl1984}, as follows:
\begin{itemize}
\item System details that are likely to change independently should be the
  secrets of separate modules.
\item Each data structure is used in only one module. 
\item Any other program that requires information stored in a module's data
  structures must obtain it by calling access programs belonging to that module.
\end{itemize}

After completing the first stage of the design, the Software Requirements
Specification (SRS), the Module Guide (MG) is developed~\citep{ParnasEtAl1984}. The MG
specifies the modular structure of the system and is intended to allow both
designers and maintainers to easily identify the parts of the software.  The
potential readers of this document are as follows:

\begin{itemize}
\item New project members: This document can be a guide for a new project member
  to easily understand the overall structure and quickly find the
  relevant modules they are searching for.
\item Maintainers: The hierarchical structure of the module guide improves the
  maintainers' understanding when they need to make changes to the system. It is
  important for a maintainer to update the relevant sections of the document
  after changes have been made.
\item Designers: Once the module guide has been written, it can be used to
  check for consistency, feasibility and flexibility. Designers can verify the
  system in various ways, such as consistency among modules, feasibility of the
  decomposition, and flexibility of the design.
\end{itemize}

The rest of the document is organized as follows. Section
\ref{SecChange} lists the anticipated and unlikely changes of the software
requirements. Section \ref{SecMH} summarizes the module decomposition that
was constructed according to the likely changes. Section \ref{SecConnection}
specifies the connections between the software requirements and the
modules. Section \ref{SecMD} gives a detailed description of the
modules. Section \ref{SecTM} includes two traceability matrices. One checks
the completeness of the design against the requirements provided in the SRS. The
other shows the relation between anticipated changes and the modules. Section
\ref{SecUse} describes the use relation between modules.

\section{Anticipated and Unlikely Changes} \label{SecChange}

This section lists possible changes to the system. According to the likeliness
of the change, the possible changes are classified into two
categories. Anticipated changes are listed in Section \ref{SecAchange}, and
unlikely changes are listed in Section \ref{SecUchange}.

\subsection{Anticipated Changes} \label{SecAchange}

Anticipated changes (AC) are the source of the information that is to be hidden
inside the modules. Ideally, changing one of the anticipated changes will only
require changing the one module that hides the associated decision. The approach
adapted here is called design for
change.

\begin{description}
\item[AC1:] The specific hardware on which the software is running.\\
\item[AC2:] The format of the initial input data.\\
\item[AC3:] The format of the output.\\
\item[AC4:] Format of the eigenfunction domain. \\
\item[AC5:] Numerical method of finding eigenvalues. \\ 
\item[AC6:] Construction of Spectrum Matrix \\ 
\item[AC7:] Analytical calculation of eigenvalues. \\
\item[AC8:] Definition of error. \\
\item[AC9:] Numerical method algorithm. \\ 
\item[AC10:] Running speed (or data storage) standards. \\ 
\item[AC11:] Necessary accuracy of integral values. \\ 
\item[AC12:] Method used for approximating functions. \\
\item[AC13:] Method used for plotting data. \\
\end{description}

\subsection{Unlikely Changes (UC)} \label{SecUchange}

The module design should be as general as possible. However, a general system is
more complex. Sometimes this complexity is not necessary. Fixing some design
decisions at the system architecture stage can simplify the software design. If
these decision should later need to be changed, then many parts of the design
will potentially need to be modified. Hence, it is not intended that these
decisions will be changed.

\begin{description}
\item[UC1:] Input/Output devices
  (Input: File and/or Keyboard, Output: File, Memory, and/or Screen).
\item[UC2:] The output (spectrum) is always displayed on the screen as a set of 
points on the complex plane. \wss{Why is this an unlikely change?  Why does the
  output have to go to the screen?  It is an easy (and possibly useful change)
  to send the output to a file, or to make the output a table of values, rather
  than a plot.}
\item[UC3:] The operator matrix and spectrum error can always be created from 
the inputs.

\end{description}

\section{Module Hierarchy} \label{SecMH}

This section provides an overview of the module design. Modules (M) are 
summarized
in a hierarchy decomposed by secrets in Table \ref{TblMH}. The modules listed
below, which are leaves in the hierarchy tree, are the modules that will
actually be implemented.
\begin{description}
\item[M1:] Hardware-Hiding.\\
\item[M2:] Input Parameters.\\
\item[M3:] Output Format.\\
\item[M4:] Spectrum Matrix.\\
\item[M5:] Exact Eigenvalue Equations.\\ 
\item[M6:] Spectrum Error Equation.\\
\item[M7:] Numerical Parameters. \\
\item[M8:] Eigenvalue and Eigenvector Solver. \\
\item[M9:] Diagonal Matrix Generator.\\ 
\item[M10:] Elliptic Integral. \\
\item[M11:] Elliptic Functions. \\
\item[M12:] Plotting. \\ 
\item[M13:] Linspace. \\ 
\item[M14:] Control. \\
\end{description}

\begin{table}[h]
\centering
\begin{tabular}{p{0.3\textwidth} p{0.6\textwidth}}
\toprule
\textbf{Level 1} & \textbf{Level 2}\\
\midrule

{Hardware-Hiding Module} & ~ \\
\midrule

\multirow{7}{0.3\textwidth}{Behaviour-Hiding Module} 
& Input Parameters \\
& Output Format \\
& Spectrum Matrix \\
& Exact Eigenvalue Equations \\
& Spectrum Error Equation \\
& Numerical Parameters \\  
& Control \\ 
\midrule

\multirow{3}{0.3\textwidth}{Software Decision Module} 
& Eigenvalue and Eigenvector Solver \\
& Diagonal Matrix Generator\\
& Elliptic Integral\\ 
& Elliptic Functions\\ 
& Plotting \\ 
& Linspace \\
\bottomrule

\end{tabular}
\caption{Module Hierarchy}
\label{TblMH}
\end{table}

\newpage 
\section{Connection Between Requirements and Design} \label{SecConnection}

The design of the system is intended to satisfy the requirements developed in
the SRS. In this stage, the system is decomposed into modules. The connection
between requirements and modules is listed in Table \ref{TblRT}.

\section{Module Decomposition} \label{SecMD}


Modules are decomposed according to the principle of ``information hiding''
proposed by \citet{ParnasEtAl1984}. The \emph{Secrets} field in a module
decomposition is a brief statement of the design decision hidden by the
module. The \emph{Services} field specifies \emph{what} the module will do
without documenting \emph{how} to do it. For each module, a suggestion for the
implementing software is given under the \emph{Implemented By} title. If the
entry is \emph{OS}, this means that the module is provided by the operating
system or by standard programming language libraries.  \emph{\progname{}} means 
the
module will be implemented by the \progname{} software.

Only the leaf modules in the hierarchy have to be implemented. If a dash
(\emph{--}) is shown, this means that the module is not a leaf and will not have
to be implemented.


\subsection{Hardware Hiding (M1)}

\begin{description}
\item[Secrets:]The data structure and algorithm used to implement the virtual
  hardware.
\item[Services:]Serves as a virtual hardware used by the rest of the
  system. This module provides the interface between the hardware and the
  software. So, the system can use it to display outputs or to accept inputs.
\item[Implemented By:] OS
\end{description}

\subsection{Behaviour-Hiding Module }

\begin{description}
\item[Secrets:]The contents of the required behaviours.
\item[Services:]Includes programs that provide externally visible behaviour of
  the system as specified in the SRS
  documents. This module serves as a communication layer between the
  hardware-hiding module and the software decision module. The programs in this
  module will need to change if there are changes in the SRS.
\item[Implemented By:] -
\end{description}

\subsubsection{Input Parameters (M2)}

\begin{description}
	\item[Secrets:] The data structure for input parameters and how the values 
	are verified. 
	\item[Services:] Gets input from the user, stores the input and verifies 
	that 
	the input variables satistfy \wss{spell check} the constraints in the SRS. Throws an error if 
	any of 
	the inputs violate a constraint and converts the input data into a data 
	structure that is 
	appropriate for the other 
	modules. 
	\item[Implemented By:]SpecSearch 
\end{description} 

\subsubsection{Output Format (M3)}

\begin{description}
	\item[Secrets:] The format and structure of the output data.
	\item[Services:] Converts the output data from the spectrum error module 
	and eigenvalue solver module into a data structure necessary for the 
	plotting module. 
	\item[Implemented By:] SpecSearch
\end{description}

\subsubsection{Spectrum Matrix (M4)}

\begin{description}
	\item[Secrets:]The structure of the spectrum matrix, its data entries, how 
	it is created, and the numerical method for approximating its 
	eigenfunctions. 
	\item[Services:]Creates the matrix that approximates the operator matrix 
	from the lax pair (see SRS). \wss{A more specific reference to the SRS
          would be great.} 
	\item[Implemented By:] SpecSearch
\end{description}

\subsubsection{Exact Eigenvalue Equations (M5)}

\begin{description}
	\item[Secrets:]The analytical expression for the two real eigenvalues. 
	\item[Services:]Calculates the two purely real eigenvalues from 
	literature for $k$. $k$ is the elliptic parameter. 
	\item[Implemented By:] SpecSearch
\end{description}

\subsubsection{Spectrum Error Equation (M6)}

\begin{description}
	\item[Secrets:] The measure for error between exact and approximated 
	eigenvalues.
      \item[Services:]Calculates the absolute value of the difference between
        the the \wss{proof read} numerical (calculated with M9 and M5) and
        theoretical eigenvalues. \wss{Do you need a specific module for this?
          Is there an algorithm here that justifies making this a secret? Why
          not relative error?  Isn't this a one line formula in the code?  Maybe
          I'll understand better when you write your MIS.}
	\item[Implemented By:] SpecSearch
\end{description} 

\subsubsection{Numerical Parameters (M7)} 

\begin{description}
	\item[Secrets:] The range of the eigenfunction domain, points in the 
	periodic domain and equation for the numerical scaling factor that computes 
	the 
	eigenfunction derivatives. \wss{How are these different from input?
          Instead of changing from execution to execution, are these values
          constant between runs?  Are they like configuration parameters for the
          numerical algorithm?}
	\item[Services:] Creates the numerical parameters used for approximating 
	the derivatives of the eigenfunctions. 
	\item[Implemented By:] SpecSearch
\end{description} 

\subsubsection{Control (M14)} 

\begin{description}
	\item[Secrets:] The algorithm that coordinates the overall program and 
	interaction between modules. 
	\item[Services:] Is the main program.
	\item[Implemented By:] SpecSearch
\end{description}

\subsection{Software Decision Module}

\begin{description}
\item[Secrets:] The design decision based on mathematical theorems, physical
  facts, or programming considerations. The secrets of this module are
  \emph{not} described in the SRS.
\item[Services:] Includes data structure and algorithms used in the system that
  do not provide direct interaction with the user. 
  % Changes in these modules are more likely to be motivated by a desire to
  % improve performance than by externally imposed changes.
\item[Implemented By:] -
\end{description}

\subsubsection{Eigenvalue and Vector Solver (M8)} 

\begin{description}
	\item[Secrets:] The numerical algorithm for calculating the eigenvalues and 
	eigenvectors of an 
	$n$ by $n$ matrix.
	\item[Services:] The eig MATLAB function finds the eigenvalues and vectors 
	of an 
	arbitrary $n$ by $n$ matrix.
	\item[Implemented By:] MATLAB
\end{description} 

\subsubsection{Diagonal Matrix (M9)} 

\begin{description}
	\item[Secrets:] The numerical algorithm for creating an $n$ by $n$ diagonal 
	matrix from an $n$ by 1 vector (and other way). 
	\item[Services:] The diag MATLAB function creates an $n$ by $n$ diagonal 
	matrix 
	from a 1 by $n$ vector. The diagonal entries of the matrix are the elements 
	of the vector. The 
	diag function also creates a 1 by $n$ vector from a diagonal
        matrix. \wss{This is more of a function than a module.  This is a
          service provided by the ``matrix'' module in MatLab.}
	\item[Implemented By:] MATLAB
\end{description} 

\subsubsection{Elliptic Integral (M10)} 

\begin{description}
	\item[Secrets:] The numerical algorithm for calculating the complete 
	elliptic integral for some real constant $k$. 
	\item[Services:] The elliptK MATLAB function calculates the integral of $$ 
	\int_{0}^{\frac{\pi}{2}} \frac{dx}{\sqrt{1-msin^{2}(x)}}$$. \\ 
	\item[Implemented By:] MATLAB
\end{description} 

\subsubsection{Elliptic Functions (M11)} 

\begin{description}
	\item[Secrets:] The numerical algorithm for calculating the values of the 
	Jacobi elliptic functions. 
	\item[Services:] The ellipj MATLAB function calculates the values of 
	$dn,cn$ 
	and $sn$ for a particular domain/vector. 
	\item[Implemented By:] MATLAB
\end{description} 

\subsubsection{Plotting (M12)} 

\begin{description}
	\item[Secrets:] The plotting methods/algorithms.
	\item[Services:] Creates a two dimensional plot given a domain vector and a 
	range vector of equal size.  
	\item[Implemented By:] MATLAB
\end{description} 

\subsubsection{Linspace (M13)} 

\begin{description}
	\item[Secrets:] The software algorithm for creating a vector with equally 
	spaced entries. 
      \item[Services:] The linspace MATLAB function creates an array with
        prescribed endpoints and an equal difference between adjacent
        points. \wss{I think this is more of a function than a module.}
	\item[Implemented By:] MATLAB
\end{description} 

\section{Traceability Matrix} \label{SecTM}

This section shows two traceability matrices: between the modules and the
requirements and between the modules and the anticipated changes.

% the table should use mref, the requirements should be named, use something
% like fref
\begin{table}[H]
\centering
\begin{tabular}{p{0.2\textwidth} p{0.6\textwidth}}
\toprule
\textbf{Req.} & \textbf{Modules}\\
\midrule
Rin & M1 , M2 , M14\\
Rfind & M4, M7, M8, M9, M10, M11,M12, M14\\
Rcon & M12,M14,M5,M6\\
Rplt & M12,M14 \\
Rstl & M13,M14\\
\bottomrule
\end{tabular}
\caption{Trace Between Requirements and Modules}
\label{TblRT}
\end{table}

\begin{table}[H]
\centering
\begin{tabular}{p{0.2\textwidth} p{0.6\textwidth}}
\toprule
\textbf{AC} & \textbf{Modules}\\
\midrule
AC1 & M1\\
AC2 & M2\\
AC3 & M3\\
AC4 & M13\\
AC5 & M8\\ 
AC6 & M4\\ 
AC7 & M5\\ 
AC8 & M6\\ 
AC9 & M7\\ 
AC10 & M9\\ 
AC11 & M10\\
AC12 & M11\\ 
AC13 & M12\\
\bottomrule
\end{tabular}
\caption{Trace Between Anticipated Changes and Modules}
\label{TblACT}
\end{table}

\section{Use Hierarchy Between Modules} \label{SecUse}

In this section, the uses hierarchy between modules is
provided. \citet{Parnas1978} said of two programs A and B that A {\em uses} B if
correct execution of B may be necessary for A to complete the task described in
its specification. That is, A {\em uses} B if there exist situations in which
the correct functioning of A depends upon the availability of a correct
implementation of B.  Figure \ref{FigUH} illustrates the use relation between
the modules. It can be seen that the graph is a Directed Acyclic Graph
(DAG). Each level of the hierarchy offers a testable and usable subset of the
system, and modules in the higher level of the hierarchy are essentially simpler
because they use modules from the lower levels. \\

%\begin{figure}[H]
%\centering
%\includegraphics[width=0.7\textwidth]{UsesHierarchy.png}
%\caption{Use hierarchy among modules}
%\label{FigUH}
%\end{figure}


% Define block styles
\tikzstyle{decision} = [diamond, draw, fill=blue!20, 
text width=4.5em, text badly centered, node distance=3cm, inner sep=0pt]
\tikzstyle{block} = [rectangle, draw, fill=blue!20, 
text width=5em, text centered, rounded corners, minimum height=4em]
\tikzstyle{line} = [draw, -latex']
\tikzstyle{cloud} = [draw, ellipse,fill=red!20, node distance=3cm,
minimum height=2em] 

\newpage
\begin{figure}[h!]
\begin{tikzpicture}[node distance = 2cm, auto]
% Place nodes
\node [block] (SpecErr) {Spectrum Error Equations (M6)};
\node [block, below of=SpecErr,node distance=2.5cm] (out) {Output Format (M3)};
\node [block, right of=out,node distance=3cm] (plot) {Plotting (M12)}; 
\node [block, left of=out,node distance=3cm] (eigen) {Eigenvalue and 
Eigenvector solver (M8)}; 
\node [block, below of=out,node distance=11cm] (elipint) {Elliptic 
Integral 
(M10)} ; 
\node [block, below of=elipint,node distance=2.5cm] (input) {Input Parameters 
(M2)} ;
\node [block, below of=eigen, node distance=3.5cm] (specmat) {Spectrum Matrix 
(M4)};
\node [block, below of=specmat, node distance=3cm] (numpar) {Numerical 
Parameters (M7)};
\node [block, left of=elipint,node distance=6cm] (linspace) {linspace (M13)};
\node [block, left of=numpar,node distance=3cm] (elipfun) {Elliptic Functions 
(M11)} ; 
\node [block, right of=numpar, node distance=9cm] (exacteig) {Exact Eigen 
Value Equations (M5)};
\node [block, left of= SpecErr, node distance= 7cm] (control) {Control (M14)};
\node [block, left of=specmat,node distance=3cm] (diag) {Diagonal Matrix 
(M9)} ;
% Draw edges
\path [line] (out) -- (SpecErr);
\path [line] (SpecErr) -| (eigen);
\path [line] (plot) -- (out);
\path [line] (out) -- (eigen); 
\path [line] (SpecErr) -| (exacteig);
\path [line] (eigen) -- (specmat); 
\path [line] (specmat) -- (numpar); 
\path [line] (numpar) -- (diag);
\path [line] (numpar) -- (elipfun);
\path [line] (numpar) -- (linspace); 
\path [line] (numpar) -- (elipint); 
\path [line] (elipint) -- (input); 
\path [line] (numpar) |- (input); 
\path [line] (exacteig) |- (input);
\path [line] (out) |- (numpar);
\end{tikzpicture}
\caption{Use hierarchy among modules}
\label{FigUH}
\end{figure} 
% \section*{References}

\wss{Nice use of tikz.}

\wss{Why doesn't the Control module use any other modules?  Try to draw your
  uses relation so that all arrows are down.  Why does numerical parameters use
  the Diagonal Matrix (and other services)?  I had the impression that numerical
parameters were configuration constants.  Why does numerical parameters use
input parameters?  I'm sorry, but I really don't understand your design.}
\newpage

\wss{Your bibliography didn't compile properly.  I had to modify the path.  I
  had to do the same thing for the Comments.tex file.}

\wss{Throughout this document you hard code in your references to other modules
  (M2, M9 etc.).  You should make your life easier and use labels and references
  in \LaTeX{}.}

\bibliographystyle {plainnat}
\bibliography{../../ReferenceMaterial/References} 

\end{document}
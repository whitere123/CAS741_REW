\documentclass[12pt, titlepage]{article}

\usepackage{fullpage}
\usepackage[round]{natbib}
\usepackage{multirow}
\usepackage{booktabs}
\usepackage{tabularx}
\usepackage{graphicx}
\usepackage{float}
\usepackage{hyperref}
\hypersetup{
    colorlinks,
    citecolor=black,
    filecolor=black,
    linkcolor=red,
    urlcolor=blue
}

%\input{../../Comments}

\newcounter{acnum}
\newcommand{\actheacnum}{AC\theacnum}
\newcommand{\acref}[1]{AC\ref{#1}}

\newcounter{ucnum}
\newcommand{\uctheucnum}{UC\theucnum}
\newcommand{\uref}[1]{UC\ref{#1}}

\newcounter{mnum}
\newcommand{\mthemnum}{M\themnum}
\newcommand{\mref}[1]{M\ref{#1}}

\begin{document}

\title{Module Guide: SpecSearch} 
\author{Robert E. White}
\date{\today}

\maketitle
%\pagenumbering{roman}
\section{Revision History}

\begin{tabularx}{\textwidth}{p{3cm}p{2cm}X}
\toprule {\bf Date} & {\bf Version} & {\bf Notes}\\
\midrule
October 26 & 1.0 & Creation of first draft for presentation.\\
\bottomrule
\end{tabularx}

\newpage

\tableofcontents

\listoftables

\listoffigures

\newpage

\pagenumbering{arabic}

\section{Introduction}

Decomposing a system into modules is a commonly accepted approach to developing
software.  A module is a work assignment for a programmer or programming
team~\citep{ParnasEtAl1984}.  We advocate a decomposition
based on the principle of information hiding~\citep{Parnas1972a}.  This
principle supports design for change, because the ``secrets'' that each module
hides represent likely future changes.  Design for change is valuable in SC,
where modifications are frequent, especially during initial development as the
solution space is explored.  

Our design follows the rules layed out by \citet{ParnasEtAl1984}, as follows:
\begin{itemize}
\item System details that are likely to change independently should be the
  secrets of separate modules.
\item Each data structure is used in only one module.
\item Any other program that requires information stored in a module's data
  structures must obtain it by calling access programs belonging to that module.
\end{itemize}

After completing the first stage of the design, the Software Requirements
Specification (SRS), the Module Guide (MG) is developed~\citep{ParnasEtAl1984}. The MG
specifies the modular structure of the system and is intended to allow both
designers and maintainers to easily identify the parts of the software.  The
potential readers of this document are as follows:

\begin{itemize}
\item New project members: This document can be a guide for a new project member
  to easily understand the overall structure and quickly find the
  relevant modules they are searching for.
\item Maintainers: The hierarchical structure of the module guide improves the
  maintainers' understanding when they need to make changes to the system. It is
  important for a maintainer to update the relevant sections of the document
  after changes have been made.
\item Designers: Once the module guide has been written, it can be used to
  check for consistency, feasibility and flexibility. Designers can verify the
  system in various ways, such as consistency among modules, feasibility of the
  decomposition, and flexibility of the design.
\end{itemize}

The rest of the document is organized as follows. Section
\ref{SecChange} lists the anticipated and unlikely changes of the software
requirements. Section \ref{SecMH} summarizes the module decomposition that
was constructed according to the likely changes. Section \ref{SecConnection}
specifies the connections between the software requirements and the
modules. Section \ref{SecMD} gives a detailed description of the
modules. Section \ref{SecTM} includes two traceability matrices. One checks
the completeness of the design against the requirements provided in the SRS. The
other shows the relation between anticipated changes and the modules. Section
\ref{SecUse} describes the use relation between modules.

\section{Anticipated and Unlikely Changes} \label{SecChange}

This section lists possible changes to the system. According to the likeliness
of the change, the possible changes are classified into two
categories. Anticipated changes are listed in Section \ref{SecAchange}, and
unlikely changes are listed in Section \ref{SecUchange}.

\subsection{Anticipated Changes} \label{SecAchange}

Anticipated changes are the source of the information that is to be hidden
inside the modules. Ideally, changing one of the anticipated changes will only
require changing the one module that hides the associated decision. The approach
adapted here is called design for
change.

\begin{description}
\item[AC1:] The specific hardware on which the software is running.\\
\item[AC2:] The format of the initial input data.\\
\item[AC3:] The format of the output.\\
\item[AC4:] Format of the eigenfunction domain. \\
\item[AC5:] Numerical method of finding eigenvalues. \\
\end{description}

\subsection{Unlikely Changes} \label{SecUchange}

The module design should be as general as possible. However, a general system is
more complex. Sometimes this complexity is not necessary. Fixing some design
decisions at the system architecture stage can simplify the software design. If
these decision should later need to be changed, then many parts of the design
will potentially need to be modified. Hence, it is not intended that these
decisions will be changed.

\begin{description}
\item[UC1:] Input/Output devices
  (Input: File and/or Keyboard, Output: File, Memory, and/or Screen).\\
\item[UC2:] There will always be
  a source of input data external to the software.\\
\item[UC3:] There will always be a two dimensional plot of the spectrum. \\
\item[UC4:] There will always be an error equation. \\
\end{description}

\newpage 
\section{Module Hierarchy} \label{SecMH}

This section provides an overview of the module design. Modules are summarized
in a hierarchy decomposed by secrets in Table \ref{TblMH}. The modules listed
below, which are leaves in the hierarchy tree, are the modules that will
actually be implemented.

\begin{description}
\item[M1:] Hardware-Hiding Module.\\
\item[M2:] Input Parameters Module.\\
\item[M3:] Input Format Module.\\
\item[M4:] Output Format Module. \\
\item[M5:] Spectrum Matrix Module. \\
\item[M6:] Exact Lambda Equations Module.\\ 
\item[M7:] Lambda Error Equation Module. \\
\item[M8:] Domain Creation Module. \\
\item[M9:] Eigenvalue Solver Module. \\
\item[M10:] Eigenvector Solver Module. \\
\item[M11:] Diagonal Matrix Generator Module.\\ 
\item[M12:] Elliptic Integral Module. \\
\item[M13:] Elliptic Function Module. \\
\item[M14:] Plotting Module. \\
\end{description}

\begin{table}[h]
\centering
\begin{tabular}{p{0.3\textwidth} p{0.6\textwidth}}
\toprule
\textbf{Level 1} & \textbf{Level 2}\\
\midrule

{Hardware-Hiding Module} & ~ \\
\midrule

\multirow{7}{0.3\textwidth}{Behaviour-Hiding Module} 
& Input Parameters Module\\
& Input Format Module \\
& Output Format Module \\
& Spectrum Matrix Module\\
& Exact Lambda Equations Module\\
& Lambda Error Equation Module\\
& Domain Creation Module\\ 
\midrule

\multirow{3}{0.3\textwidth}{Software Decision Module} 
& Eigenvalue Solver Module \\
& Eigenvector Solver Module \\
& Diagonal Matrix Generator Module\\
& Elliptic Integral Module\\ 
& Ellipitic Function Module\\ 
& Plotting Module\\
\bottomrule

\end{tabular}
\caption{Module Hierarchy}
\label{TblMH}
\end{table}

\newpage 
\section{Connection Between Requirements and Design} \label{SecConnection}

The design of the system is intended to satisfy the requirements developed in
the SRS. In this stage, the system is decomposed into modules. The connection
between requirements and modules is listed in Table \ref{TblRT}.

\section{Module Decomposition} \label{SecMD}

Modules are decomposed according to the principle of ``information hiding''
proposed by \citet{ParnasEtAl1984}. The \emph{Secrets} field in a module
decomposition is a brief statement of the design decision hidden by the
module. The \emph{Services} field specifies \emph{what} the module will do
without documenting \emph{how} to do it. For each module, a suggestion for the
implementing software is given under the \emph{Implemented By} title. If the
entry is \emph{OS}, this means that the module is provided by the operating
system or by standard programming language libraries.  Also indicate if the
module will be implemented specifically for the software.

Only the leaf modules in the
hierarchy have to be implemented. If a dash (\emph{--}) is shown, this means
that the module is not a leaf and will not have to be implemented. Whether or
not this module is implemented depends on the programming language
selected.

\subsection{Hardware Hiding Module (M1)}

\begin{description}
\item[Secrets:]The data structure and algorithm used to implement the virtual
  hardware.
\item[Services:]Serves as a virtual hardware used by the rest of the
  system. This module provides the interface between the hardware and the
  software. So, the system can use it to display outputs or to accept inputs.
\item[Implemented By:] OS
\end{description}

\subsection{Behaviour-Hiding Module }

\begin{description}
\item[Secrets:]The contents of the required behaviours.
\item[Services:]Includes programs that provide externally visible behaviour of
  the system as specified in the software requirements specification (SRS)
  documents. This module serves as a communication layer between the
  hardware-hiding module and the software decision module. The programs in this
  module will need to change if there are changes in the SRS.
\item[Implemented By:] --
\end{description}

\subsubsection{Input Parameters Module (M2)}

\begin{description}
	\item[Secrets:]Names of the input parameters. 
	\item[Services:]Requests input from the user to be used by the Input format 
	module. 
	\item[Implemented By:]SpecSearch 
\end{description} 

\subsubsection{Input Format Module (M3)}

\begin{description}
\item[Secrets:]The format and structure of the input data.
\item[Services:]Converts the input data into a data structure for the other 
modules.
\item[Implemented By:]SpecSearch 
\end{description} 

\subsubsection{Output Format Module (M4)}

\begin{description}
	\item[Secrets:]The format and structure of the output data.
	\item[Services:]Converts the output data into the data structure used by the
	plotting module. 
	\item[Implemented By:] SpecSearch
\end{description}

\subsubsection{Operator Matrix Module (M5)}

\begin{description}
	\item[Secrets:]The format and structure of the approximated operator matrix 
	from the lax 
	pair (see SRS). The entry values of the matrix operator and numerical 
	method for approximating eigenfunction derivatives are also secrets. 
	\item[Services:]Creates the matrix meant to approximate the operator matrix 
	from the lax pair (see SRS). This matrix will be used by the Eigenvalue and 
	Eigenvector Solver Modules. 
	\item[Implemented By:] SpecSearch
\end{description}

\subsubsection{Exact Lambda Equations Module (M6)}

\begin{description}
	\item[Secrets:]The analytical expression for the two real eigenvalues. 
	\item[Services:]Calculates the two real eigenvalues from 
	Segal et al. These two values will be used by the Lambda Error Equation 
	Module. 
	\item[Implemented By:] SpecSearch
\end{description}

\subsubsection{Lambda Error Equation Module (M7)}

\begin{description}
	\item[Secrets:] The eigenvalue error equation.
	\item[Services:]Calculates the difference between the the actual and 
	theoretical eigenvalues from M7. This module will be used by the plotting 
	module. 
	\item[Implemented By:] SpecSearch
\end{description} 

\subsubsection{Domain Creation Module (M8)} 

\begin{description}
	\item[Secrets:] The format and structure of the input data.
	\item[Services:] Calculates the domain for which the eigenfunctions will be 
	calculated. This module will be used by the operator matrix module. 
	\item[Implemented By:] SpecSearch
\end{description}

\subsection{Software Decision Module}

\begin{description}
\item[Secrets:] The design decision based on mathematical theorems, physical
  facts, or programming considerations. The secrets of this module are
  \emph{not} described in the SRS.
\item[Services:] Includes data structure and algorithms used in the system that
  do not provide direct interaction with the user. 
  % Changes in these modules are more likely to be motivated by a desire to
  % improve performance than by externally imposed changes.
\item[Implemented By:] --
\end{description}

\subsubsection{Eigenvalue Solver Module (M9)} 

\begin{description}
	\item[Secrets:] The numerical algorithm for finding the eigenvalues of an 
	$n$ by $n$ matrix.
	\item[Services:] The eig MATLAB function finds the eigenvalues of an 
	arbitrary n by n matrix.
	\item[Implemented By:] SpecSearch
\end{description} 

\subsubsection{Eigenvector Solver Module (M10)} 

\begin{description}
	\item[Secrets:] The numerical algorithm for solving the eigenvalues of an 
	$n$ by $n$ matrix.
	\item[Services:] The eig MATLAB function finds the eigenvectors of an 
	arbitrary n by n matrix. 
	\item[Implemented By:] SpecSearch
\end{description} 

\subsubsection{Diagonal Matrix Module (M11)} 

\begin{description}
	\item[Secrets:] The numerical algorithm for creating an n by n diagonal 
	matrix.
	\item[Services:] The diag MATLAB function creates an n by n diagonal matrix 
	from a 1 by n vector. The diagonal entries correspond to the matrix. The 
	diag function also creates a 1 by n vector from a diagonal matrix. This 
	module will be used by the operator matrix module. 
	\item[Implemented By:] SpecSearch
\end{description} 

\subsubsection{Elliptic Integral Module (M12)} 

\begin{description}
	\item[Secrets:] The numerical algorithm for calculating the complete 
	elliptic integral for some constant k. 
	\item[Services:] The elliptK MATLAB function calculates the integral of $$ 
	\int_{0}^{\frac{\pi}{2}} \frac{dx}{\sqrt{1-msin^{2}(x)}}$$. \\ 
	This module will be used by the Domain Creation Module. 
	\item[Implemented By:] SpecSearch
\end{description} 

\subsubsection{Elliptic Function Module (M13)} 

\begin{description}
	\item[Secrets:] The numerical algorithm for calculating the values of the 
	jacobi elliptic functions. 
	\item[Services:] The ellipj MATLAB function calculates the values of dn,cn 
	and sn for a particular domain. This module will be used by the operator 
	matrix module. 
	\item[Implemented By:] SpecSearch
\end{description} 

\subsubsection{Plotting Module (M14)} 

\begin{description}
	\item[Secrets:] The plotting methods/algorithms.
	\item[Services:] Creates a plot with prescribed input and output.  
	\item[Implemented By:] SpecSearch
\end{description}

\section{Traceability Matrix} \label{SecTM}

This section shows two traceability matrices: between the modules and the
requirements and between the modules and the anticipated changes.

% the table should use mref, the requirements should be named, use something
% like fref
\begin{table}[H]
\centering
\begin{tabular}{p{0.2\textwidth} p{0.6\textwidth}}
\toprule
\textbf{Req.} & \textbf{Modules}\\
\midrule
Rin & M1 , M2, M3 \\
Rfind & M5, M7, M8, M9, M10, M11, M12,M13\\
Rcon & M8\\
Rplt & M14 \\
Rstl & M14\\
\bottomrule
\end{tabular}
\caption{Trace Between Requirements and Modules}
\label{TblRT}
\end{table}

\begin{table}[H]
\centering
\begin{tabular}{p{0.2\textwidth} p{0.6\textwidth}}
\toprule
\textbf{AC} & \textbf{Modules}\\
\midrule
AC1 & M1\\
AC2 & M3\\
AC3 & M4\\
AC4 & M5, M8\\
AC5 & M9,M10\\
\bottomrule
\end{tabular}
\caption{Trace Between Anticipated Changes and Modules}
\label{TblACT}
\end{table}

\section{Use Hierarchy Between Modules} \label{SecUse}

In this section, the uses hierarchy between modules is
provided. \citet{Parnas1978} said of two programs A and B that A {\em uses} B if
correct execution of B may be necessary for A to complete the task described in
its specification. That is, A {\em uses} B if there exist situations in which
the correct functioning of A depends upon the availability of a correct
implementation of B.  Figure \ref{FigUH} illustrates the use relation between
the modules. It can be seen that the graph is a directed acyclic graph
(DAG). Each level of the hierarchy offers a testable and usable subset of the
system, and modules in the higher level of the hierarchy are essentially simpler
because they use modules from the lower levels.

\begin{figure}[H]
\centering
%\includegraphics[width=0.7\textwidth]{UsesHierarchy.png}
\caption{Use hierarchy among modules}
\label{FigUH}
\end{figure}

%\section*{References}

\bibliographystyle {plainnat}
\bibliography{../../../ReferenceMaterial/References}

\end{document}
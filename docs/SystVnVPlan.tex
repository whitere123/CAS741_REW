\documentclass[12pt, titlepage]{article}

\usepackage{booktabs}
\usepackage{tabularx}
\usepackage{hyperref}
\hypersetup{
    colorlinks,
    citecolor=black,
    filecolor=black,
    linkcolor=red,
    urlcolor=blue
}
\usepackage[round]{natbib}

%% Comments

\usepackage{color}

\newif\ifcomments\commentstrue

\ifcomments
\newcommand{\authornote}[3]{\textcolor{#1}{[#3 ---#2]}}
\newcommand{\todo}[1]{\textcolor{red}{[TODO: #1]}}
\else
\newcommand{\authornote}[3]{}
\newcommand{\todo}[1]{}
\fi

\newcommand{\wss}[1]{\authornote{blue}{SS}{#1}}
\newcommand{\an}[1]{\authornote{magenta}{Author}{#1}}



\begin{document}

\title{Project Title: System Verification and Validation Plan} 
\author{Robert White}
\date{\today}
	
\maketitle

\pagenumbering{roman}

\section{Revision History}

\begin{tabularx}{\textwidth}{p{3cm}p{2cm}X}
\toprule {\bf Date} & {\bf Version} & {\bf Notes}\\
\midrule
2018-10-13 & 1.0 & Creation of first draft for VnV plan presentation.\\
\bottomrule
\end{tabularx}

~\newpage

\section{Symbols, Abbreviations and Acronyms}

\renewcommand{\arraystretch}{1.2}
\begin{tabular}{l l} 
  \toprule		
  \textbf{symbol} & \textbf{description}\\
  \midrule 
  T & Test\\
  \bottomrule
\end{tabular}\\

\wss{symbols, abbreviations or acronyms -- you can simply reference the SRS
  tables, if appropriate}

\newpage

\tableofcontents

\listoftables

\listoffigures

\newpage

\pagenumbering{arabic}

This document discusses the verification and validation requirements for 
SpecSearch. The Project Management Body of Knowledge (PMBOK) guide provides 
unambigious definitions for verification and validation. It defines 
verification as the evaluation of whether or not a product, service, or system 
complies with a regulation, requirement, specification, or imposed condition. 
It is often an internal process. It defines validation as the assurance that a 
product, service, or system meets the needs of the customer and other 
identified stakeholders. It often involves acceptance and suitability with 
external customers. (PMBOK)

\section{General Information}

\subsection{Summary}

\wss{Say what software is being tested.  Give its name and a brief overview of
  its general functions.} \\
The software being tested is called SpecSearch. SpeacSearch will search for the 
spectrum of a particular lax equation from a lax pair that is compatbilible 
with solutions to the Non-Linear Schrodinger (NLS) Equation. It will also use 
the spectral information to determine the stability of the solutions. 
\subsection{Objectives}

\wss{State what is intended to be accomplished.  The objective will be around
  the qualities that are most important for your project.  You might have
  something like: ``build confidence in the software correctness,''
  ``demonstrate adequate usability.'' etc.  You won't list all of the qualities,
  just those that are most important.} \\
\begin{itemize}
	\item Build confidence in software correctness.
	\item Ensure maintainability and manageability (can easily 
	have more features/numerical algorithms added).
	\item Satisfy the requirements of my thesis supervisor and those outlined 
	in the SRS. 
	\item Verify effectiveness and ease of use (usability).	
\end{itemize}

\subsection{References}

\wss{Reference relevant documentation. This will definitely include your SRS}
\begin{itemize} 
	\item Dr. Smith's VnV Template 
	\item PMBOK 
	\item 
	https://www.wqusability.com/articles/more-tincorpohan-ease-of-use.html 
	\item 
	http://people.ucalgary.ca/~design/engg251/First%20Year%20Files/design_verif.pdf
	\item Bernard Deconinck and Benjamin L.Segal.
	
\end{itemize}

\section{Plan}
	
\subsection{Verification and Validation Team}

The verification and validation team consists of my theis supervisor, Dr. 
Dmitry Pelinovsky, and I. 

\subsection{SRS Verification Plan}

\wss{List any approaches you intend to use for SRS verification.  This may just
  be ad hoc feedback from reviewers, like your classmates, or you may have
  something more rigorous/systematic in mind..} \\ 
\begin{itemize}
	\item Feedback from thesis supervisor about mathematical theory, 
	assumptions, constraints and goals.
	\item Feedback from classmates and Dr. Smith about document outline, 
	readability, clarity and requirements. 
	
\end{itemize}


\subsection{Design Verification Plan}

\wss{Plans for design verification} 
\begin{itemize}
	\item Inspection of the software by my thesis supervisor. 
\end{itemize}

\subsection{Implementation Verification Plan}

\wss{You should at least point to the tests listed in this document and the unit
  testing plan.} 
\begin{itemize}
\item A software verification checklist: Does the software allow the user to 
fufill their responsibilities? Does the software fufill its intended 
responsibilities? Can the intended user understand the software? Are the system 
constraints violated? Does the software achieve its intended goals? Does the 
solution have the required properties outlined in the SRS? 
\item Did the functional and nonfunctional requirement tests pass (see below)? 
\end{itemize}
\subsection{Software Validation Plan}

\wss{If there is any external data that can be used for validation, you should
  point to it here.  If there are no plans for validation, you should state that
  here.} 
\begin{itemize} 
	\item The stability spectrum for elliptic solutions to the focusing NLS 
	equation. (paper) 
\end{itemize}

\section{System Test Description}
	
\subsection{Tests for Functional Requirements}

\wss{Subsets of the tests may be in related, so this section is divided into
  different areas.  If there are no identifiable subsets for the tests, this
  level of document structure can be removed.}

\begin{enumerate}

\item{test-Rin1\\}

Control: Enough versus Not Enough Information
					
Initial State: Software with no inputed data.
					
Input: Varying initial data configurations (ie missing certain inputs and 
having all inputs)
					
Output: Error or pass message.
					
How test will be performed: All possible combinations of ommited variables (and 
writing variables non-numerically, such as accidently characters or symbols) in 
the input will be considered. A successful test in this case will be an error 
message upon running the software with these inputs. On the other hand we will 
test cases with each variable in the input having a numerical value. These 
numerical values will vary between negative,positive and zero values. A 
successful test in this case will be a pass message.
					
\item{test-Rfind1} 

Control: Full versus non-full eigenvalue array. 

Initial State: Software with prescribed input data. 

Input: The array from SpecSearch that stores the eigenvalues. 

Output: Binary output (Full or non-full array)

How the test will be performed: A program will go through the eigen-value array 
to ensure that each entry is filled with a complex number. 


\item{test-Rcon1} 

Initial State: The software system with prescribed input.

Input: The approximated continuous spectrum with a tag on explicitly calculated 
values.

Output: Binary Variable (Connected or disconnected)

How test will be performed: This test will check to see if there is a 
sufficient amount of points between the tagged portions of the spectrum. This 
will require a criteria for 'computer continuous'. 

\item{test-Rplt} 

Initial State: The software system with prescribed input. 

Input: A plot of the spectrum. 

Output: A plot with regions of calculated eigenvalues emphasized/circled. 

How test will be performed: I will write a program that takes in the spectrum 
as input and puts explicity labels on explicitly caculated eigen values. 

\item{test-Rstl} 

Initial State: A completed run of SpecSearch with prescribed input. 

Input: The stability results from SpecSearch. 

Output: Binary Variable (Verification of stability) 

How test will be performed: The stability results will be compared with the 
stability analysis in (ref1). 

\end{enumerate}

\subsection{Tests for Nonfunctional Requirements}


\begin{enumerate}

\item{test-NFR1\\}

Type: Static
					
Initial State: -
					
Input/Condition: SpecSearch MATLAB code
					
Output/Result: Pass or Fail
					
How test will be performed: The software will be manually read by the developer 
to see if there is a more effective code structure to allow implementation of 
new numerical algorithms. 
					
\item{test-NFR2a\\}

Type: Manual
					
Initial State: Software system with prescribed input.
					
Input: Matrix generated from an instance of SpecSearch.
					
Output: Pass or fail.
					
How test will be performed: The matrix created with the prescribed inputs in 
SpecSearch will be extracted and its eigenvalues will be found using another 
software package. The eigenvalues calculated from SpecSearch and the other 
package will be compared. The test will be successful if SpecSearch deviates 
within a certain percentage of the other software's eigenvalues.

\item{test-NFR2b\\} 

Type: Manual 

Initial State: Software system with prescribed input.

Input: Matrix derived from the time indepent lax equation.

Output: Pass or fail.

How test will be performed: The matrix output ought to have a specific form for 
each instance of SpecSearch. This test will loop through each element of the 
matrix to ensure it has the correct form (ie two of the quadrants are diagonal 
matrices and the other have a certain 'diagonal pattern') 

\item{test-NFR2c\\} 

Type: Manual 

Initial State: Software system with prescribed input.

Input: Matrix derived from the time indepent lax equation.

Output: Pass or fail.

How test will be performed: The output will be tested against the boundary 
value eigen-values derived analytically in Deconinck and Segal. The standard of 
necessary accuracy will be determined by my supervisor.  

\end{enumerate}

\subsection{Traceability Between Test Cases and Requirements}

\wss{Provide a table that shows which test cases are supporting which
  requirements.}

\section{Static Verification Techniques}

\wss{In this section give the details of any plans for static verification of
  the implementation.  Potential techniques include code walkthroughs, code
  inspection, static analyzers, etc.} 

\begin{itemize}
	\item Code inspection : I will 
	go through the code to see if each step is correct with respect to the
	mathematical theory. In particular I will ensure that: 
	\begin{itemize}
		\item variables are being used in the right context. 
		\item any discretization of functions is performed accurately about the 
		origin. For example, equal step sizes in either directions. 
		\item functions from other packages are being used in the right 
		context. For example, some packages have different standards for 
		constants. Theoretical convention may square a constant while the code 
		may take the squared value directly as the constant. 
		\item the dimensions of the vectors and matrices are appropriate. For 
		example, multiplication of row with column versus column with row. Or 
		that rows in matrices are seperated with semi-colons. 
		\item ensure a varible is not accidently overwritten or cleared. 
	\end{itemize} 
	\item Code walkthrough: My supervisor and I will go through the code 
	together to ensure that: 
	\begin{itemize}
		\item I correctly implemented the mathematical theory and numerical 
		algorithms.
		\item I made the code manageable and maintainable for future use.
	\end{itemize}
\end{itemize}
				
\bibliographystyle{plainnat}

\bibliography{SRS}

\newpage

\section{Appendix}

This is where you can place additional information.

\subsection{Symbolic Parameters}

The definition of the test cases will call for SYMBOLIC\_CONSTANTS.
Their values are defined in this section for easy maintenance.

\subsection{Usability Survey Questions?}

\wss{This is a section that would be appropriate for some projects.}

\begin{itemize}
	\item How long did it take to learn how to run the software? 
	\item Was it easy to interpret the output? 
	\item Was this program useful for your research? 
	\item What aspects of this software do you feel need improvement?
	\item Was the output recieved and processed in an adequate amount of time?
	\item How does this program compare with other software that finds this 
	particular spectrum?
\end{itemize}

\end{document}
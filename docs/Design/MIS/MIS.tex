\documentclass[12pt, titlepage]{article}

\usepackage{amsmath, mathtools}

\usepackage[round]{natbib}
\usepackage{amsfonts}
\usepackage{amssymb}
\usepackage{graphicx}
\usepackage{colortbl}
\usepackage{xr}
\usepackage{hyperref}
\usepackage{longtable}
\usepackage{xfrac}
\usepackage{tabularx}
\usepackage{float}
\usepackage{siunitx}
\usepackage{booktabs}
\usepackage{multirow}
\usepackage[section]{placeins}
\usepackage{caption}
\usepackage{fullpage} 

\hypersetup{
bookmarks=true,     % show bookmarks bar?
colorlinks=true,       % false: boxed links; true: colored links
linkcolor=red,          % color of internal links (change box color with linkbordercolor)
citecolor=blue,      % color of links to bibliography
filecolor=magenta,  % color of file links
urlcolor=cyan          % color of external links
}

\usepackage{array}

\externaldocument{../../SRS/SRS}

%% Comments

\usepackage{color}

\newif\ifcomments\commentstrue

\ifcomments
\newcommand{\authornote}[3]{\textcolor{#1}{[#3 ---#2]}}
\newcommand{\todo}[1]{\textcolor{red}{[TODO: #1]}}
\else
\newcommand{\authornote}[3]{}
\newcommand{\todo}[1]{}
\fi

\newcommand{\wss}[1]{\authornote{blue}{SS}{#1}}
\newcommand{\an}[1]{\authornote{magenta}{Author}{#1}}



\newcommand{\progname}{Program Name}

\begin{document}

\title{Module Interface Specification for SpecSearch}

\author{Robert E. White}

\date{\today}

\maketitle

\pagenumbering{roman}

\section{Revision History}

\begin{tabularx}{\textwidth}{p{3cm}p{2cm}X}
\toprule {\bf Date} & {\bf Version} & {\bf Notes}\\
\midrule
2018-11-09 & 1.0 & Creation of first draft for presentation. \\
\bottomrule
\end{tabularx}

~\newpage

\section{Symbols, Abbreviations and Acronyms}

See SRS Documentation at \wss{give url}

\wss{Also add any additional symbols, abbreviations or acronyms}

\newpage

\tableofcontents

\newpage

\pagenumbering{arabic}

\section{Introduction}

The following document details the Module Interface Specifications for
SpecSearch. SpecSearch is scientific computing software that computes and plots 
the spectrum of a matrix operator that appears in a LAX pair that is compatible 
for solutions of the Non-Linear Schr\dots{o}dinger (NLS) equation. 
Consequently, this spectrum carries useful information regarding the stability 
of solutions to the NLS equation. Physicists are interested in this spectrum as 
the NLS equation models many physical phenomena, such as rogue waves or 
modulated wave packets. Mathematicans are interested in the analytical 
behaviour of this spectrum. 

Complementary documents include the System Requirement Specifications
and Module Guide.  The full documentation and implementation can be
found at \url{https://github.com/whitere123/CAS741_REW}.

The purpose of this document is to describe the intended behaviour of the 
access routines for SpecSearch's modules. In particular, mathematical notation 
will be used to describe the input/output relationships and external behaviour 
of the 
modules. The MIS is still abstract since it does not cover implementation of 
the modules and does not outline any code.  
\section{Notation}

The structure of the MIS for modules comes from \citet{HoffmanAndStrooper1995},
with the addition that template modules have been adapted from
\cite{GhezziEtAl2003}.  The mathematical notation comes from Chapter 3 of
\citet{HoffmanAndStrooper1995}.  For instance, the symbol := is used for a
multiple assignment statement and conditional rules follow the form $(c_1
\Rightarrow r_1 | c_2 \Rightarrow r_2 | ... | c_n \Rightarrow r_n )$.

The following table summarizes the primitive data types used by \progname. 

\begin{center}
\renewcommand{\arraystretch}{1.2}
\noindent 
\begin{tabular}{l l p{7.5cm}} 
\toprule 
\textbf{Data Type} & \textbf{Notation} & \textbf{Description}\\ 
\midrule
character & char & a single symbol or digit\\
integer & $\mathbb{Z}$ & a number without a fractional component in (-$\infty$, $\infty$) \\
natural number & $\mathbb{N}$ & a number without a fractional component in [1, $\infty$) \\
real & $\mathbb{R}$ & any number in (-$\infty$, $\infty$)\\
complex & $\mathbb{C}$ &any number $x + iy$ with $x \in \mathbb{R}$ , $y \in 
\mathbb{R}$ and $i^{2} = -1$. \\
\bottomrule
\end{tabular} 
\end{center}

\noindent
The specification of \progname \ uses some derived data types: sequences, strings, and
tuples. Sequences are lists filled with elements of the same data type. Strings
are sequences of characters. Tuples contain a list of values, potentially of
different types. In addition, \progname \ uses functions, which
are defined by the data types of their inputs and outputs. Local functions are
described by giving their type signature followed by their specification.

\section{Module Decomposition}

The following table is taken directly from the Module Guide document for this project.

\begin{table}[h!]
\centering
\begin{tabular}{p{0.3\textwidth} p{0.6\textwidth}}
\toprule
\textbf{Level 1} & \textbf{Level 2}\\
\midrule

{Hardware-Hiding} & ~ \\
\midrule

\multirow{7}{0.3\textwidth}{Behaviour-Hiding} 
& Input Parameters \\
& Output Format \\
& Spectrum Matrix \\
& Exact Eigenvalue Equations \\
& Spectrum Error Equation \\
& Numerical Parameters \\  
& Control \\ 
\midrule

\multirow{3}{0.3\textwidth}{Software Decision} & {Sequence Data Structure}\\
& Eigenvalue and Eigenvector Solver \\
& Diagonal Matrix Generator\\
& Elliptic Integral\\ 
& Elliptic Functions\\ 
& Plotting \\ 
& Linspace \\
\bottomrule

\end{tabular}
\caption{Module Hierarchy}
\label{TblMH}
\end{table}

\newpage

\section{MIS of Input Parameters} \label{IPM} 

The secrets of this module are the data structure for input parameters and 
methods for verifying input. 

\subsection{Module}

InParams

\subsection{Uses}

-

\subsection{Syntax}

\begin{center}
\begin{tabular}{p{3cm} p{3cm} p{3cm} >{\raggedright\arraybackslash}p{7cm}}
\toprule
\textbf{Name} & \textbf{In} & \textbf{Out} & \textbf{Exceptions} \\
\hline
Call\_params & $k : \mathbb{R}$ & - & NonNumericalError \\
& $N : \mathbb{N}$ &  &  \\
Verify\_params & - & - & BadkRange, BadNRange \\
k & - & $\mathbb{R}$ & - \\
N & - & $\mathbb{N}$ & - \\
\hline
\end{tabular}
\end{center}

\subsection{Semantics}

\subsubsection{State Variables}

$k : \mathbb{R}$ \\ 
$N : \mathbb{N}$ 

\subsubsection{Environment Variables}

InputParameters: The two values entered by the user. 

\subsubsection{Assumptions}

\begin{itemize}
	\item Call\_params is called before the values of any state variables will 
	be accessed.
	\item The user inputs the state variables in the following order: $(k,N)$
\end{itemize}

\subsubsection{Access Routine Semantics}

\noindent InParams.k():
\begin{itemize}
\item output: $out = k$
\item exception: None
\end{itemize}

\noindent InParams.N():
\begin{itemize}
	\item output: $out = N$
	\item exception: None
\end{itemize}

\noindent Call\_Params():
\begin{itemize}
	\item transition: The data is read sequentially from the command line. The 
	data are seperated by the return key. This data is used to populate the 
	previously mentioned state variables. 
	\item exception: If non-numerical data is entered. 
\end{itemize} 

\noindent Verify\_Params():
\begin{itemize}
	\item exception: exc:=
\end{itemize}  

\noindent $\neg (k \in (0,1)) \Rightarrow \ $ BadkRange\\
$\neg (N \in \mathbb{N}) \Rightarrow \ $ BadNRange

\subsubsection{Local Functions}
None
\newpage

\section{MIS of Output Format} 

The secret of this module is the format and structure of the output data.

\subsection{Module}

OutForm

\subsection{Uses}

Eigenvalue solver, Spectrum Error Equations and Numerical Parameters. 

\begin{center}
	\begin{tabular}{p{3cm} p{3cm} p{3cm} >{\raggedright\arraybackslash}p{7cm}}
		\toprule
		\textbf{Name} & \textbf{In} & \textbf{Out} & \textbf{Exceptions} \\
		\hline
		capture & $D_{j} :\mathbb{R}^{2n}$ & Spectrum & - \\ 
		 & $Er_{j} :\mathbb{R}^{2}$ &  & - \\ 
		 & $V_{j} :\mathbb{C}^{4nx4n}$ &  & - \\ 
		 & $Vl_{j} :\mathbb{R}^{4n}$ & & - \\
		Spectral & - & $\mathbb{C}^{4n(4n+2)}$ & - \\
		D & - & $\mathbb{R}^{2n}$ & - \\
		Er & - & $\mathbb{R}$ & - \\
		\hline
	\end{tabular}
\end{center}

\subsection{Semantics}

The j refers to the $j^{th}$ numerical method. $j \in {O(1), O(4), cheb}$.   

\subsubsection{State Variables}

$Spectral_{j}$: $\mathbb{C}^{4n(4n+2)}$ \\
$D_j \in \mathbb{R}^{2n}$ \\
$Er_j \in \mathbb{R}^{2}$ \\  

\subsubsection{Environment Variables}

None.

\subsubsection{Assumptions}

\begin{itemize}
	\item Eigenvalue solver, Spectrum Error Equations and Numerical Parameters 
	are called and their values are temporarily stored prior to OutForm.
\end{itemize}

\subsubsection{Access Routine Semantics}

\noindent OutForm.Dj():
\begin{itemize}
	\item output: $out = D_{j}$
	\item exception: None
\end{itemize}

\noindent OutForm.Erj():
\begin{itemize}
	\item output: $out = Er_{j}$
	\item exception: None
\end{itemize} 

\noindent OutForm.Spectralj():
\begin{itemize}
	\item output: $out = Spectral_{j}$
	\item exception: None
\end{itemize}

\noindent capture():
\begin{itemize}
	\item Transition: The data ($D_{j}$,$Er_{j}$,$V_{j}$,$Vl_{j}$) will be 
	captured from the other 
	modules and then stored into $Spectral$ for convienience. Spectral is a 
	data 
	structure with four cell components. The first four cells are
	$4n$ by $4n + 1$ matrices created as follows: the $4n$ by $1$ vector of 
	eigenvalues, $Vl_j$, will be transposed and turned into the $(4n+1)^{th}$ 
	row 
	of 
	spectral. The previous $4n$ rows of spectral will be the matrix,$ V_j$, 
	composed 
	of the eigenvectors for the $j$ numerical method. Each column of $V_j$ is 
	an eigenvector from the $j$ numerical method. D will be stored in Spectral 
	as the fourth 
	cell component.  
	\item Exception: None
\end{itemize}  

\subsubsection{Local Functions}

None

\newpage
\section{MIS of Spectrum Matrix} 

The structure of the spectrum matrix, its data entries, how 
it is created, and the numerical method for approximating its 
eigenfunctions are the secrets of this module.  

\subsection{Module}

SpecMat

\subsection{Uses}

This module uses Numerical Parameters. 

\subsection{Syntax}

\begin{center}
	\begin{tabular}{p{3cm} p{3cm} p{3cm} >{\raggedright\arraybackslash}p{7cm}}
		\toprule
		\textbf{Name} & \textbf{In} & \textbf{Out} & \textbf{Exceptions} \\
		\hline
		create & $n : \mathbb{N} $ &SpecMat01 & - \\ 
		 & $h : \mathbb{R} $& SpecMat04 & - \\ 
		 & $elipdn : \mathbb{R}^{2n} $ &SpecCheb & - \\ 
		 & $k : \mathbb{R} $ & & - \\
		SpecMatO1 & - & $\mathbb{R}^{4nx4n}$ & - \\
		SpecMatO4 & - & $\mathbb{R}^{4nx4n}$ & - \\ 
		SpecCheb & - & $\mathbb{R}^{4nx4n}$ & - \\
		\hline
	\end{tabular}
\end{center}

\subsection{Semantics}

\subsubsection{State Variables}
$SpecCheb : \mathbb{R}^{4nx4n}$\\ 
$SpecMat04 : \mathbb{R}^{4nx4n}$\\
$SpecMat01 : \mathbb{R}^{4nx4n}$

\subsubsection{Assumptions}

\begin{itemize}
	\item All of the numerical parameters are calculated correctly and used as 
	arguments in create. 
\end{itemize}

\subsubsection{Access Routine Semantics}

SpecMat.SpecMatO1:
\begin{itemize}
	\item output: out = SpecMatO1 
	\item exception: None 
\end{itemize}

SpecMat.SpecMatO4:
\begin{itemize}
	\item output: out = SpecMatO4
	\item exception: None
\end{itemize}

SpecMat.SpecFour:
\begin{itemize}
	\item output: out = SpecCheb 
	\item exception: None
\end{itemize}

SpecMat.create():
\begin{itemize}
	\item transition: The data (n,h,elipdn,k) will be captured from the other 
	modules and used to create the spectral matrices. There will be three 
	instances of the spectral matrix. Each corresponds do a different numerical 
	algorithm for approximating the eigenfunction derivatives. Top top left and 
	bottom right sections of the matrices are diagonal matrices with elements 
	equal to the elipdn function evaluated at points in the discretized domain. 
	The remain parts of the matrix are coffecients such that when multiplied by 
	the eigenfunctions will create the appropriate approximation of the 
	derivatives. 
	\item exception: None
\end{itemize}

\newpage

\section{MIS of Exact Eigenvalue Equations} 

The secrets of this module are the analytical expressions for the two real 
eigenvalues. 

\subsection{Module}

TheorEigenValues

\subsection{Uses}

This module uses input parameters. 

\subsection{Syntax}

\begin{center}
	\begin{tabular}{p{3cm} p{3cm} p{3cm} >{\raggedright\arraybackslash}p{7cm}}
		\toprule
		\textbf{Name} & \textbf{In} & \textbf{Out} & \textbf{Exceptions} \\
		\hline
		$\lambda_O1$ & $k \in \mathbb{R}$ & $\mathbb{R}$ & - \\
		$\lambda_O2$ & $k \in \mathbb{R}$ & $\mathbb{R}$ & - \\ 
		\hline
	\end{tabular}
\end{center}

\subsection{Semantics}

\subsubsection{State Variables}

None

\subsubsection{Environment Variables}

None

\subsubsection{Assumptions}

\begin{itemize}
	\item These are the eigenvalues computed from segal et al. Refer to (segal) 
	for more details regarding assumptions.
\end{itemize}

\subsubsection{Access Routine Semantics}


$lambda_O1$:
\begin{itemize}
	\item output: $\frac{1}{2} (1+\sqrt{1-k^{2}})$ 
	\item exception: None
\end{itemize}


\noindent $lambda_O2$
\begin{itemize} 
	\item output: $\frac{1}{2} (1-\sqrt{1-k^{2}})$ 
	\item exception: None
\end{itemize}


\newpage

\section{MIS of Spectrum Error Equation} 

The secret of this module is the equation for error between exact and 
approximated eigenvalues.

\subsection{Module}

ErrCalc

\subsection{Uses}

This module uses Exact Eigenvalue equations and Eigenvalue solver. 

\subsection{Syntax}

\begin{center}
	\begin{tabular}{p{3cm} p{6cm} p{3cm} >{\raggedright\arraybackslash}p{3cm}}
		\toprule
		\textbf{Name} & \textbf{In} & \textbf{Out} & \textbf{Exceptions} \\
		\hline
		$\lambda_O1$ & - & $\mathbb{R}$ & - \\
		$\lambda_O2$ & - & $\mathbb{R}$ & - \\ 
		$\lambda_C1$ & - & $\mathbb{R}$ & - \\
		$\lambda_C2$ & - & $\mathbb{R}$ & - \\
		$Err1$ & $\lambda_O1 : \mathbb{R}$ & $\mathbb{R}$ &- \\
		 & $\lambda_C1 : \mathbb{R}$ &  &- \\
		$Err2$ & $\lambda_C2 : \mathbb{R}$ & $\mathbb{R}$ &- \\
		 & $\lambda_C2 : \mathbb{R}$ &  &- \\
		\hline
	\end{tabular}
\end{center}

\subsection{Semantics}

\subsubsection{State Variables}

$\lambda_O1 : \mathbb{R}$ \\
$\lambda_O2 : \mathbb{R}$ \\
$\lambda_C1 : \mathbb{R}$ \\ 
$\lambda_C2 : \mathbb{R}$

\subsubsection{Environment Variables}

None

\subsubsection{Assumptions}

\begin{itemize}
	\item These are the eigenvalues computed from segal et al. Refer to (segal) 
	for more details regarding assumptions.
\end{itemize}

\subsubsection{Access Routine Semantics}


$Err1$:
\begin{itemize}
	\item output: $| \lambda_O1 - \lambda_C1| $ 
	\item exception: None
\end{itemize}


\noindent $Err2$
\begin{itemize} 
	\item output: $| \lambda_O2 - \lambda_C2| $ 
	\item exception: None
\end{itemize} 

\noindent $ErrCalc.\lambda_O1$:
\begin{itemize}
	\item output: out = $\lambda_O1$
	\item exception: None 
\end{itemize}

\noindent $ErrCalc.lambda_O2$:
\begin{itemize}
	\item output: out = $\lambda_O2$
	\item exception: None
\end{itemize}

\noindent $ErrCalc.\lambda_C1$:
\begin{itemize}
	\item output: out = $\lambda_C1$
	\item exception: None
\end{itemize}

\noindent $ErrCalc.\lambda_C2$:
\begin{itemize}
	\item output: out = $\lambda_C2$ 
	\item exception: None
\end{itemize}

\newpage 

\section{MIS of Numerical Parameters} 

The secrets of this module are the range of the eigenfunction domain, points in 
the 
periodic domain and equation for the numerical scaling factor that computes 
the 
eigenfunction derivatives. 

\subsection{Module}

Numpars

\subsection{Uses}

This module uses Elliptic Functions, Diagonal Matrix, linspace, Elliptic 
Integral and input parameters. 

\subsection{Syntax}

\begin{center}
	\begin{tabular}{p{3cm} p{3cm} p{7cm} >{\raggedright\arraybackslash}p{3cm}}
		\toprule
		\textbf{Name} & \textbf{In} & \textbf{Out} & \textbf{Exceptions} \\
		\hline
		CreateVars & $k :\mathbb{R}$ & 
		$xend: \mathbb{R}$ & - \\ 
		  &$N: \mathbb{N}$ & $Domain:\mathbb{R}^{2N}$ &  \\
		  &  & $ellipj: \mathbb{R}^{2N}$ &  \\ 
		  &  & $ellipMAT: \mathbb{R}^{2Nx2N}$ &  \\
		xend & - & $\mathbb{R}$ & - \\
		Domain & - & $\mathbb{R}^{2N}$ &- \\
		ellipjdn & - & $\mathbb{R}^{2N}$ &- \\
		ellipjMAT & - & $\mathbb{R}^{2Nx2N}$ &- \\
		h & - & $\mathbb{R}$ & -\\ 
		\hline
	\end{tabular}
\end{center}

\subsection{Semantics}

\subsubsection{State Variables}
$k :\mathbb{R}$ \\  
$Domain:\mathbb{R}^{2N}$ \\  
$ellipj: \mathbb{R}^{2N}$ \\ 
$ellipMAT: \mathbb{R}^{2Nx2N}$ \\  

\subsubsection{Environment Variables}

None

\subsubsection{Assumptions}

\begin{itemize}
	\item Input parameters is called before Numerical parameters. 
	\item Input parameters does not throw an exception. 
\end{itemize}

\subsubsection{Access Routine Semantics}


$Numpars.xend$:
\begin{itemize}
	\item output: $xend$ 
	\item exception: None
\end{itemize}


\noindent $Numpars.Domain$
\begin{itemize} 
	\item output: $Domain$ 
	\item exception: None
\end{itemize} 

\noindent $Numpars.ellipjdn$:
\begin{itemize}
	\item output: out = $ellipjdn$
	\item exception: None 
\end{itemize}

\noindent $Numpars.ellipjMAT$:
\begin{itemize}
	\item output: out = $ellipjMAT$
	\item exception: None
\end{itemize}

\noindent $Numpars.h$:
\begin{itemize}
	\item output: out = $h$ 
	\item exception: None
\end{itemize} 

\noindent $Numpars.CreateVars$:
\begin{itemize}
	\item transition: The data $(N,k)$ will be captured from the input 
	parameters and used to create the state variables. $k$ will be inputted as 
	an argument into the elliptic integral module. The resulting integral is 
	equal to $xend$. \\ 
	The domain is created using linspace. The endpoint arguments of linspace 
	are $-xend$ and $xend$, respectively. The distance between partition points 
	in the resulting domain is $h=\frac{xend}{N}$. Ellipjdn is derived from 
	from computing the ellipjdn value of each point in Domain. EllipjMAT is a 
	diagonal matrix whose diagonal is Ellipjdn. 
	\item exception: None
\end{itemize}

\newpage

\section{MIS of Eigenvalue and Eigenvector Solver} 

The secret of this module is the numerical algorithm for calculating the 
eigenvalues and eigenvectors of an $n$ by $n$ matrix.

\subsection{Module}

eig (\url{https://www.mathworks.com/help/matlab/ref/eig.html})

\subsection{Uses}

This module uses Spectrum Matrix. 

\subsection{Syntax}

\begin{center}
	\begin{tabular}{p{2cm} p{4cm} p{4cm} p{4cm}}
		\hline
		\textbf{Name} & \textbf{In} & \textbf{Out} & \textbf{Exceptions} \\
		\hline
		solver & $A : \mathbb{C}^{nxn}$  & 
		$\mathbb{C}^{n}$ x $\mathbb{R}^{nxn}$ & NotsquareMat \\
		\hline
	\end{tabular}
\end{center}

\subsection{Semantics}

\subsubsection{State Variables}

None.

\subsubsection{Environment Variables}

None.

\subsubsection{Assumptions}

\begin{itemize}
	\item The input is a square matrix. 
\end{itemize}

\subsubsection{Access Routine Semantics}

\noindent eig():
\begin{itemize}
	\item output: out:= $\lambda$ and $\bar{v}$ such that:\\
	$A\bar{v} = \lambda \bar{v}$ \\
	$\bar{v} : \mathbb{C}^{n}$ and $\lambda : \mathbb{C}$\\
	\item exception: exce:= $A \not\in \mathbb{R}^{nxn} \Rightarrow$ 
	NotSquareMatrix 
\end{itemize}

\newpage

\section{MIS of Diagonal Matrix} 

The secrets of this module are the numerical algorithm for creating an $n$ by 
$n$ diagonal 
matrix from an $n$ by 1 vector and the numerical algorithm for creating an $n$ 
by 1 vector from an $n$ by $n$ diagonal 
matrix. 

\subsection{Module}

diag (\url{https://www.mathworks.com/help/matlab/ref/diag.html})

\subsection{Uses}

This module uses Numerical Parameters. 

\subsection{Syntax}

\begin{center}
	\begin{tabular}{p{2cm} p{6cm} p{6cm} p{3cm}}
		\hline
		\textbf{Name} & \textbf{In} & \textbf{Out} & \textbf{Exceptions} \\
		\hline
		solver & $A :$ Diagonal $n$ by $n$ matrix  & $\mathbb{C}^{n}$ & 
		NotDiagMat \\ 
		solver & $v :\mathbb{C}^{n}$  & Diagonal $n$ by $n$ matrix & 
		NotVector \\
		\hline
	\end{tabular}
\end{center}

\subsection{Semantics}

\subsubsection{State Variables}

None.

\subsubsection{Environment Variables}

None.

\subsubsection{Assumptions}

\begin{itemize}
	\item The input is a square matrix or column vector.  
\end{itemize}

\subsubsection{Access Routine Semantics}

\noindent diag.solver():
\begin{itemize}
	\item output (if In=A): out:= v such that:\\
	$A(j,j) = v(j)$ for j=1,2,...,n \\
	\item exception (if In=A): exce:= $A \not\in (Diagonal Matrix) \Rightarrow$ 
	NotSquareMatrix 
	\item output (if In=v): out:= A such that:\\
	$A(j,j) = v(j)$ for j=1,2,...,n and zero else. \\
	\item exception (if In=v): exce:= $v \not\in \mathbb{C}^{n} \Rightarrow$ 
	Not vector 
\end{itemize}

\newpage 

\section{MIS of Elliptic Integral} 

The secret of this module is the the numerical algorithm for calculating the 
complete 
elliptic integral for some real constant $k$.

\subsection{Module}

ellipK (\url{https://www.mathworks.com/help/symbolic/elliptick.html})

\subsection{Uses}

This module uses Numerical Parameters. 

\subsection{Syntax}

\begin{center}
	\begin{tabular}{p{2cm} p{6cm} p{6cm} p{3cm}}
		\hline
		\textbf{Name} & \textbf{In} & \textbf{Out} & \textbf{Exceptions} \\
		\hline
		solver & $k : \mathbb{R}$ & $\mathbb{R}$ & 
		NonNumeric \\ 
		\hline
	\end{tabular}
\end{center}

\subsection{Semantics}

\subsubsection{State Variables}

None.

\subsubsection{Environment Variables}

None.

\subsubsection{Assumptions}

\begin{itemize}
	\item The argument is a real number. 
	\item The user understands the physical context of the number k.  
\end{itemize}

\subsubsection{Access Routine Semantics}

\noindent ellipK.solver():
\begin{itemize}
	\item output : \\
	$$ 
	\int_{0}^{\frac{\pi}{2}} \frac{dx}{\sqrt{1-ksin^{2}(x)}}$$ \\
	\item exception : exce:= $\neg(k\in \mathbb{R}) \Rightarrow$ 
	NonNumeric
\end{itemize}

\newpage

\section{MIS of Elliptic Functions} 

The secret of this module is the the numerical algorithm for calculating the 
values of the elliptic functions.

\subsection{Module}

ellipj (\url{https://www.mathworks.com/help/matlab/ref/ellipj.html})

\subsection{Uses}

This module uses Numerical Parameters. 

\subsection{Syntax}

\begin{center}
	\begin{tabular}{p{2cm} p{6cm} p{6cm} p{3cm}}
		\hline
		\textbf{Name} & \textbf{In} & \textbf{Out} & \textbf{Exceptions} \\
		\hline
		solver & $X : \mathbb{R}^{n}$ & $\mathbb{R}^{3xn}$ & 
		BadVal \\ 
		 & $k : \mathbb{R}$ &  &  \\ 
		\hline
	\end{tabular}
\end{center}

\subsection{Semantics}

\subsubsection{State Variables}

None.

\subsubsection{Environment Variables}

None.

\subsubsection{Assumptions}

None.

\subsubsection{Access Routine Semantics}

\noindent ellipj.solver():
\begin{itemize}
	\item output : $Y: \mathbb{R}^{n}$ such that $Y(1,j)= sn(X(j),k)$, 
	$Y(2,j)=cn(X(j),k)$ and $Y(3,j)=dn(X(j),k)$. See Module link for more 
	detail regarding jacobi ellptic functions sn,dn and cn. 
	\item exception : exce:= $\neg(X \in \mathbb{R}^{n}) \Rightarrow$ 
	BadVal
\end{itemize}

\newpage 

\section{MIS of Plotting} 

The secret of this module is the plotting methods/algorithms. 

\subsection{Module}

plot (\url{https://www.mathworks.com/help/matlab/ref/plot.html})

\subsection{Uses}

This module uses Output. 

\subsection{Syntax}

\begin{center}
	\begin{tabular}{p{2cm} p{6cm} p{6cm} p{3cm}}
		\hline
		\textbf{Name} & \textbf{In} & \textbf{Out} & \textbf{Exceptions} \\
		\hline
		plot & $X : \mathbb{R}^{n}$ & graph(Y) & 
		BadVal, DimensionErr \\ 
		& $Y : \mathbb{R}^{n}$ &  &  \\ 
		\hline
	\end{tabular}
\end{center}

\subsection{Semantics}

\subsubsection{State Variables}

None.

\subsubsection{Environment Variables}

None.

\subsubsection{Assumptions}

None.

\subsubsection{Access Routine Semantics}

\noindent plot.plot():
\begin{itemize}
	\item Transition: Plot accepts two vectors as input. A figure is created 
	with the following properties. The domain ranges from the minimum of X to 
	the maximum of X. The range ranges from the minimum of Y to the maximum of 
	Y. The pairs $(X(j),Y(j))$ for $j=1,2,...,n$ are plotted on a figure with 
	respect to the aforementioned axes. 
	\item exception : exce1:= $\neg(X,Y \in \mathbb{C}^{n}) \Rightarrow$ 
	BadVal \\
	exce2:= $\neg (dim(Y) = dim(X)) \Rightarrow$ DimensionErr \\
\end{itemize}

\newpage 

\section{MIS of Linspace} 

The secret of this module is the software algorithm for creating a vector with 
equally spaced entries. 

\subsection{Module}

Linspace (\url{https://www.mathworks.com/help/matlab/ref/linspace.html})

\subsection{Uses}

This module uses Numerical Parameters. 

\subsection{Syntax}

\begin{center}
	\begin{tabular}{p{2cm} p{6cm} p{6cm} p{3cm}}
		\hline
		\textbf{Name} & \textbf{In} & \textbf{Out} & \textbf{Exceptions} \\
		\hline
		create & $a : \mathbb{R}$ & $\mathbb{R}^{c}$ & 
		BadVal,notInt \\ 
		& $b : \mathbb{R}$ &  &  \\  
		& $c : \mathbb{N}$ &  &  \\ 
		\hline
	\end{tabular}
\end{center}

\subsection{Semantics}

\subsubsection{State Variables}

None.

\subsubsection{Environment Variables}

None.

\subsubsection{Assumptions}

None.

\subsubsection{Access Routine Semantics}

\noindent linspace.create():
\begin{itemize}
	\item output : $X \in \mathbb{R}^{c}$ such that X(1)=a, X(n)=b and 
	$|x(k)-x(k-1)|=\frac{b-a}{n-1}$ for $k \in {2,3,4,..,n}$. 
	\item exception : exce:= $\neg(c \in \mathbb{N}) \Rightarrow$ 
	notInt 
	\item exception : exce:= $\neg(a: \mathbb{R}) \Rightarrow$ BadVal 
	\item exception : exce:= $\neg(b: \mathbb{R}) \Rightarrow$ BadVal
\end{itemize}

\newpage 

\section{MIS of Control} 

The secret of this module is the algorithm that coordinates the overall program 
and 
interaction between modules.

\subsection{Module}

Main

\subsection{Uses}

This module uses Input Parameters, Numerical Parameters, Spectrum Matrix, 
Eigenvalue solver, Spectrum Error , Output and plotting.

\subsection{Syntax}

\begin{center}
	\begin{tabular}{p{2cm} p{6cm} p{6cm} p{3cm}}
		\hline
		\textbf{Name} & \textbf{In} & \textbf{Out} & \textbf{Exceptions} \\
		\hline
		main & - & - & - \\ 
		\hline
	\end{tabular}
\end{center}

\subsection{Semantics}

\subsubsection{State Variables}

None.

\subsubsection{Environment Variables}

None.

\subsubsection{Assumptions}

None.

\subsubsection{Access Routine Semantics}

\noindent main():
\begin{itemize}
	\item transition: This function controls the running of the scientific 
	software. First, input is taken from the user. This input is used to create 
	useful numerical parameters that will drive the rest of the calculations. 
	The input is brought to the numerical parameters module where useful 
	constants, matrices and elliptic function values are calculated. The state 
	variables from Numerical parameters are used as arguments in Spectrum 
	Matrix. The spectrum (eigenvalues) can be calculated once the spectrum 
	matrix is created. These eigenvalues are plotted and their deviation from 
	theoretical values are computed (err modules). 
\end{itemize}

\newpage 


\bibliographystyle {plainnat}
\bibliography {../../../ReferenceMaterial/References}

\newpage

\section{Appendix} \label{Appendix}

\wss{Extra information if required}

\end{document}
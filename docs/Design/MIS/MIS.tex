\documentclass[12pt, titlepage]{article}

\usepackage{amsmath, mathtools}

\usepackage[round]{natbib}
\usepackage{amsfonts}
\usepackage{amssymb}
\usepackage{graphicx}
\usepackage{colortbl}
\usepackage{xr}
\usepackage{hyperref}
\usepackage{longtable}
\usepackage{xfrac}
\usepackage{tabularx}
\usepackage{float}
\usepackage{siunitx}
\usepackage{booktabs}
\usepackage{multirow}
\usepackage[section]{placeins}
\usepackage{caption}
\usepackage{fullpage} 
\usepackage{amsmath}

\hypersetup{
bookmarks=true,     % show bookmarks bar?
colorlinks=true,       % false: boxed links; true: colored links
linkcolor=red,          % color of internal links (change box color with linkbordercolor)
citecolor=blue,      % color of links to bibliography
filecolor=magenta,  % color of file links
urlcolor=cyan          % color of external links
}

\usepackage{array}

\externaldocument{../../SRS/SRS}

\input{../../Comments}

\newcommand{\progname}{SpecSearch}

\begin{document}

\title{Module Interface Specification for \progname}

\author{Robert E. White}

\date{\today}

\maketitle

\pagenumbering{roman}

\section{Revision History}

\begin{tabularx}{\textwidth}{p{3cm}p{2cm}X}
\toprule {\bf Date} & {\bf Version} & {\bf Notes}\\
\midrule
2018-11-09 & 1.0 & Created the first draft for my MIS presentation. \\ 
2018-11-15 & 1.1 & Updated the presentation draft based on feedback from CAS 
741 class. 
Completed half of the module interface specifications. \\ 
2018-11-20 & 1.2 & Completion of the module interface specifications. 
Completion of section 2. \\
2018-11-22 & 1.3 & Edit of 1.2. First submission. \\ 
2018-12-09 & 1.4 & Creation of final draft for final submission. \\ 
\bottomrule
\end{tabularx}

~\newpage

\section{Symbols, Abbreviations and Acronyms} \label{abbrev}

See SRS Documentation at \url{https://github.com/whitere123/CAS741_REW}. 

\section*{Abbreviations and Acronyms}

\renewcommand{\arraystretch}{1.2}
\begin{tabular}{l l} 
	\toprule		
	\textbf{symbol} & \textbf{description}\\
	\midrule 
	SRS & System Requirements Specification\\ 
	$:$ & ``Contained in"\\ 
	$\neg$ & Negation\\ 
	$\Rightarrow$ & ``It follows"\\ 
	$cn$ & Elliptic cosine function\\ 
	$sn$ & Elliptic sine function\\ 
	$dn$ & Elliptic delta function \\ 
	NLS & Non-linear Schr\"{o}dinger\\  
	$O(8)$ & Eigth order central numerical differentiation method \\ 
	$O(10)$ & Tenth order central numerical differentiation method \\ 
	$O(12)$ & Twelfth numerical differentiation method\\
	dim(in) & The dimension of item ``in" \\
	$\notin$ & Not an element of\\
	\bottomrule
\end{tabular}\\

\newpage

\tableofcontents

\newpage

\pagenumbering{arabic}

\section{Introduction}

The following document details the Module Interface Specifications for
SpecSearch. It will describe the intended behaviour of the 
access routines in SpecSearch's modules. Mathematical notation 
will be used to explain the external 
behaviour 
of the modules and the relationships between input and output. The MIS is still 
abstract since it does not outline any code. 

SpecSearch is scientific computing software that computes and plots 
the spectrum of a matrix operator from a LAX pair that is compatible 
for solutions of the Non-Linear Schr\"{o}dinger (NLS) equation. 
This spectrum carries useful information regarding the stability 
of solutions to the NLS equation. Physicists are interested in this spectrum 
because 
the NLS equation models physical phenomena; such as rogue waves or 
modulated wave packets. Mathematicans are interested in the analytical 
behaviour of this spectrum. 

Complementary documents include the System Requirement Specifications
and Module Guide.  The full documentation and implementation can be
found at \url{https://github.com/whitere123/CAS741_REW}.
  
\section{Notation}

The structure of the MIS for modules comes from \citet{HoffmanAndStrooper1995},
with the addition that template modules have been adapted from
\cite{GhezziEtAl2003}.  The mathematical notation comes from Chapter 3 of
\citet{HoffmanAndStrooper1995}.  For instance, the symbol := is used for a
multiple assignment statement and conditional rules follow the form $(c_1
\Rightarrow r_1 | c_2 \Rightarrow r_2 | ... | c_n \Rightarrow r_n )$.

The following table summarizes the primitive data types used by \progname. 

\begin{center}
\renewcommand{\arraystretch}{1.2}
\noindent 
\begin{tabular}{l l p{7.5cm}} 
\toprule 
\textbf{Data Type} & \textbf{Notation} & \textbf{Description}\\ 
\midrule
character & char & a single symbol or digit\\
integer & $\mathbb{Z}$ & a number without a fractional component in (-$\infty$, $\infty$) \\
natural number & $\mathbb{N}$ & a number without a fractional component in [1, $\infty$) \\
real & $\mathbb{R}$ & any number in (-$\infty$, $\infty$)\\
complex & $\mathbb{C}$ &any number $x + iy$ with $x \in \mathbb{R}$ , $y \in 
\mathbb{R}$ and $i^{2} = -1$. \\ 
real matrix &$\mathbb{R}^{mxn}$ & An $m$ by $n$ Matrix with real elements.\\ 
complex matrix &$\mathbb{C}^{mxn}$ & An $m$ by $n$ Matrix with complex 
elements.\\
\bottomrule
\end{tabular} 
\end{center}

\noindent
The specification of \progname \ uses some derived data types: sequences, strings, and
tuples. Sequences are lists filled with elements of the same data type. Strings
are sequences of characters. Tuples contain a list of values, potentially of
different types. In addition, \progname \ uses functions, which
are defined by the data types of their inputs and outputs. Local functions are
described by giving their type signature followed by their specification.

\section{Module Decomposition}

The following table is taken directly from the Module Guide document for this project.

\begin{table}[h!]
\centering
\begin{tabular}{p{0.3\textwidth} p{0.6\textwidth}}
\toprule
\textbf{Level 1} & \textbf{Level 2}\\
\midrule

{Hardware-Hiding} & ~ \\
\midrule

\multirow{7}{0.3\textwidth}{Behaviour-Hiding} 
& Input Parameters \\
& Output Format \\
& Spectrum Matrix \\
& Exact Eigenvalue Equations \\
& Numerical Parameters \\  
& Control \\ 
\midrule

\multirow{3}{0.3\textwidth}{Software Decision} 
& Eigenvalue and Eigenvector Solver \\
& Diagonal Matrix Generator\\
& Elliptic Integral\\ 
& Elliptic Functions\\ 
& Plotting \\ 
& Linspace \\
& Toeplitz \\ 
\bottomrule

\end{tabular}
\caption{Module Hierarchy}
\label{TblMH}
\end{table}

\newpage

\section{MIS of Input Parameters} \label{IPM} 

The secrets of this module are the methods for verifying input and the data 
structure for storing the inputed parameters. 

\subsection{Module} 

InParams

\subsection{Uses}

-

\subsection{Syntax}

\begin{center}
\begin{tabular}{p{3cm} p{3cm} p{3cm} >{\raggedright\arraybackslash}p{7cm}}
\toprule
\textbf{Name} & \textbf{In} & \textbf{Out} & \textbf{Exceptions} \ref{Appendix} 
\\
\hline
Load\_params & $k : \mathbb{R}$ & - & NonNumericalError \\
& $N : \mathbb{N}$ &  &  \\ 
& $P : \{2,4\}$ &  &  \\
Verify\_params & - & - & BadkRange, BadNRange, BadPRange \\
k & - & $\mathbb{R}$ & - \\
N & - & $\mathbb{N}$ & - \\
\hline
\end{tabular}
\end{center}

\subsection{Semantics}

\subsubsection{State Variables}

$k : \mathbb{R}$ \\ 
$N : \mathbb{N}$ \\
$P : \{2,4\}$
\subsubsection{Environment Variables}

InputParameters: The three values entered by the user $(k,N,P)$. 

\subsubsection{Assumptions}

\begin{itemize}
	\item Load\_params is called before the values of any state variables are 
	accessed.
	\item The user inputs the environment variables in the following order: 
	$(k,N,P)$.
\end{itemize}

\subsubsection{Access Routine Semantics}

\noindent InParams.k():
\begin{itemize}
\item output: $out = k$
\item exception: None
\end{itemize}

\noindent InParams.N():
\begin{itemize}
	\item output: $out = N$
	\item exception: None
\end{itemize}

\noindent InParams.P():
\begin{itemize}
	\item output: $out = P$
	\item exception: None
\end{itemize}

\noindent Load\_Params():
\begin{itemize}
	\item transition: The data is read as a vector from the command line. The 
	three elements of the vector correspond to $(k,N,P)$ , respectively. This 
	data is used to populate the 
	previously mentioned state variables. 
	\item exception: exec:=NonNumericalError if non-numerical data is entered. 
\end{itemize} 

\noindent Verify\_Params():
\begin{itemize}
	\item exception: exc:=
\end{itemize}  

\noindent 
$\neg (k : [0,1]) \Rightarrow \ $ BadkRange\\
$\neg (N : \mathbb{N})$ and $\neg(N>20) \Rightarrow \ $ BadNRange\\
$\neg (P : \{2,4\} ) \Rightarrow \ $ BadPRange

\subsubsection{Local Functions}
None
\newpage

\section{MIS of Output Format} \label{OFM}

The secret of this module is the structure used for storing the output data.

\subsection{Module}

OutForm

\subsection{Uses}

Eigenvalue and Eigenvector solver \ref{SEES}, Numerical 
Parameters \ref{MNP} and Exact Eigenvalue Equations \ref{SEE}. 

\subsection{Syntax} 

\begin{center}
	\begin{tabular}{p{3cm} p{3cm} p{3cm} >{\raggedright\arraybackslash}p{7cm}}
		\toprule
		\textbf{Name} & \textbf{In} & \textbf{Out} & \textbf{Exceptions}  \\
		\hline
		capture & $\lambda_{o}1 :\mathbb{C}$ & out & - \\ 
		 & $\lambda_{o}2 :\mathbb{C}$ &  & - \\ 
		 & $\lambda_{o}3 :\mathbb{C}$ &  & - \\
		 & $\lambda_{o}4 :\mathbb{C}$ &  & - \\
		 & $\Lambda : \mathbb{R}^{4n x 6}$ & & -\\
		label & - & char & - \\ 
		theor & - & $\mathbb{R}^{2 x 6}$ & - \\
		$\Lambda$ & - & $\mathbb{R}^{4n x 6}$ & - \\
		\hline
	\end{tabular}
\end{center}

\subsection{Semantics}

\subsubsection{State Variables}

$label$ : $char$ \\
$theor: \mathbb{R}^{2 x 6}$ \\
$\Lambda : \mathbb{R}^{4nx6}$

\subsubsection{Environment Variables}

None.

\subsubsection{Assumptions}

\begin{itemize}
	\item Eigenvalue and Eigenvector solver, 
	Numerical Parameters and exact eigenvalue equations 
	are called and their outputs are fed into Outform.capture() 
	before any of the state variables are accessed. 
\end{itemize}

\subsubsection{Access Routine Semantics}

\noindent OutForm.$label$():
\begin{itemize}
	\item output: $out = label$
	\item exception: None
\end{itemize}


\noindent OutForm.$theor$():
\begin{itemize}
	\item output: $out = theor$
	\item exception: None
\end{itemize} 

\noindent OutForm.$\Lambda$():
\begin{itemize}
	\item output: $out = \Lambda$
	\item exception: None
\end{itemize}

\noindent capture():
\begin{itemize}
	\item Transition: The eigenvalues from the six spectral matrices, $\Lambda$ 
	,
	and theoretical eigenvalues, $\lambda_{o}i$ ,are put into an OutForm data 
	structure. Each state variable is a six element cell structure. The $ith$ 
	cell of each state variable corresponds to one of the six spectral 
	matrices (ie for cn order 10). OutForm.label$\{i\}$ is the 
	$i^{th}$ 
	spectral matrices' plot title. OutForm.theor$\{i\}$ are the $i^{th}$ 
	spectral 
	matrices' theoretically calculated eigenvalues. OutForm.$\Lambda$$\{i\}$ is 
	the 
	$i^{th}$ spectral matrices' spectrum.This structure is convienient for 
	creating all six plots in a 
	for loop. 
	\item Exception: None
\end{itemize}  

\subsubsection{Local Functions}

None

\newpage
\section{MIS of Spectrum Matrix} \label{SSM}

The structure of the spectrum matrix, its data entries, how 
it is created, and the numerical method for approximating its 
eigenfunctions are the secrets of this module.  

\subsection{Module}

SpecMat

\subsection{Uses}

This module uses Numerical Parameters \ref{MNP}. 

\subsection{Syntax}

\begin{center}
	\begin{tabular}{p{3cm} p{5cm} p{3cm} >{\raggedright\arraybackslash}p{5cm}}
		\toprule
		\textbf{Name} & \textbf{In} & \textbf{Out} & \textbf{Exceptions} \\
		\hline
		SpecMat & $n : \mathbb{N} $ &$SpecMatdn_{O(8)}$ & - \\ 
		 & $h : \mathbb{R} $& $SpecMatdn_{O(10)}$ & - \\ 
		 & $elipMAT_{cn} : \mathbb{R}^{2nx2n} $ &$SpecMatdn_{O(12)}$ & - \\ 
		 & $elipMAT_{dn} : \mathbb{R}^{2nx2n} $ &$SpecMatcn_{O(8)}$ & - \\ 
		 & $k : \mathbb{R} $ & $SpecMatcn_{O(10)}$& - \\
		 &  &  $SpecMatcn_{O(12)}$ &- \\
		$SpecMatdn_{O(8)}$ & - & $\mathbb{R}^{4nx4n}$ & - \\
		$SpecMatdn_{O(10)}$ & - & $\mathbb{R}^{4nx4n}$ & - \\ 
		$SpecMatdn_{O(12)}$ & - & $\mathbb{R}^{4nx4n}$ & - \\
		$SpecMatcn_{O(8)}$ & - & $\mathbb{R}^{4nx4n}$ & - \\
		$SpecMatcn_{O(10)}$ & - & $\mathbb{R}^{4nx4n}$ & - \\ 
		$SpecMatcn_{O(12)}$ & - & $\mathbb{R}^{4nx4n}$ & - \\
		\hline
	\end{tabular}
\end{center}

\subsection{Semantics}

\subsubsection{State Variables}
None

\subsubsection{Environment Variables}

None. 

\subsubsection{Assumptions}

\begin{itemize}
	\item Numerical parameters is called, its state variables are validated and 
	its state variables are fed into SpecMat() before the SpecMat function is 
	called. 
\end{itemize}

\subsubsection{Access Routine Semantics}

\noindent
SpecMat():
\begin{itemize}
	\item transition: The data (n,h,elipMAT,k) will be captured from numerical 
	parameters and used to create the spectral matrices. There will be six 
	spectral matrices. Each matrix corresponds to one of three 
	numerical 
	algorithms for approximating the eigenfunction derivatives and one of two 
	boundary wave solutions to the NLS equation. The top right 
	and 
	bottom left quadrants of the spectral matrices are diagonal matrices with 
	diagonal  
	elements 
	equal to the negative elipdn (or cn) function (\ref{MELF}) evaluated at 
	points in 
	the 
	discretized domain \ref{MNP}. 
	The remaining quadrants of the matrix are equal to $-NUM_{j}$ or 
	$+NUM_{j}$. $NUM_{j}$ is an extremely large matrix and its explicit form 
	will be omitted from this document. It is a matrix of coefficients that 
	when multiplied by an eigenvector procude a matrix whose entries are the 
	$j^{th}$ order derivative's central difference approximation. For more 
	details about $NUM_{j}$ see 
	\cite{grasPel}. For the derivation of the spectral matrix please see the 
	"justification of Compatibility conditions and reduction to spectral 
	problem" under the Instance Model 1 (IM1) chart of the SRS document 
	\url{https://github.com/whitere123/CAS741_REW} .\\
Therefore the output matrices are: 

 \[
SpecMatdn_{O(8)}=
\left[ {\begin{array}{cc}
	NUM_{O(8)} & -EllipMat_{dn} \\
	-EllipMat_{dn} & -NUM_{O(8)} \\
	\end{array} } \right]
\] 

 \[
SpecMatdn_{O(10)}=
\left[ {\begin{array}{cc}
	NUM_{O(10)} & -EllipMat_{dn} \\
	-EllipMat_{dn} & -NUM_{O(10)} \\
	\end{array} } \right]
\] 

 \[
SpecMatdn_{O(12)}=
\left[ {\begin{array}{cc}
	NUM_{O(12)} & -EllipMat_{dn} \\
	-EllipMat_{dn} & -NUM_{O(12)} \\
	\end{array} } \right]
\] 

 \[
SpecMatcn_{O(8)}=
\left[ {\begin{array}{cc}
	NUM_{O(8)} & -EllipMat_{cn} \\
	-EllipMat_{cn} & -NUM_{O(8)} \\
	\end{array} } \right]
\] 

\[
SpecMatcn_{O(10)}=
\left[ {\begin{array}{cc}
	NUM_{O(10)} & -EllipMat_{cn} \\
	-EllipMat_{cn} & -NUM_{O(10)} \\
	\end{array} } \right]
\] 

\[
SpecMatcn_{O(12)}=
\left[ {\begin{array}{cc}
	NUM_{O(12)} & -EllipMat_{cn} \\
	-EllipMat_{cn} & -NUM_{O(12)} \\
	\end{array} } \right]
\] 
	\item exception: None
\end{itemize}

\subsubsection{Local Functions} 

None. 

\newpage

\section{MIS of Exact Eigenvalue Equations} \label{SEE}

The secrets of this module are the analytical expressions for the theoretical 
eigenvalues corresponding to the cn and dn boundary wave solutions.

\subsection{Module}

TheorEigenValues

\subsection{Uses}

This module uses input parameters \ref{IPM}. 

\subsection{Syntax}

\begin{center}
	\begin{tabular}{p{3cm} p{3cm} p{3cm} >{\raggedright\arraybackslash}p{7cm}}
		\toprule
		\textbf{Name} & \textbf{In} & \textbf{Out} & \textbf{Exceptions} \\
		\hline
		$\lambda_O1$ & $k : \mathbb{R}$ & $\mathbb{R}$ & - \\
		$\lambda_O2$ & $k : \mathbb{R}$ & $\mathbb{R}$ & - \\ 
		$\lambda_O3$ & $k : \mathbb{R}$ & $\mathbb{C}$ & - \\
		$\lambda_O4$ & $k : \mathbb{R}$ & $\mathbb{C}$ & - \\ 
		\hline
	\end{tabular}
\end{center}

\subsection{Semantics}

\subsubsection{State Variables}

None

\subsubsection{Environment Variables}

None

\subsubsection{Assumptions}

\begin{itemize}
	\item These are the eigenvalues computed from \citet{SegaletAl}. 
\end{itemize}

\subsubsection{Access Routine Semantics}


$\lambda_O1$:
\begin{itemize}
	\item output: $\frac{1}{2} (1+\sqrt{1-k^{2}})$ 
	\item exception: None
\end{itemize}


\noindent $\lambda_O2$
\begin{itemize} 
	\item output: $\frac{1}{2} (1-\sqrt{1-k^{2}})$ 
	\item exception: None
\end{itemize} 

$\lambda_O3$:
\begin{itemize}
	\item output: $\frac{1}{2} (k+i\sqrt{1-k^{2}})$ 
	\item exception: None
\end{itemize}


\noindent $\lambda_O4$
\begin{itemize} 
	\item output: $\frac{1}{2} (k-i\sqrt{1-k^{2}})$ 
	\item exception: None
\end{itemize}

\subsubsection{Local Functions} 

None. 

\newpage

\section{MIS of Numerical Parameters} \label{MNP}

The secrets of this module are the range of the eigenfunction domain, points in 
the 
periodic domain and equation for the numerical scaling factor that computes 
the 
eigenfunction derivatives. 

\subsection{Module}

Numpars

\subsection{Uses}

This module uses Elliptic Functions \ref{MELF}, Diagonal Matrix \ref{MDM}, 
linspace \ref{MLIN}, Elliptic 
Integral \ref{MEI} and input parameters \ref{IPM}. 

\subsection{Syntax}

\begin{center}
	\begin{tabular}{p{3cm} p{3cm} p{7cm} >{\raggedright\arraybackslash}p{3cm}}
		\toprule
		\textbf{Name} & \textbf{In} & \textbf{Out} & \textbf{Exceptions} \\
		\hline
		Numpars & $k :\mathbb{R}$ & 
		$xend$ & - \\ 
		  &$N: \mathbb{N}$ & $Domain$ &  \\
		  &$P:\{2,4\}$  & $ellipjdn$ &  \\ 
		  &  & $ellipjcn$ &  \\ 
		  &  & $ellipMAT_{dn}$ &  \\
		  &  & $ellipMAT_{cn}$ &  \\ 
		   &  & $h$ &  \\
		$xend$ & - & $\mathbb{R}$ & - \\
		$Domain$ & - & $\mathbb{R}^{2N}$ &- \\
		$ellipjdn$ & - & $\mathbb{R}^{2N}$ &- \\
		$ellipjcn$ & - & $\mathbb{R}^{2N}$ &- \\
		$ellipjMAT_{cn}$ & - & $\mathbb{R}^{2Nx2N}$ &- \\
		$ellipjMAT_{dn}$ & - & $\mathbb{R}^{2Nx2N}$ &- \\
		h & - & $\mathbb{R}$ & -\\ 
		\hline
	\end{tabular}
\end{center}

\subsection{Semantics}

\subsubsection{State Variables}
None

\subsubsection{Environment Variables}

None

\subsubsection{Assumptions}

\begin{itemize}
	\item Input parameters is called before Numerical parameters. 
	\item Input parameters does not throw an exception. 
\end{itemize}

\subsubsection{Access Routine Semantics}

\noindent $Numpars$:
\begin{itemize}
	\item transition: The data $(N,k)$ will be captured from the input 
	parameters module and used to create a cell structure,NumPars, whose 
	components are xend, Domain,ellipjdn, 
	ellipjdc, $ellipMAT_{dn}$, 
	$elliptMAT_{cn}$ and h. $k$ will be 
	inputted as 
	an argument into the elliptic integral module \ref{MEI}. The resulting 
	integral is 
	equal to $xend$. \\ 
	The domain is created using linspace \ref{MLIN}. The endpoint arguments of 
	linspace 
	are $-xend$ and $xend$, respectively. The distance between partition points 
	in the resulting domain is $h=\frac{xend}{N}$. Ellipjdn (cn) is derived by 
	computing the ellipjdn (cn) value \ref{MELF} of each point in the Domain. 
	$EllipjMAT_{cn}$ (dn) is a 
	diagonal matrix whose diagonal is Ellipjcn (dn). 
	\item exception: None
\end{itemize} 

\subsubsection{Local Functions} 

None. 

\newpage

\section{MIS of Eigenvalue and Eigenvector Solver} \label{SEES}

The secret of this module is the numerical algorithm for calculating the 
eigenvalues and eigenvectors of an $n$ by $n$ matrix.

\subsection{Module}

eig (\url{https://www.mathworks.com/help/matlab/ref/eig.html})

\subsection{Uses}

-

\subsection{Syntax}

\begin{center}
	\begin{tabular}{p{2cm} p{4cm} p{4cm} p{4cm}}
		\hline
		\textbf{Name} & \textbf{In} & \textbf{Out} & \textbf{Exceptions} 
		\ref{Appendix} \\
		\hline
		solver & $A : \mathbb{C}^{nxn}$  & 
		$\mathbb{C}^{n}$ x $\mathbb{R}^{nxn}$ & NotsquareMat \\
		\hline
	\end{tabular}
\end{center}

\subsection{Semantics}

\subsubsection{State Variables}

None.

\subsubsection{Environment Variables}

None.

\subsubsection{Assumptions}

\begin{itemize}
	\item The input is a square matrix. 
\end{itemize}

\subsubsection{Access Routine Semantics}

\noindent eig():
\begin{itemize}
	\item output: out:= $\lambda$ and $\bar{v}$ such that:\\
	$A\bar{v} = \lambda \bar{v}$ \\
	$\bar{v} : \mathbb{C}^{n}$ and $\lambda : \mathbb{C}$
	\item exception: exce:= $ \neg (A : \mathbb{C}^{nxn}) \Rightarrow$ 
	NotSquareMatrix 
\end{itemize}

\subsubsection{Local Functions} 

None. 

\newpage

\section{MIS of Diagonal Matrix} \label{MDM}

The secrets of this module are the numerical algorithm for creating an $n$ by 
$n$ diagonal 
matrix from an $n$ by 1 vector and the numerical algorithm for creating an $n$ 
by 1 vector from an $n$ by $n$ diagonal 
matrix. 

\subsection{Module}

diag (\url{https://www.mathworks.com/help/matlab/ref/diag.html})

\subsection{Uses}

-

\subsection{Syntax}

\begin{center}
	\begin{tabular}{p{2cm} p{6cm} p{6cm} p{3cm}}
		\hline
		\textbf{Name} & \textbf{In} & \textbf{Out} & \textbf{Exceptions} 
		\ref{Appendix} \\
		\hline
		solver & $A :$ Diagonal $n$ by $n$ matrix  & $\mathbb{C}^{n}$ & 
		NotDiagMat \\ 
		solver & $v :\mathbb{C}^{n}$  & Diagonal $n$ by $n$ matrix & 
		NotVector \\
		\hline
	\end{tabular}
\end{center}

\subsection{Semantics}

\subsubsection{State Variables}

None.

\subsubsection{Environment Variables}

None.

\subsubsection{Assumptions}

\begin{itemize}
	\item The input is a square matrix or column vector.  
\end{itemize}

\subsubsection{Access Routine Semantics}

\noindent diag.solver():
\begin{itemize}
	\item output (if In=$A$): out:= v such that:\\
	$A(j,j) = v(j)$ for j=1,2,...,n \\
	\item exception (if In=$A$): exce:= $A \not\in (Diagonal Matrix) 
	\Rightarrow$ 
	NotDiagonalMatrix 
	\item output (if In=$v$): out:= A such that:\\
	$A(j,j) = v(j)$ for j=1,2,...,n and zero else. \\
	\item exception (if In=$v$): exce:= $v \not\in \mathbb{C}^{n} \Rightarrow$ 
	NotVector 
\end{itemize} 

\subsubsection{Local Functions} 

None. 

\newpage 

\section{MIS of Elliptic Integral} \label{MEI}

The secret of this module is the the numerical algorithm for calculating the 
complete 
elliptic integral for some real constant $k$.

\subsection{Module}

ellipK (\url{https://www.mathworks.com/help/symbolic/elliptick.html})

\subsection{Uses}

-

\subsection{Syntax}

\begin{center}
	\begin{tabular}{p{2cm} p{4cm} p{4cm} p{3cm}}
		\hline
		\textbf{Name} & \textbf{In} & \textbf{Out} & \textbf{Exceptions} 
		\ref{Appendix} \\
		\hline
		solver & $k : \mathbb{R}$ & $\mathbb{R}$ & 
		NonNumericalError \\ 
		\hline
	\end{tabular}
\end{center}

\subsection{Semantics}

\subsubsection{State Variables}

None.

\subsubsection{Environment Variables}

None.

\subsubsection{Assumptions}

\begin{itemize}
	\item The user understands the physical context of the number k.  
\end{itemize}

\subsubsection{Access Routine Semantics}

\noindent ellipK.solver():
\begin{itemize}
	\item output : \\
	$$ 
	\int_{0}^{\frac{\pi}{2}} \frac{dx}{\sqrt{1-ksin^{2}(x)}}$$ \\
	\item exception : exce:= $\neg(k : \mathbb{R}) \Rightarrow$ 
	NonNumeric
\end{itemize}

\subsubsection{Local Functions} 

None. 

\newpage

\section{MIS of Elliptic Functions} \label{MELF}

The secret of this module is the the numerical algorithm for calculating the 
values of the elliptic functions.

\subsection{Module}

ellipj (\url{https://www.mathworks.com/help/matlab/ref/ellipj.html})

\subsection{Uses}

- 

\subsection{Syntax}

\begin{center}
	\begin{tabular}{p{2cm} p{4cm} p{4cm} p{3cm}}
		\hline
		\textbf{Name} & \textbf{In} & \textbf{Out} & \textbf{Exceptions} 
		\ref{Appendix}\\
		\hline
		solver & $X : \mathbb{R}^{n}$ & $\mathbb{R}^{3xn}$ & 
		NotVector \\ 
		 & $k : \mathbb{R}$ &  &  \\ 
		\hline
	\end{tabular}
\end{center}

\subsection{Semantics}

\subsubsection{State Variables}

None.

\subsubsection{Environment Variables}

None.

\subsubsection{Assumptions}

None.

\subsubsection{Access Routine Semantics}

\noindent ellipj.solver():
\begin{itemize}
	\item output : $Y: \mathbb{R}^{n}$ such that $Y(1,j)= sn(X(j),k)$, 
	$Y(2,j)=cn(X(j),k)$ and $Y(3,j)=dn(X(j),k)$. See Module link for more 
	detail regarding jacobi ellptic functions sn,dn and cn. 
	\item exception : exce:= $\neg(X : \mathbb{R}^{n}) \Rightarrow$ 
	NotVector
\end{itemize} 

\subsubsection{Local Functions} 

None. 

\newpage 

\section{MIS of Plotting} \label{MPLT} 

The secret of this module is the plotting algorithm. 

\subsection{Module}

plot (\url{https://www.mathworks.com/help/matlab/ref/plot.html})

\subsection{Uses}

-

\subsection{Syntax}

\begin{center}
	\begin{tabular}{p{2cm} p{4cm} p{4cm} p{5cm}}
		\hline
		\textbf{Name} & \textbf{In} & \textbf{Out} & \textbf{Exceptions} 
		\ref{Appendix} \\
		\hline
		plot & $X : \mathbb{R}^{n}$ & graph(Y) & 
		BadVal, DimensionErr \\ 
		& $Y : \mathbb{R}^{n}$ &  &  \\ 
		\hline
	\end{tabular}
\end{center}

\subsection{Semantics}

\subsubsection{State Variables}

None.

\subsubsection{Environment Variables}

None.

\subsubsection{Assumptions}

None.

\subsubsection{Access Routine Semantics}

\noindent plot.plot():
\begin{itemize}
	\item Transition: Plot accepts two vectors as input. A figure is created 
	with the following properties. The domain ranges from the minimum of X to 
	the maximum of X. The range ranges from the minimum of Y to the maximum of 
	Y. The pairs $(X(j),Y(j))$ for $j=1,2,...,n$ are plotted on a figure with 
	respect to the aforementioned axes. 
	\item exception : exce1:= $\neg(X,Y \in \mathbb{C}^{n}) \Rightarrow$ 
	BadVal \\
	exce2:= $\neg (dim(Y) = dim(X)) \Rightarrow$ DimensionErr \\
\end{itemize}

\subsubsection{Local Functions} 

None. 

\newpage 

\section{MIS of Linspace} \label{MLIN}

The secret of this module is the software algorithm for creating a vector with 
equally spaced entries. 

\subsection{Module}

Linspace (\url{https://www.mathworks.com/help/matlab/ref/linspace.html})

\subsection{Uses}

-

\subsection{Syntax}

\begin{center}
	\begin{tabular}{p{2cm} p{6cm} p{6cm} p{3cm}}
		\hline
		\textbf{Name} & \textbf{In} & \textbf{Out} & \textbf{Exceptions} 
		\ref{Appendix}\\
		\hline
		create & $a : \mathbb{R}$ & $\mathbb{R}^{c}$ & 
		NonNumerical \\
		& $b : \mathbb{R}$ &  & notNat \\  
		& $c : \mathbb{N}$ &  &  \\ 
		\hline
	\end{tabular}
\end{center}

\subsection{Semantics}

\subsubsection{State Variables}

None.

\subsubsection{Environment Variables}

None.

\subsubsection{Assumptions}

None.

\subsubsection{Access Routine Semantics}

\noindent linspace.create():
\begin{itemize}
	\item output : $X : \mathbb{R}^{c}$ such that X(1)=a, X(n)=b and 
	$|x(k)-x(k-1)|=\frac{b-a}{n-1}$ for $k \in {2,3,4,..,n}$. 
	\item exception : exce:= $\neg(c \in \mathbb{N}) \Rightarrow$ 
	notNat
	\item exception : exce:= $\neg(a: \mathbb{R}) \Rightarrow$ 
	NonNumericalError 
	\item exception : exce:= $\neg(b: \mathbb{R}) \Rightarrow$ NonNumericalError
\end{itemize}

\subsubsection{Local Functions} 

None. 

\newpage 

\section{MIS of Toeplitz} \label{TOE}

The secret of this module is the software algorithm for creating a toeplitz 
matrix.

\subsection{Module}

Toeplitz (\url{https://www.mathworks.com/help/matlab/ref/toeplitz.html})

\subsection{Uses}

- 

\subsection{Syntax}

\begin{center}
	\begin{tabular}{p{2cm} p{6cm} p{6cm} p{3cm}}
		\hline
		\textbf{Name} & \textbf{In} & \textbf{Out} & \textbf{Exceptions} 
		\ref{Appendix}\\
		\hline
		create & $c : \mathbb{R}^{m}$ & $\mathbb{R}^{mxm}$ & 
		NonNumerical \\
		& $r : \mathbb{R}^{m}$ &  & notVec \\ 
		\hline
	\end{tabular}
\end{center}

\subsection{Semantics}

\subsubsection{State Variables}

None.

\subsubsection{Environment Variables}

None.

\subsubsection{Assumptions}

None.

\subsubsection{Access Routine Semantics}

\noindent toeplitz: Returns a nonsymmetric Toeplitz matrix with c as its first 
column and r as its first row.

\subsubsection{Local Functions} 

None. 

\newpage 


\section{MIS of Control} \label{MCON}

The secret of this module is the algorithm that coordinates the overall program 
and 
interaction between modules.

\subsection{Module}

Main

\subsection{Uses}

This module uses Input Parameters \ref{IPM}, Numerical Parameters\ref{MNP}, 
Spectrum Matrix \ref{SSM}, 
Eigenvalue solver \ref{SEES},  Output \ref{OFM}, plotting \ref{MPLT} and Exact 
Eigenvalue equations \ref{SEE}.

\subsection{Syntax}

\begin{center}
	\begin{tabular}{p{2cm} p{6cm} p{6cm} p{3cm}}
		\hline
		\textbf{Name} & \textbf{In} & \textbf{Out} & \textbf{Exceptions} \\
		\hline
		main & - & - & - \\ 
		\hline
	\end{tabular}
\end{center}

\subsection{Semantics}

\subsubsection{State Variables}

None.

\subsubsection{Environment Variables}

None.

\subsubsection{Assumptions}

None.

\subsubsection{Access Routine Semantics}

\noindent main():
\begin{itemize}
	\item transition: This function controls the running of the scientific 
	software. First, input is taken from the user. 
	The input \ref{IPM} is brought to the numerical parameters module \ref{MNP} 
	where useful 
	constants, matrices and elliptic function values are calculated. The state 
	variables from Numerical parameters are used as arguments in Spectrum 
	Matrix. The spectrums (eigenvalues) can be calculated \ref{SEES} once the 
	six spectral 
	matrices \ref{SSM} are created. These eigenvalues are plotted \ref{MPLT}. 
\end{itemize}

\subsubsection{Local Funtions} 

None. 

\newpage 


\bibliographystyle {plainnat}
\bibliography{../../../ReferenceMaterial/References} 

\newpage

\section{Appendix} \label{Appendix}

\renewcommand{\arraystretch}{1.2}
\begin{tabular}{l l} 
	\toprule		
	\textbf{exception name} & \textbf{description}\\
	\midrule 
	NonNumericalError & The input is not a number.\\ 
	BadkRange & The value of k is out of its necessary range.\\ 
	BadNRange & The value of N is out of its necessary range.\\ 
	BadPRange & The value of P is out of its necessary range.\\ 
	NotsquareMat & The input matrix is not square (or n by n for n: 
	$\mathbb{N}$)\\ 
	NotDiagMat & The input matrix is not diagonal \\ 
	NotVector & The input was not a vector. \\ 
	BadVal & At least one of the vector elements is not a number. \\ 
	NotNat & The input was not a natural number. \\
	NotVec & The input was not a vector.\\
	\bottomrule
\end{tabular}\\


\end{document}
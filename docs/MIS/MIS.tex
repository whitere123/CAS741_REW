\documentclass[12pt, titlepage]{article}

\usepackage{amsmath, mathtools}

\usepackage[round]{natbib}
\usepackage{amsfonts}
\usepackage{amssymb}
\usepackage{graphicx}
\usepackage{colortbl}
\usepackage{xr}
\usepackage{hyperref}
\usepackage{longtable}
\usepackage{xfrac}
\usepackage{tabularx}
\usepackage{float}
\usepackage{siunitx}
\usepackage{booktabs}
\usepackage{multirow}
\usepackage[section]{placeins}
\usepackage{caption}
\usepackage{fullpage}

\hypersetup{
bookmarks=true,     % show bookmarks bar?
colorlinks=true,       % false: boxed links; true: colored links
linkcolor=red,          % color of internal links (change box color with linkbordercolor)
citecolor=blue,      % color of links to bibliography
filecolor=magenta,  % color of file links
urlcolor=cyan          % color of external links
}

\usepackage{array}

\externaldocument{../../SRS/SRS}

%% Comments

\usepackage{color}

\newif\ifcomments\commentstrue

\ifcomments
\newcommand{\authornote}[3]{\textcolor{#1}{[#3 ---#2]}}
\newcommand{\todo}[1]{\textcolor{red}{[TODO: #1]}}
\else
\newcommand{\authornote}[3]{}
\newcommand{\todo}[1]{}
\fi

\newcommand{\wss}[1]{\authornote{blue}{SS}{#1}}
\newcommand{\an}[1]{\authornote{magenta}{Author}{#1}}



\newcommand{\progname}{Program Name}

\begin{document}

\title{Module Interface Specification for SpecSearch}

\author{Robert E. White}

\date{\today}

\maketitle

\pagenumbering{roman}

\section{Revision History}

\begin{tabularx}{\textwidth}{p{3cm}p{2cm}X}
\toprule {\bf Date} & {\bf Version} & {\bf Notes}\\
\midrule
2018-11-09 & 1.0 & Creation of first draft for presentation. \\
\bottomrule
\end{tabularx}

~\newpage

\section{Symbols, Abbreviations and Acronyms}

See SRS Documentation at \wss{give url}

\wss{Also add any additional symbols, abbreviations or acronyms}

\newpage

\tableofcontents

\newpage

\pagenumbering{arabic}

\section{Introduction}

The following document details the Module Interface Specifications for
SpecSearch. SpecSearch is scientific computing software that computes and plots 
the spectrum of a matrix operator that appears in a LAX pair that is compatible 
for solutions of the Non-Linear Schr\dots{o}dinger (NLS) equation. 
Consequently, this spectrum carries useful information regarding the stability 
of solutions to the NLS equation. Physicists are interested in this spectrum as 
the NLS equation models many physical phenomena, such as rogue waves or 
modulated wave packets. Mathematicans are interested in the analytical 
behaviour of this spectrum. 

Complementary documents include the System Requirement Specifications
and Module Guide.  The full documentation and implementation can be
found at \url{...}.  \wss{provide the url for your repo}

The purpose of this document is to describe the intended behaviour of the 
access routines for SpecSearch's modules. In particular, mathematical notation 
will be used to describe the input/output relationships and external behaviour 
of the 
modules. The MIS is still abstract since it does not cover implementation of 
the modules and does not outline any code.  
\section{Notation}

The structure of the MIS for modules comes from \citet{HoffmanAndStrooper1995},
with the addition that template modules have been adapted from
\cite{GhezziEtAl2003}.  The mathematical notation comes from Chapter 3 of
\citet{HoffmanAndStrooper1995}.  For instance, the symbol := is used for a
multiple assignment statement and conditional rules follow the form $(c_1
\Rightarrow r_1 | c_2 \Rightarrow r_2 | ... | c_n \Rightarrow r_n )$.

The following table summarizes the primitive data types used by \progname. 

\begin{center}
\renewcommand{\arraystretch}{1.2}
\noindent 
\begin{tabular}{l l p{7.5cm}} 
\toprule 
\textbf{Data Type} & \textbf{Notation} & \textbf{Description}\\ 
\midrule
character & char & a single symbol or digit\\
integer & $\mathbb{Z}$ & a number without a fractional component in (-$\infty$, $\infty$) \\
natural number & $\mathbb{N}$ & a number without a fractional component in [1, $\infty$) \\
real & $\mathbb{R}$ & any number in (-$\infty$, $\infty$)\\
complex & $\mathbb{C}$ &any number $x + iy$ with $x \in \mathbb{R}$ , $y \in 
\mathbb{R}$ and $i^{2} = -1$. \\
\bottomrule
\end{tabular} 
\end{center}

\noindent
The specification of \progname \ uses some derived data types: sequences, strings, and
tuples. Sequences are lists filled with elements of the same data type. Strings
are sequences of characters. Tuples contain a list of values, potentially of
different types. In addition, \progname \ uses functions, which
are defined by the data types of their inputs and outputs. Local functions are
described by giving their type signature followed by their specification.

\section{Module Decomposition}

The following table is taken directly from the Module Guide document for this project.

\begin{table}[h!]
\centering
\begin{tabular}{p{0.3\textwidth} p{0.6\textwidth}}
\toprule
\textbf{Level 1} & \textbf{Level 2}\\
\midrule

{Hardware-Hiding} & ~ \\
\midrule

\multirow{7}{0.3\textwidth}{Behaviour-Hiding} 
& Input Parameters \\
& Output Format \\
& Spectrum Matrix \\
& Exact Eigenvalue Equations \\
& Spectrum Error Equation \\
& Numerical Parameters \\  
& Control \\ 
\midrule

\multirow{3}{0.3\textwidth}{Software Decision} & {Sequence Data Structure}\\
& Eigenvalue and Eigenvector Solver \\
& Diagonal Matrix Generator\\
& Elliptic Integral\\ 
& Elliptic Functions\\ 
& Plotting \\ 
& Linspace \\
\bottomrule

\end{tabular}
\caption{Module Hierarchy}
\label{TblMH}
\end{table}

\newpage

\section{MIS of Input Parameters} \label{IPM} 

The secrets of this module are the data structure for input parameters and 
methods for verifying input. 

\subsection{Module}

InParams

\subsection{Uses}

-

\subsection{Syntax}

\begin{center}
\begin{tabular}{p{3cm} p{3cm} p{3cm} >{\raggedright\arraybackslash}p{7cm}}
\toprule
\textbf{Name} & \textbf{In} & \textbf{Out} & \textbf{Exceptions} \\
\hline
Call\_params & $(k,N) \in \mathbb{R}$ x $\mathbb{N}$ & - & NonNumericalError \\
Verify\_params & - & - & BadkRange, BadNRange \\
k & - & $\mathbb{R}$ & - \\
N & - & $\mathbb{N}$ & - \\
\hline
\end{tabular}
\end{center}

\subsection{Semantics}

\subsubsection{State Variables}

$k \in \mathbb{R}$ \\ 
$N \in \mathbb{N}$ 

\subsubsection{Environment Variables}

InputParameters: The two values entered by the user. 

\subsubsection{Assumptions}

\begin{itemize}
	\item Call\_params is called before the values of any state variables will 
	be accessed.
	\item The user inputs the state variables in the following order: $(k,N)$
\end{itemize}

\subsubsection{Access Routine Semantics}

\noindent InParams.k():
\begin{itemize}
\item output: $out = k$
\item exception: None
\end{itemize}

\noindent InParams.N():
\begin{itemize}
	\item output: $out = N$
	\item exception: None
\end{itemize}

\noindent Call\_Params():
\begin{itemize}
	\item transition: The data is read sequentially from the command line. The 
	data are seperated by the return key. This data is used to populate the 
	previously mentioned state variables. 
	\item exception: If non-numerical data is entered. 
\end{itemize} 

\noindent Verify\_Params():
\begin{itemize}
	\item exception: exc:=
\end{itemize}  

\noindent $\neg (k \in (0,1)) \Rightarrow \ $ BadkRange\\
$\neg (N \in \mathbb{N}) \Rightarrow \ $ BadNRange

\subsubsection{Local Functions}
\wrw{cells and data structures in MATLAB??}
\newpage

\section{MIS of Output Format} 

The secret of this module is the format and structure of the output data.

\subsection{Module}

OutForm

\subsection{Uses}

Eigenvalue solver, Spectrum Error Equations and Numerical Parameters. 

\begin{center}
	\begin{tabular}{p{3cm} p{3cm} p{3cm} >{\raggedright\arraybackslash}p{7cm}}
		\toprule
		\textbf{Name} & \textbf{In} & \textbf{Out} & \textbf{Exceptions} \\
		\hline
		capture & $(D,Er,V,Vl) \in
		\mathbb{R}^{2n}$x$\mathbb{R}$x$M_{4n(4n+1)}(\mathbb{C})$
		x$\mathbb{C}^{4n}$ & $Spectrum$ & - \\
		Spectral & - & $M_{4n(4n+2)}(\mathbb{C})$ & - \\
		D & - & $\mathbb{R}^{2n}$ & - \\
		Er & - & $\mathbb{R}$ & - \\
		\hline
	\end{tabular}
\end{center}

\subsection{Semantics}

\subsubsection{State Variables}

Spectral $\in M_{4n(4n+1)}(\mathbb{C})$ \\
$D \in \mathbb{R}^{2n}$ \\
$Er \in \mathbb{R}^{2}$ \\ 

\subsubsection{Assumptions}

\begin{itemize}
	\item Eigenvalue solver, Spectrum Error Equations and Numerical Parameters 
	are called and their values are temporarily stored prior to OutForm.
\end{itemize}

\subsubsection{Access Routine Semantics}

\noindent OutForm.D():
\begin{itemize}
	\item output: $out = D$
	\item exception: None
\end{itemize}

\noindent OutForm.Er():
\begin{itemize}
	\item output: $out = Er$
	\item exception: None
\end{itemize} 

\noindent OutForm.Spectral():
\begin{itemize}
	\item output: $out = Spectral$
	\item exception: None
\end{itemize}

\noindent capture():
\begin{itemize}
	\item Transition: The data (D,Er,V,Vl) will be captured from the other 
	modules and then stored into Spectral for convienience. Spectral is a 
	$4n$ by $4n + 2$ matrix created as follows: the $4n$ by $1$ vector of 
	eigenvalues, Vl, will be transposed and turned into the $(4n+1)^{th}$ row 
	of 
	spectral. The previous $4n$ rows of spectral will be the matrix, V, 
	composed 
	of the eigenvectors as columns. D will be the $(4n+2)$ row of spectral. 
	\item Exception: None
\end{itemize}  

\subsubsection{Local Functions}

\wrw{what are local functions??}

\newpage
\section{MIS of Spectrum Matrix} 

The structure of the spectrum matrix, its data entries, how 
it is created, and the numerical method for approximating its 
eigenfunctions are the secrets of this module.  

\subsection{Module}

SpecMat

\subsection{Uses}

This module uses Numerical Parameters. 

\subsection{Syntax}

\begin{center}
	\begin{tabular}{p{3cm} p{3cm} p{3cm} >{\raggedright\arraybackslash}p{7cm}}
		\toprule
		\textbf{Name} & \textbf{In} & \textbf{Out} & \textbf{Exceptions} \\
		\hline
		create & $(n,h,elipdn,k) \in
		\mathbb{N}$x$\mathbb{R}$x$\mathbb{R}^{2n}$x$\mathbb{R}$ & 
		SpecMatO1,
		SpecMatO4,
		SpecCheb & - \\
		SpecMatO1 & - & $M_{4n(4n)}(\mathbb{C})$ & - \\
		SpecMatO4 & - & $M_{4n(4n)}(\mathbb{C})$ & - \\ 
		SpecCheb & - & $M_{4n(4n)}(\mathbb{C})$ & - \\
		\hline
	\end{tabular}
\end{center}

\subsection{Semantics}

\subsubsection{State Variables}

SpecMatO1, SpecMatO4, SpecCheb $\in M_{4n(4n)}(\mathbb{C})$ 

\subsubsection{Assumptions}

\begin{itemize}
	\item All of the numerical parameters are calculated correctly and used as 
	arguments in create. 
\end{itemize}

\subsubsection{Access Routine Semantics}

SpecMat.SpecMatO1:
\begin{itemize}
	\item output: out = SpecMatO1 
	\item exception: None 
\end{itemize}

SpecMat.SpecMatO4:
\begin{itemize}
	\item output: out = SpecMatO4
	\item exception: None
\end{itemize}

SpecMat.SpecFour:
\begin{itemize}
	\item output: out = SpecCheb 
	\item exception: None
\end{itemize}

SpecMat.create():
\begin{itemize}
	\item transition: The data (n,h,elipdn,k) will be captured from the other 
	modules and used to create the spectral matrices. There will be three 
	instances of the spectral matrix. Each corresponds do a different numerical 
	algorithm for approximating the eigenfunction derivatives. Top top left and 
	bottom right sections of the matrices are diagonal matrices with elements 
	equal to the elipdn function evaluated at points in the discretized domain. 
	The remain parts of the matrix are coffecients such that when multiplied by 
	the eigenfunctions will create the appropriate approximation of the 
	derivatives. 
	\item exception: None
\end{itemize}

\newpage

\section{MIS of Exact Eigenvalue Equations} 

The secrets of this module are the analytical expressions for the two real 
eigenvalues. 

\subsection{Module}

TheorEigenValues

\subsection{Uses}

This module uses input parameters. 

\subsection{Syntax}

\begin{center}
	\begin{tabular}{p{3cm} p{3cm} p{3cm} >{\raggedright\arraybackslash}p{7cm}}
		\toprule
		\textbf{Name} & \textbf{In} & \textbf{Out} & \textbf{Exceptions} \\
		\hline
		$lambda_O1$ & $k \in \mathbb{R}$ & $\mathbb{R}$ & - \\
		$lambda_O2$ & $k \in \mathbb{R}$ & $\mathbb{R}$ & - \\ 
		\hline
	\end{tabular}
\end{center}

\subsection{Semantics}

\subsubsection{State Variables}

None

\subsubsection{Environment Variables}

None

\subsubsection{Assumptions}

\begin{itemize}
	\item These are the eigenvalues computed from segal et al. Refer to (segal) 
	for more details regarding assumptions.
\end{itemize}

\subsubsection{Access Routine Semantics}


$lambda_O1$:
\begin{itemize}
	\item output: $\frac{1}{2} (1+\sqrt{1-k^{2}})$ 
	\item exception: None
\end{itemize}


\noindent $lambda_O2$
\begin{itemize} 
	\item output: $\frac{1}{2} (1-\sqrt{1-k^{2}})$ 
	\item exception: None
\end{itemize}


\newpage

\section{MIS of Spectrum Error Equation} 

The secret of this module is the equation for error between exact and 
approximated eigenvalues.

\subsection{Module}

ErrCalc

\subsection{Uses}

This module uses Exact Eigenvalue equations and Eigenvalue solver. 

\subsection{Syntax}

\begin{center}
	\begin{tabular}{p{3cm} p{6cm} p{3cm} >{\raggedright\arraybackslash}p{3cm}}
		\toprule
		\textbf{Name} & \textbf{In} & \textbf{Out} & \textbf{Exceptions} \\
		\hline
		$lambda_O1$ & - & $\mathbb{R}$ & - \\
		$lambda_O2$ & - & $\mathbb{R}$ & - \\ 
		$lambda_C1$ & - & $\mathbb{R}$ & - \\
		$lambda_C2$ & - & $\mathbb{R}$ & - \\
		$Err1$ & $(lambda_O1,lambda_C1)\in \mathbb{R}^{2}$ & $\mathbb{R}$ &- \\
		$Err2$ & $(lambda_O2,lambda_C2)\in \mathbb{R}^{2}$ & $\mathbb{R}$ &- \\
		\hline
	\end{tabular}
\end{center}

\subsection{Semantics}

\subsubsection{State Variables}

$(lambda_O1,lambda_O2,lambda_C1,lambda_C2) \in \mathbb{R}^{4}$

\subsubsection{Environment Variables}

None

\subsubsection{Assumptions}

\begin{itemize}
	\item These are the eigenvalues computed from segal et al. Refer to (segal) 
	for more details regarding assumptions.
\end{itemize}

\subsubsection{Access Routine Semantics}


$Err1$:
\begin{itemize}
	\item output: $| lambda_O1 - lambda_C1| $ 
	\item exception: None
\end{itemize}


\noindent $Err2$
\begin{itemize} 
	\item output: $| lambda_O2 - lambda_C2| $ 
	\item exception: None
\end{itemize} 

\noindent $ErrCalc.lambda_O1$:
\begin{itemize}
	\item output: out = $lambda_O1$
	\item exception: None 
\end{itemize}

\noindent $ErrCalc.lambda_O2$:
\begin{itemize}
	\item output: out = $lambda_O2$
	\item exception: None
\end{itemize}

\noindent $ErrCalc.lambda_C1$:
\begin{itemize}
	\item output: out = $lambda_C1$
	\item exception: None
\end{itemize}

\noindent $ErrCalc.lambda_C2$:
\begin{itemize}
	\item output: out = $lambda_C2$ 
	\item exception: None
\end{itemize}

\newpage 

\section{MIS of Numerical Parameters} 

The secrets of this module are the range of the eigenfunction domain, points in 
the 
periodic domain and equation for the numerical scaling factor that computes 
the 
eigenfunction derivatives. 

\subsection{Module}

Numpars

\subsection{Uses}

This module uses Elliptic Functions, Diagonal Matrix, linspace, Elliptic 
Integral and input parameters. 

\subsection{Syntax}

\begin{center}
	\begin{tabular}{p{3cm} p{3cm} p{7cm} >{\raggedright\arraybackslash}p{3cm}}
		\toprule
		\textbf{Name} & \textbf{In} & \textbf{Out} & \textbf{Exceptions} \\
		\hline
		CreateVars & $(k,N) \in \mathbb{R}$x$\mathbb{N}$ & 
		$(xend,Domain,ellipj,ellipjMAT)$ & - \\
		xend & - & $\mathbb{R}$ & - \\
		Domain & - & $\mathbb{R}^{2N}$ &- \\
		ellipjdn & - & $\mathbb{R}^{2N}$ &- \\
		ellipjMAT & - & $M_{2n(2n)}(\mathbb{R})$ &- \\
		h & - & $\mathbb{R}$ - \\ 
		\hline
	\end{tabular}
\end{center}

\subsection{Semantics}

\subsubsection{State Variables}

$(xend,Domain,ellipjdn,ellipjMAT) \in 
\mathbb{R}$x$\mathbb{R}^{2N}$x$\mathbb{R}^{2N}$x$M_{2n(2n)}(\mathbb{R})$ 

\subsubsection{Environment Variables}

None

\subsubsection{Assumptions}

\begin{itemize}
	\item Input parameters is called before Numerical parameters. 
	\item Input parameters does not throw an exception. 
\end{itemize}

\subsubsection{Access Routine Semantics}


$Numpars.xend$:
\begin{itemize}
	\item output: $xend$ 
	\item exception: None
\end{itemize}


\noindent $Numpars.Domain$
\begin{itemize} 
	\item output: $Domain$ 
	\item exception: None
\end{itemize} 

\noindent $Numpars.ellipjdn$:
\begin{itemize}
	\item output: out = $ellipjdn$
	\item exception: None 
\end{itemize}

\noindent $Numpars.ellipjMAT$:
\begin{itemize}
	\item output: out = $ellipjMAT$
	\item exception: None
\end{itemize}

\noindent $Numpars.h$:
\begin{itemize}
	\item output: out = $h$ 
	\item exception: None
\end{itemize} 

\noindent $Numpars.CreateVars$:
\begin{itemize}
	\item transition: The data $(N,k)$ will be captured from the input 
	parameters and used to create the state variables. $k$ will be inputted as 
	an argument into the elliptic integral module. The resulting integral is 
	equal to $xend$. \\ 
	The domain is created using linspace. The endpoint arguments of linspace 
	are $-xend$ and $xend$, respectively. The distance between partition points 
	in the resulting domain is $h=\frac{xend}{N}$. Ellipjdn is derived from 
	from computing the ellipjdn value of each point in Domain. EllipjMAT is a 
	diagonal matrix whose diagonal is Ellipjdn. 
	\item exception: None
\end{itemize}

\newpage

\section{MIS of Eigenvalue and Eigenvector Solver} 

The secret of this module is the numerical algorithm for calculating the 
eigenvalues and eigenvectors of an $n$ by $n$ matrix.

\subsection{Module}

eig

\subsection{Uses}

This module uses Spectrum Matrix. 

\subsection{Syntax}

\begin{center}
	\begin{tabular}{p{2cm} p{4cm} p{4cm} p{2cm}}
		\hline
		\textbf{Name} & \textbf{In} & \textbf{Out} & \textbf{Exceptions} \\
		\hline
		solver & $A\in M_{(nxn)}(\mathbb{R})$  & 
		$\mathbb{C}^{n}$x$M_{(nxn)}(\mathbb{R})$ & NotsquareMat \\
		\hline
	\end{tabular}
\end{center}

\subsection{Semantics}

\subsubsection{State Variables}

None.

\subsubsection{Environment Variables}

None.

\subsubsection{Assumptions}

\begin{itemize}
	\item The input is a square matrix. 
\end{itemize}

\subsubsection{Access Routine Semantics}

\noindent eig():
\begin{itemize}
	\item output: out:= $(\lambda, \bar{v})$ such that:\\
	$A\bar{v} = \lambda \bar{v}$ 
	\item exception: exce:= $A \not\in M_{(nxn)}(\mathbb{R}) \Rightarrow 
	NotSquareMatrix$ 
\end{itemize}

\newpage


\bibliographystyle {plainnat}
\bibliography {../../../ReferenceMaterial/References}

\newpage

\section{Appendix} \label{Appendix}

\wss{Extra information if required}

\end{document}
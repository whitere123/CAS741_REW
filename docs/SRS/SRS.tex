\documentclass[12pt]{article}

\usepackage{amsmath, mathtools}
\usepackage{amsfonts}
\usepackage{amssymb}
\usepackage{graphicx}
\usepackage{colortbl}
\usepackage{xr}
\usepackage{hyperref}
\usepackage{longtable}
\usepackage{xfrac}
\usepackage{tabularx}
\usepackage{float}
\usepackage{siunitx}
\usepackage{booktabs}
\usepackage{caption}
\usepackage{pdflscape}
\usepackage{afterpage}
\usepackage{enumitem}
\usepackage[round]{natbib}

%\usepackage{refcheck}

\hypersetup{
   % bookmarks=true,         % show bookmarks bar?
      colorlinks=true,       % false: boxed links; true: colored links
    linkcolor=red,          % color of internal links (change box color with linkbordercolor)
    citecolor=green,        % color of links to bibliography
    filecolor=magenta,      % color of file links
    urlcolor=cyan           % color of external links
}

\input{../Comments}
%% Common Parts

\newcommand{\progname}{SpecSearch} % PUT YOUR PROGRAM NAME HERE %Every program
                                % should have a name
 
% For easy change of table widths
\newcommand{\colZwidth}{1.0\textwidth}
\newcommand{\colAwidth}{0.13\textwidth}
\newcommand{\colBwidth}{0.82\textwidth}
\newcommand{\colCwidth}{0.1\textwidth}
\newcommand{\colDwidth}{0.05\textwidth}
\newcommand{\colEwidth}{0.8\textwidth}
\newcommand{\colFwidth}{0.17\textwidth}
\newcommand{\colGwidth}{0.5\textwidth}
\newcommand{\colHwidth}{0.28\textwidth}

% Used so that cross-references have a meaningful prefix
\newcounter{defnum} %Definition Number
\newcommand{\dthedefnum}{GD\thedefnum}
\newcommand{\dref}[1]{GD\ref{#1}}
\newcounter{datadefnum} %Datadefinition Number
\newcommand{\ddthedatadefnum}{DD\thedatadefnum}
\newcommand{\ddref}[1]{DD\ref{#1}}
\newcounter{theorynum} %Theory Number
\newcommand{\tthetheorynum}{T\thetheorynum}
\newcommand{\tref}[1]{T\ref{#1}}
\newcounter{tablenum} %Table Number
\newcommand{\tbthetablenum}{T\thetablenum}
\newcommand{\tbref}[1]{TB\ref{#1}}
\newcounter{assumpnum} %Assumption Number
\newcommand{\atheassumpnum}{P\theassumpnum}
\newcommand{\aref}[1]{A\ref{#1}}
\newcounter{goalnum} %Goal Number
\newcommand{\gthegoalnum}{P\thegoalnum}
\newcommand{\gsref}[1]{GS\ref{#1}}
\newcounter{instnum} %Instance Number
\newcommand{\itheinstnum}{IM\theinstnum}
\newcommand{\iref}[1]{IM\ref{#1}}
\newcounter{reqnum} %Requirement Number
\newcommand{\rthereqnum}{P\thereqnum}
\newcommand{\rref}[1]{R\ref{#1}}
\newcounter{lcnum} %Likely change number
\newcommand{\lthelcnum}{LC\thelcnum}
\newcommand{\lcref}[1]{LC\ref{#1}}

%\newcommand{\progname}{ProgName} % PUT YOUR PROGRAM NAME HERE

\usepackage{fullpage}

\begin{document}

\title{CAS 741: SRS \\ \progname : A Numerical Search For The Spectrum Related 
To 
Travelling 
Periodic Waves} 
\author{Robert White}
\date{\today}
	
\maketitle

~\newpage

\pagenumbering{roman}

\section{Revision History}

\begin{tabularx}{\textwidth}{p{3cm}p{2cm}X}
\toprule {\bf Date} & {\bf Version} & {\bf Notes}\\
\midrule
25-09-2018 & 1.0 & Creation of the first draft\\
29-09-2018 & 1.1 & Edit of 1.0. Added a GS, responsibilites, requirenments, 
definitions and verification of compatibility. Updated comments.tex. My 
comments are in red, SS comments are in blue.\\ 
01-10-2018 & 1.2 & Edit of 1.1. Added more theory to support the instance 
model. Made a clearer distinction between user and reader charaterisitcs. 
Modified the goal statements.  \\ 
03-10-2018 & 1.3 & Edit of 1.2. Removed a redundant goal statement. Made some 
of Dr. 
Smith's suggested changes. Updated the traceability matrices.  \\
04-10-2018 & 1.4 & Edit of 1.3. Created IM2 and inserted tracebility graphs. 
Completed all of Dr. Smith's suggested changes. Performed a grammar and 
spelling check.\\ 
11-10-2018 & 1.5& Edit of 1.4. Draft 1.4 was peer reviewed by Hanane Zlitni. 
Most of her recommended changes were made. \\
28-11-2018& 1.6& Edit of 1.5. Draft 1.5 was edited for the final 
documentation.\\
\bottomrule
\end{tabularx}

~\newpage

\section{Reference Material}

This section records information for easy reference.

\subsection{Table of Units}

Throughout this document SI (Syst\`{e}me International d'Unit\'{e}s) is employed
as the unit system.  In addition to the basic units, several derived units are
used as described below.  For each unit, the symbol is given followed by a
description of the unit and the SI name. \\

~\newline

\renewcommand{\arraystretch}{1.2}
%\begin{table}[ht]
  \noindent \begin{tabular}{l l l} 
    \toprule		
    \textbf{symbol} & \textbf{unit} & \textbf{SI}\\
    \midrule 
    \si{\metre} & length & metre\\
    t & time & second\\
    \bottomrule
  \end{tabular}
  %	\caption{Provide a caption}
%\end{table}


\subsection{Table of Symbols}

The table that follows summarizes the symbols used in this document along with
their units. Some of the symbols are unitless, 
they are marked with a dash. The choice of symbols was made to be consistent 
with the rogue 
wave and mathematical physics literature.  

\renewcommand{\arraystretch}{1.2}
%\noindent \begin{tabularx}{1.0\textwidth}{l l X}
\noindent \begin{longtable*}{l l p{12cm}} \toprule
\textbf{symbol} & \textbf{unit} & \textbf{description}\\
\midrule 
$\lambda$ & $\frac{1}{s}$ & spectral parameter
\\
$u$ & m & wave amplitude 
\\  
$\phi$ & - & eigen function
\\ 
$c$ & $\frac{m}{s}$ & wave speed 
\\
$\omega$ & $\frac{1}{s}$ & angular frequency 
\\ 
$e$ & - & euler's Number \\ 
$i$ & - & imaginary number\\ 
$(g,d)$ & - & conserved quantities\\
  
\bottomrule
\end{longtable*}

\subsection{Abbreviations and Acronyms}

\renewcommand{\arraystretch}{1.2}
\begin{tabular}{l l} 
  \toprule		
  \textbf{symbol} & \textbf{description}\\
  \midrule 
  A & Assumption\\
  $\mathbb{C}$ & Complex numbers\\
  cn & Elliptic cosine \\
  DD & Data Definition\\
  dn & Delta amplitude \\ 
  GD & General Definition\\
  GS & Goal Statement\\
  IM & Instance Model\\
  LC & Likely Change\\
  NLS & Nonlinear Schrodinger\\
  ODE & Ordinary Differential Equation\\
  PDE & Partial Differential Equation\\
  PS & Physical System Description\\
  $\mathbb{R}$ & Real number line\\
  R & Requirement\\ 
  sn & Elliptic sine\\
  SRS & Software Requirements Specification\\
  progname & \progname \\
  T & Theoretical Model\\  
  \bottomrule
\end{tabular}\\

\subsection{Mathematical Notation}

Let $\delta$ be a complex number of the form $\delta= m + ni$ ($m,n \in 
\mathbb{R}$). The real part of $u$ is $m$ and the imaginary part of $u$ is $n$. 
$i$ is 
the imaginary number such that $i=\sqrt{-1}$.\\

The complex 
conjugate of $\delta$ is defined to be $\bar{\delta}= m - ni$. 
The modulus of $u$ is defined to be $|\delta| = \sqrt{m^{2}+n^{2}}$.\\

Let $z(x,t): \mathbb{R}^{2} \rightarrow \mathbb{C}$. We will let $z_{t}$ and 
$z_{x}$ denote 
the partial derivatives of $z$ with respect to $t$ and $x$ respectively. If $z$ 
has 
one independent 
variable we will denote its derivative by $z'$. \\ 

$ z_{t} = \lim_{h \to 0 } \frac{z(x,t+ t_{0}h) - z(x,t)}{h}$ \\ 

Where $z_{t}$ is differentiable if the above limit is the same regardless of 
the complex number (direction) $t_{0}$. \\

If H is a matrix then $H_{t}$ 
is the 
matrix formed by taking the $t$ derivative in each component of $H$.\\ 

$ e^{i \omega t} = cos(\omega t) + i sin(\omega t)$. So it is clear from the 
above definitions that \\

$ |e^{i \omega t}| = \sqrt{sin^{2} (\omega t) + cos^{2} (\omega t)} =1$ \\  

To see this consider the angle $\omega t$ for fixed $t$. Let the opposite side 
be $a$, denote 
the adjacent side by $b$ and the hypotenuse by $c$. This forms a right angle 
triangle (see picture below). By definition,  $sin(\omega t) = \frac{a}{c} $ 
and $ cos(\omega t) = \frac{b}{c} $. Now, by pythagoras we have that\\

$ \sqrt{sin^{2} (\omega t) + cos^{2} (\omega t)} = 
\sqrt{\frac{a^{2}+b^{2}}{c^{2}} } = \sqrt{ \frac{c^{2}}{c^{2} } }= 1 $ \\ 

For each $t$. So $e^{i \omega t}$ stays on the complex unit circle. \\

\begin{figure}[h!]
	\begin{center}
		{
			\includegraphics[width=\textwidth]{pytha.png}
		} 
	\caption{Trigonometric Ratios}
	\end{center}
\end{figure}

\newpage
\tableofcontents

\pagenumbering{arabic}
\newpage 
\section{Introduction}

Rogue waves are large and unexpected waves that occur in nature. For example,  
giant water waves will can randomly appear on the ocean surface. These freak 
occurences pose a threat to ships and ocean liners. Rogue waves have also been 
documented in laser beam channels. 
This phenomenon has opened up many doors in research related to non-linear 
optics. (\cite{roguewaves})

Phycisists and mathematicans are interested in modeling and understanding 
rogues waves. They are particularly interested in classifying the stability of 
these waves and using equations to model their behaviour as a function of 
measurable physical parameters. My supervisor, Dr. Dmitry Pelinovsky, and I 
will use a particular matrix equation (lax pair), that is satisfied for wave 
solutions of the Non-Linear Schr\"{o}dinger (NLS) equation, in our 
investigation of rogue waves. The NLS equation is a good tool for modeling 
rogue waves and the spectrum of the previously mentioned matrix equation 
contains useful information regarding the behaviour of such waves. Further 
details can be read in section 5.1. 

The purpose of the spectral search program (\progname) is to find a numerical 
algorithm that is best for 
finding the spectrum of the previously mentioned lax operator for particular 
boundary conditions. We will run simulations with these different algorithms to 
see which is best at mimicing the current theoretical work. Once we build 
confidence we will extend the program beyond the boundary conditions. This 
extension will provide insight into the next stage of our research - 
analytically finding the entire continuous spectrum of the operator. 

\subsection{Purpose of Document}
The purpose of this document is to describe the requirements for \progname. 
\progname will search for the 
continuous spectrum of a lax pair that is compatible with solutions 
of the NLS equation. 

This document will explain the mathematical models, assumptions, 
solution characterisitcs and goals of the software. After reading this document 
one should be able to understand the mathematical and physical context of the 
inputs and outputs. They should also recognize the connections and 
dependencies between these SRS topics. 

The SRS is meant to be simaltaneously unambigious and abstract. It details the 
quality attributes of the software, such as functionality, 
without dicussing any code, numerical algorithms or scientific computing 
solutions. It is a precursor to the documents required in the development of 
scientific software. One should read and understand this document before the 
VnV, design or code itself. 

\subsection{Scope of Requirements} 

The scope of SpecSearch is limited to finding the spectrum of a particular 
linear lax equation and determining the stability of the corresponding waves. 
The waves being investigated are solutions to 
the focusing NLS equation. 

\subsection{Characteristics of Intended Reader} 

The reader of this document should have taken an introductory course 
in partial differential equations and complex analysis. They should also have a 
first year undergraduate understanding of linear algebra.

\subsection{Organization of Document}

This document follows a template provided by Dr. Spencer Smith at McMaster 
University (\cite{SmithAndLai2005},\cite{SmithEtAl2007}). It begins by 
introducing the document. This 
introduction is 
followed by a general overview of the system and then an outline of the goals 
and mathematical theory. The document continues by describing the behaviour 
between inputs 
and outputs, judging output and then foreshadows changes to the software. It 
concludes with useful graphs and figures for tracebility. 

The latex template is available in Dr. Spencer Smith's gitlab 
repository CAS 741. \url{https://gitlab.cas.mcmaster.ca/smiths/cas741}

\section{General System Description}

This section identifies the interfaces between the system and its environment,
describes the user characteristics and lists the system constraints.

\subsection{System Context} 
Diagram 1 shows the system context. The circle represents the user of the 
scientific software. The rectangle resembles the software system. The arrows 
indicate the flow of data between software and environment. The inputs are the 
physical wave parameters and the output is the spectrum of the matrix operator 
and error analysis.
 \begin{figure}[h!]
	\begin{center}
		{
			\includegraphics[width=15cm,height=6cm]{SystemContext.png}
		} 
	\caption{System Context}
	\end{center}
 \end{figure}

\begin{itemize}
\item User Responsibilities:
\begin{itemize}
\item Ensure that the input variables resemble the wave that you intend 
to analyze.
\item Ensure that the assumptions imposed on waves in this software are 
reasonable 
for your reasearch.
\item Ensure that the software is used in a legal and ethical manner.

\end{itemize}
\item SpecSearch Responsibilities:

\begin{itemize}
\item Detect data type mismatch. 
\item Solve the lax equation associated with the given wave parameters. 
\item Connect the discrete spectrum in order to form a continuous spectrum. 
\item Display the continuous spectrum on the complex plane.
\end{itemize}
\end{itemize}

\subsection{User Characteristics} \label{SecUserCharacteristics}

The user should have a basic understanding of wave mechanics (wave 
speed, 
amplitude and angular frequency). They should also understand the concept of 
conserved quantities. 

\subsection{System Constraints}

My supervisor requires that software will be created with MATLAB.


\section{Specific System Description}

This section first presents the problem description, which gives a high-level
view of the problem to be solved.  This is followed by the solution 
characteristics specification, which presents the assumptions, theories, 
definitions and finally the instance models. 

\subsection{Problem Description} \label{Sec_pd}
A lax pair is a set of matrices or operators that satisfy differential 
equations. The NLS equation, a PDE used to model rogue waves and modulated wave 
packets, appears as a compatibility condition of a particular lax pair of 
equations. One equation is a spectral problem, with matrix U, and the other is 
a time evolution problem, with matrix V. See \ref{sec_instance} for more 
details. The spectral problem is important because the spectral parameter 
within it contains information about the stability of solutions.

SpecSearch will attempt to produce a numerical approximation of the continuous 
spectrum for the previously mentioned spectral problem. In particular, 
it will find the spectrum for general travelling periodic wave solutions of the 
focusing
NLS equation.  

Previous attempts have used an algebraic method to calculate the spectral 
parameter \cite{SegaletAl}. These attempts have only been 
successful at finding a countable number of points on the spectrum. This 
software will 
search for more points and ``connect" the points in an attempt to approximate 
and search for the continuous spectrum.  

\subsubsection{Terminology and  Definitions}

This subsection provides a list of terms that are used in the subsequent
sections and their meaning, with the purpose of reducing ambiguity and making it
easier to correctly understand the requirements:

\begin{itemize}

\item \textbf{Spectrum}: The set of allowable values for the spectral parameter 
of a matrix or operator.
\item \textbf{Operator}: A mapping that transforms elements of a space into 
other elements of the same space. 
\item \textbf{Traveling Periodic Wave}: A periodic and one-dimensional wave 
that travels with a constant speed, c, and angular frequency, $\omega$. 
\item \textbf{Compatibility Condition}: Conditions for which the lax pair of 
equations is guaranteed. 
\item \textbf{Stability}: A solution is stable if slight perturbations in 
initial conditions lead 
to slight perturbations in the entire solution.
\item \textbf{Initial Data}: The graph of a wave (or its derivative) 
at a fixed point in time, tyically t=0.
\item \textbf{Rogue Wave}: A wave is considered rogue if its amplitude is more 
than double of the average of the upper third wave amplitudes in its 
surroundings. 
\item \textbf{PDE}: A partial differential equation.An equation containing one 
or more partial derivatives. The partial derivative of a multivariable function 
is a derivative with respect to one of its arguments. 
\item \textbf{Countable:} A set that is finite or has the same cardinality as 
the natural numbers (1,2,3,4,5 ....). 
\item \textbf{Focusing wave:} When the nonlinear Schr\"{o}dinger equation adds 
the 
nonlinear term ($+|u|^{2}u$).  
\item \textbf{Hamiltonian:} Is an operator corresponding to the total energy of 
a system. 

\end{itemize}

\subsubsection{Physical System Description}

The physical system of SpecSearch includes the following elements:

\begin{itemize}

\item[PS1:] 
Unspecified body of water or laser channel with a traveling periodic 
wave.

\end{itemize}

\subsubsection{Goal Statements}

\noindent Given physical parameters of the rogue wave and numerical 
differentiation method, the goal 
statement(s) are:

\begin{itemize}[leftmargin=.75in]

\item[GSlocate:] {Locate elements in the spectrum of the lax pair matrix for 
particular solutions wave solutions(see 
\ref{sec_instance}) }
\item[GStheor:] {Compare theoretical results to the approximated spectrum.}

\end{itemize}

\subsection{Solution Characteristics Specification}

The instance model that governs SpecSearch is presented in
Subsection~\ref{sec_instance}.  The information to understand the meaning of the
instance models and their derivation is also presented, so that the instance
models can be verified.

\subsubsection{Assumptions}

This section simplifies the original problem and helps in developing the
theoretical model by filling in the missing information for the physical
system. The numbers given in the square brackets refer to the theoretical model
[T], general definition [GD], data definition [DD], instance model [IM], or
likely change [LC], in which the respective assumption is used.

\begin{itemize}[leftmargin=.5in]

\item[Aham:]The 
wave equation is a complex hamiltonian evolution equation. 
\item[Amom:]The 
linear momentum densities are proportional to the integrands of the 
corresponding velocity functionals 
\item[Anls:]The 
wave is a solution to the NLS equation.
\item[Afoc:]The 
system only deals with focusing waves.
\end{itemize} 
Aham, Amom, Anls and Afoc are assumptions that are used in the derivation of 
the focusing NLS Equation \ref{TM1}. This equation is the foundation for all of 
the general defitions, data definitions and instance models. 

\begin{itemize}[leftmargin=.5in]
\item[Awav:]All 
waves in the model are general traveling perioidic waves with constant speed 
and frequency. 
\end{itemize} 

This assumption allows us to reduce TM1 \ref{T1} to GD1 \ref{GD1}.

\begin{itemize}[leftmargin=.5in]
	\item[Astat:] The system will start by assuming that the wave is stationary 
	and 
	then use 
	the stationary solution to build the moving solution.	
\end{itemize} 

This assumption allows us to set $c=0$ in DD1 \ref{DD1}. The equation in DD2 
\ref{DD2} follows from this simplification. 

\begin{itemize}[leftmargin=.5in]
	\item[Asmooth:] The eigenfunctions of the spectrum are two times 
	differentiable.
\end{itemize} 

This assumption is necessary for the compatibility condition in IM1 \ref{IM1}.

\newpage
\subsubsection{Theoretical Models}\label{sec_theoretical}

This section focuses on the Theoretical Models that SpecSearch is based
on. 
~\newline

\noindent
\begin{minipage}{\textwidth} \label{T1}
	\renewcommand*{\arraystretch}{1.5}
	\begin{tabular}{| p{\colAwidth} | p{\colBwidth}|}
		\hline
		\rowcolor[gray]{0.9}
		Number& TM1\\
		\hline
		Label&\bf Focusing NLS Equation\\
		\hline
		Equation&  $iq_{t} + q_{xx} + 2|q|^{2}q=0$\\
		\hline
		Description & 
		``The NLS equation appears as one of universal equations that describe 
		the evolution of slowly varying packets of quasi-monochromatic waves in 
		weakly nonlinear media that have dispersion."(Malomed 2005) Here $t$ 
		represents time, $x$ represents position and $q$ is the wave.\\
		\hline
		Source &
		\cite{SegaletAl}\\
		% The above web link should be replaced with a proper citation to a 
		%publication
		\hline
		Ref.\ By & \ref{DD1}, \ref{GD1}\\
		\hline 
		Assumptions & Aham, Amom, Anls, Afoc \\
		\hline 
	\end{tabular}
\end{minipage}\\ 

\subsubsection{General Definitions}\label{sec_gendef}

This section focuses on the general equations and laws that SpecSearch is based
on. 
~\newline

\noindent
\begin{minipage}{\textwidth} \label{GD1} 
	\renewcommand*{\arraystretch}{1.5}
	\begin{tabular}{| p{\colAwidth} | p{\colBwidth}|}
		\hline
		\rowcolor[gray]{0.9}
		Number& GD1\\
		\hline
		Label&\bf Focusing NLS for General travelling periodic waves\\
		\hline
		Equation&  $ u'' + 2|u|^{2}u+2icu'=\omega u$\\
		\hline
		Description & 
		General traveling periodic waves that satisfy the Schr\"{o}dinger 
		equation 
		will have this form. $c$ is the wave speed and $\omega$ is the 
		frequency. \\
		\hline
		Source &
		\cite{SegaletAl}\\
		% The above web link should be replaced with a proper citation to a 
		%publication
		\hline
		Ref.\ By & \ref{DD1}, \ref{IM1}\\
		\hline 
		Assumptions & Aham, Amom, Anls, Afoc, Awav \\
		\hline 
	\end{tabular}
\end{minipage}\\ 

\begin{center}
	\begin{flushleft}
		\textbf{Derivatition of the Focusing NLS for 
			General Traveling Periodic 
			Waves}
	\end{flushleft} 	
\end{center} 
We start with the focusing NLS equation: $iq_{t} + q_{xx} + 
2|q|^{2}q=0$ (\ref{sec_theoretical}) \\ 

The NLS equation is a PDE meant to model modulated wave packets and rogue waves 
in physics. $u$ is the complex envelope of the wave, $t$ is time and $x$ is the 
position in one dimensional space. \\ 

A general traveling periodic wave has the form $z(x,t)=u(x+2ct)e^{i \omega 
	t}$.$(u: \mathbb{R} \rightarrow \mathbb{C})$. 
We will now search for general traveling periodic waves that are solutions to 
the NLS equation. This is done by setting z=q in the NLS equation. This 
yields: \\ 
\begin{center}
	$ iz_{t} + z_{xx} + 2|z|^{2}z = 0$ \\ 
	$ \Rightarrow i(2cu'e^{i \omega t} + i \omega ue^{i \omega t}) + u''e^{i 
	\omega 
		t} + 2|u|^{2}|e^{i \omega t}|^{2} u e^{i \omega t} = 0 $ \\
	$\Rightarrow e^{i \omega t} (2icu' + i^{2} \omega u + u'' + 2|u|^{2}u) = 0 
	$ 
	\\ 
	$ \Rightarrow 2icu' - \omega u + u'' + 2|u|^{2}u =0$ \\ 
\end{center}
Which is the result in GD1.

\subsubsection{Data Definitions}\label{sec_datadef}

This section collects and defines all the data needed to build the instance
models. 
~\newline

\noindent
\begin{minipage}{\textwidth} \label{DD1}
\renewcommand*{\arraystretch}{1.5}
\begin{tabular}{| p{\colAwidth} | p{\colBwidth}|}
\hline
\rowcolor[gray]{0.9}
Number& DD\refstepcounter{datadefnum}\thedatadefnum \\
\hline
Label& \bf Conservation Equations\\
\hline
Symbols &$g, c, d, \omega$\\
\hline
% Units& $Mt^{-3}$\\
% \hline
  SI Units & -\\
  \hline
  Equation(1)&$\bar{u}u' - u\bar{u}' + 2ic|u|^{2} = 2ig$\\
  Equation(2)&$|u'|^{2} + |u|^{4} + d = \omega |u|^{2}$\\
  \hline
  Description & 
                These are conserved quantities in the NLS equation. For fixed 
                g and d the above relations are true for all space and time.
  \\
  \hline
  Sources& \cite{SegaletAl} \\
  \hline
  Ref.\ By & \ref{IM1}\\
  \hline 
   Assumptions & Aham, Amom, Anls, Afoc, Awav \\
  \hline 
\end{tabular}\\
\end{minipage} 

\begin{center}
	\begin{flushleft}
		\textbf{Justification of a Conserved Quantity}
	\end{flushleft} 
	
\end{center} 

I will first multiply T1 by $\bar{u}$. \\ 
\begin{center}
$  u''\bar{u} + 2|u|^{2}u \bar{u}+2icu' \bar{u}=\omega u \bar{u}$ (*) \\ 
\end{center}
Taking the complex conjugate yields: \\ 
\begin{center}
$ \bar{u''} u + 2|u|^{2} \bar{u} u -2ic\bar{u'}u = \omega \bar{u} u$ (**) \\ 
\end{center} 
Subtraction of * with ** yields: \\ 
\begin{center} 
$  u''\bar{u} + 2|u|^{2}u \bar{u}+2icu' \bar{u} - (\bar{u''} u + 2|u|^{2} 
\bar{u} u -2ic\bar{u'}u) =\omega u \bar{u} - (\omega \bar{u} u)$\\ 
$ \Rightarrow  u''\bar{u} - \bar{u''} u + 2ic (u' \bar{u} + \bar{u'}u ) = 0$ \\
$ \Rightarrow \int u''\bar{u} - \int \bar{u''} u + 2ic \int (u' \bar{u} + 
\bar{u'}u ) = 2ig  $ \\  
\end{center} 
Noticing that the term in the integral attached to 2ic is a product rule.\\ 
\begin{center}
$ \Rightarrow \int u''\bar{u} - \int \bar{u''} u + 2icu \bar{u} = 2ig $ \\
$ \Rightarrow \bar{u}u' - \int \bar{u'} u' - u \bar{u'} + \int \bar{u'} u' + 
2icu \bar{u} = 2ig$ \\ 
$ \Rightarrow \bar{u}u' - u\bar{u}' + 2ic|u|^{2} = 2ig$ \\ 
\end{center} 
The second conserved quantity is obtained by multiplying $\bar{u'}$ by T1, 
adding the complex conjugate and integrating. \\ 

\noindent
\begin{minipage}{\textwidth} \label{DD2}
	\renewcommand*{\arraystretch}{1.5}
	\begin{tabular}{| p{\colAwidth} | p{\colBwidth}|}
		\hline
		\rowcolor[gray]{0.9}
		Number& DD\refstepcounter{datadefnum}\thedatadefnum \label{FluxCoil}\\
		\hline
		Label& \bf Solutions of GD1\\
		\hline
		Symbols &$k, cn, sn, dn,a,b$\\
		\hline
		% Units& $Mt^{-3}$\\
		% \hline
		SI Units & -\\
		\hline
		Equation &$ \rho(x) = b - a^{2}k^{2}sn^{2}(ax;k)$\\
		\hline
		Description & 
		This is a solution to the focusing NLS for 
		General Traveling Periodic Waves. $k\in [0,1]$, $sn$ is the jacobi 
		elliptic function and $a,b$ are related to the conserved quantities. 
		The constraints on $a,b$ are $b \leq a^{2}$ and $b \geq a^{2}k^{2}$. 
		Along two boundaries of this triangular region and with $a=1$ the 
		equation simplifies to $ \rho(x) = dn(x;k) $ and $ \rho(x) = kcn(x;k)$.
		The derivation of these formulae are extremely long and will be omitted 
		from this document.  
		\\
		\hline
		Sources& \cite{SegaletAl} \\
		\hline
		Ref.\ By & \ref{IM1}\\
		\hline 
		Assumptions & Aham, Amom, Anls, Afoc, Awav \\
		\hline 
	\end{tabular}\\
\end{minipage} 

\subsubsection{Instance Models} \label{sec_instance}    

This section transforms the problem defined in Section~\ref{Sec_pd} into 
one which is expressed in mathematical terms. It uses concrete symbols defined 
in Section~\ref{sec_datadef} to replace the abstract symbols in the models 
identified in Sections~\ref{sec_theoretical} and~\ref{sec_gendef}. \\

The goal GSlocate is solved by IM1.  

%Instance Model 1

\noindent
\begin{minipage}{\textwidth} \label{IM1}
\renewcommand*{\arraystretch}{1.5}
\begin{tabular}{| p{\colAwidth} | p{\colBwidth}|}
  \hline
  \rowcolor[gray]{0.9}
  Number& IM\refstepcounter{instnum}\theinstnum \label{ewat}\\
  \hline
  Label& \bf Searching for the continuous spectrum\\
  \hline
  Input& Boundary wave solutions to NLS equation (\ref{DD2})\\
  \hline
  Output&$\lambda$ such that:\\
  &$\phi_x = U(u,\lambda) \phi$\\
  \hline 
  
  Description& 
  Let $u$ be a solution to \ref{sec_gendef}. 
  $U(u,\lambda) = \begin{pmatrix} 
  \lambda & u \\
  -\bar{u} &-\lambda 
  \end{pmatrix}$ \\
  &T1 is a compabililty condition for the lax pair comprised of the above row's 
  ODE and $\phi_{t}=V(u,\lambda)\phi$\\
  &$V(u,\lambda)=i \begin{pmatrix} 
  2 \lambda^{2} + |u|^{2} & u_{x}+2 \lambda u \\
  \bar{u}_{x}-2\lambda \bar{u} & -2\lambda^{2} - |u|^{2}
  \end{pmatrix}$ \\
  &$\phi \in \mathbb{C}$\\
  \hline
  Sources& \cite{SegaletAl} \\
  \hline
  Ref.\ By & \\
  \hline 
  Assumptions & Aham, Amom, Anls, Afoc, Asmooth \\
  \hline 
\end{tabular}
\end{minipage}\\
 

\begin{center}
	\begin{flushleft}
		\textbf{Justification of Compatibility Conditions and Reduction to 
		Spectral Problem}
	\end{flushleft} 
	
\end{center} 
For a smooth $C^{2}$ function ,$\phi$, we have that $\phi_{tx}=\phi_{xt}$. \\
From IM1 $\Rightarrow \phi_{xt} = \delta_{t} U \phi + UV \phi$ and $\phi_{tx} = 
\delta_{x} V \phi + VU \phi$. \\
Combining the above two equations : \\ 

$\Rightarrow \delta_{t} U \phi + UV \phi = \delta_{x} V \phi + VU \phi$ \\
$\Rightarrow \delta_{t} U + UV = \delta_{x} V + VU$ (--) \\ 
$\Rightarrow$
$\begin{pmatrix} 
	0 & u_{t} \\
	-\bar{u}_{t} & 0
\end{pmatrix}$ +
$i \begin{pmatrix} 
\lambda & u \\
-\bar{u} & -\lambda
\end{pmatrix}$ 
$ \begin{pmatrix} 
2\lambda^{2} + |u|^{2} & u_{x}+2\lambda u \\
\bar{u}_{x} -2\lambda\bar{u} & -2\lambda^{2} - |u|^{2}
\end{pmatrix} =$ 
$ i \begin{pmatrix} 
u_{x}\bar{u} + u\bar{u}_{x} & u_{xx} + 2\lambda u_{x} \\
\bar{u}_{xx} - 2 \lambda \bar{u}_{x} & -u_{x} \bar{u} - u_{x}\bar{u}_{x}
\end{pmatrix} +$ 
$i \begin{pmatrix} 
2 \lambda^{2} + 2|u|^{2} & u_{x} + 2\lambda u_{x} \\
\bar{u}_{x} - 2 \lambda \bar{u} & -2 \lambda^{2} - |u|^{2}
\end{pmatrix}$ 
$\begin{pmatrix} 
	\lambda & u \\
	- \bar{u} & - \lambda
\end{pmatrix}$
\\
$\Rightarrow \begin{pmatrix} 
	i(2 \lambda ^{3} + \lambda |u|^{2} + u\bar{u}_{x} -2\lambda |u|^{2}) & 
	u_{t} + i(\lambda u_{x} + 2 \lambda^{2} u - 2 \lambda^{2} u - u|u|^{2}) \\
	-\bar{u}_{t} + i(-2 \lambda^{2}\bar{u} - \bar{u}|u|^{2} - 
	\lambda\bar{u}_{x} + 2 \lambda^{2} \bar{u}) & i(-\bar{u} u_{x} - 2 \lambda 
	u \bar{u} + 2 \lambda^{3} + \lambda |u|^{2})
\end{pmatrix} $ = \\
$ i \begin{pmatrix} 
	u_{x}\bar{u} + u\bar{u}_{x} + 2 \lambda^{3} + \lambda |u|^{2} - 
	u_{x}\bar{u} - 2\lambda |u|^{2} & u_{xx} + 2\lambda u_{x} + 2\lambda^{2} u 
	+ u|u|^{2} - 2|u|^{2} - \lambda u_{x} \\
	\bar{u}_{xx} - 2\lambda \bar{u}_{x} + \lambda \bar{u}_{x} - 
	2\lambda^{2}\bar{u} + 2\lambda^{2}\bar{u} + \bar{u}|u|^{2} & -u_{x}\bar{u} 
	- u_{x}\bar{u}_{x} + u\bar{u}_{x} - 2\lambda |u|^{2} + 2\lambda^{3} + 
	\lambda |u|^{2}
\end{pmatrix}$ \\
$ \Rightarrow \begin{pmatrix} 
0 & u_{t} - iu|u|^{2} \\
-\bar{u}_{t} - i\bar{u}|u|^{2} & 0
\end{pmatrix} $ = 
$ \begin{pmatrix} 
0 & iu_{xx} + iu|u|^{2} \\
i\bar{u}_{xx} + i \bar{u}|u|^{2} & 0
\end{pmatrix} $ \\ 
% new line
$ \Rightarrow \begin{pmatrix} 
0 & iu_{t} + 2u|u|^{2} + u_{xx} \\
iu_{t} + 2u|u|^{2} + u_{xx} & 0
\end{pmatrix} $ = 
$ \begin{pmatrix} 
0 & 0 \\
0 & 0
\end{pmatrix} $ \\ 

This is true if u satisfies the NLS equation (T1). Thus, the NLS equation is a 
compatibility 
condition for the lax pair in IM1. \\

Now let $\phi = (\phi_{1} , \phi_{2})^{T} $. The lax operator in the output row 
of the IM1 table \ref{IM1} can be rewritten as follows: \\  

$\lambda $
$ \begin{pmatrix} 
\phi_{1} \\
\phi_{2} 
\end{pmatrix} $ = 
$ \begin{pmatrix} 
\frac{d}{dx} & -u \\
-u & -\frac{d}{dx}
\end{pmatrix} $ 
$ \begin{pmatrix} 
\phi_{1} \\
\phi_{2} 
\end{pmatrix} $\\ 

Where $u$ is a boundary general traveling periodic wave solution to the 
focusing NLS equation from the description of \ref{DD2} and multiplication with 
operator $\frac{d}{dx}$ is differentation. Thus, finding $\lambda$ is 
equivalent to finding the eigenvalues of IM1. \\

\newpage
\subsubsection{Data Constraints} \label{sec_DataConstraints}    

There are no data constraints. 

\subsubsection{Properties of a Correct Solution} \label{sec_CorrectSolution}

\noindent
A correct solution will be a list of complex numbers. There should be at least 
four purely complex elements of the spectrum for each cn instance. There should 
also 
be at least two purely real elements of the spectrum for each dn instance. 

\section{Requirements}

This section provides the functional requirements, the business tasks that the
software is expected to complete, and the nonfunctional requirements, the
qualities that the software is expected to exhibit.

\subsection{Functional Requirements}

\noindent \begin{itemize}

\item[Rin:] SpecSearch shall take in quantity $k$ and numerical constants 
$N,P$ as input. (\ref{sec_gendef})

\item[Rfind:] SpecSearch 
will find values of $\lambda$. (\ref{IM1})

\item[Rplt:] SpecSearch will plot the spectrum on the complex plane for the cn 
and dn solutions for three different numerical algorithms. The 
$x$-axis is the real part and the $y$-axis is the imaginary part. Certain 
spectral elements from (segal) will be plotted on the appropriate figures.


\end{itemize}

\subsection{Nonfunctional Requirements}
\noindent \begin{itemize}
\item[NFR1:] The software should be maintainable and manageable as it will 
be modified and continually updated during the research process. Adding a new 
numerical method should simply involve modifying a single matrix definition in 
the MATLAB code.  The elements of the matrix should be thouroughly commented. 
Modification of this matrix should take at most $\frac{1}{4}$ of the 
development time of the original development of the matrix. 
\item[NFR2:] The software should be accurate within the standards of my 
supervisor and reliable for researchers studying rogue waves. This means that 
the numerical algorithms implemented should produce minor error terms and no 
sporous elements of the spectrum. 
\end{itemize}


\section{Likely Changes}    

\noindent \begin{itemize}

\item[LC\refstepcounter{lcnum}\thelcnum\label{LC_meaningfulLabel}:] We might 
add constraints or bounds to the input variables. 

\end{itemize}

\section{Traceability Matrices and Graphs}

The purpose of the traceability matrices is to provide easy references on what
has to be additionally modified if a certain component is changed.  Every time a
component is changed, the items in the column of that component that are marked
with an ``X'' may have to be modified as well.  Table~\ref{Table:B_trace} shows 
the
dependencies of theoretical models, general definitions, data definitions, and
instance models with each other. Table~\ref{Table:R_trace} shows the
dependencies of instance models, requirements, and data constraints on each
other. Table~\ref{Table:A_trace} shows the dependencies of theoretical models,
general definitions, data definitions, instance models, and likely changes on
the assumptions.\\

\begin{table}[h!]
\centering
\begin{tabular}{|c|c|c|c|c|c|c|c|c|c|c|c|c|c|c|c|c|c|c|c|c|c|c|c|}
\hline        
	& T1& GD1& DD1 & DD2&IM1 \\
\hline
T1     &X &X &X &X &  \\ \hline
GD1    & &X &X &X & \\ \hline
DD1    & & &X &X & \\ \hline 
DD2    & & & &X &   \\ \hline 
IM1    & & & & &X \\
\hline
\end{tabular}\\
\caption{Traceability Matrix Showing the Connections Between Items of 
Different 
Sections}
\label{Table:B_trace}
\end{table}

\begin{table}[h!]
	\centering
	\begin{tabular}{|c|c|c|c|c|c|c|c|c|c|c|c|c|c|c|c|c|c|c|c|c|c|c|c|}
		\hline        
		& IM1& Rin& Rfind & Rplt  \\
		\hline
		IM1     &X & & &  \\ \hline
		Rin    & &X & &  \\ \hline
		Rfind    &X &X &X & \\ \hline 
		Rplt    &X &X &X &X   \\ 
		\hline
	\end{tabular}\\
	\caption{Traceability Matrix Showing the Connections Between Requirements 
	and Instance Models}
	\label{Table:R_trace}
\end{table} 

\newpage

	\begin{table}[h!]
		\centering
		\begin{tabular}{|c|c|c|c|c|c|c|c|c|c|c|c|c|c|c|c|c|c|c|c|}
			\hline
			& Aham & Amom& Anls& Afoc& Awav& Astat & Asmooth \\
			\hline
			TM1        &X &X &X &X && & \\ \hline
			GD1       &X &X &X &X &X && \\ \hline
			DD1       &X &X &X &X &X &X&\\ \hline
			DD2       &X  &X  &X  &X  &X &X& \\ \hline
			IM1       &X &X &X &X &X &X&X \\ 
			\hline
		\end{tabular}
		\caption{Traceability Matrix Showing the Connections Between 
		Assumptions and 
			Other Items}
		\label{Table:A_trace}
	\end{table}


The purpose of the traceability graphs is also to provide easy references on
what has to be additionally modified if a certain component is changed.  The
arrows in the graphs represent dependencies. The component at the tail of an
arrow is dependent on the component at the head of that arrow. Therefore, if a
component is changed, the components that it points to should also be
changed. Figure~\ref{Fig_ATrace} shows the dependencies of theoretical models,
general definitions, data definitions and instance models on each other. 
Figure~\ref{Fig_RTrace} shows the dependencies of
instance models, requirements, and data constraints on each other.

\begin{figure}[h!]
	\begin{center}
		{
			\includegraphics[width=\textwidth]{ATrace.png}
		}
		\caption{\label{Fig_ATrace} Traceability Graph Showing the Connections 
		Between Items of Different Sections}
 	\end{center}
 \end{figure}

\begin{figure}[h!]
	\begin{center}
		{
			\includegraphics[width=\textwidth]{RTrace.png}
		}
 		\caption{\label{Fig_RTrace} Traceability Graph Showing the Connections 
 			Between Items of Different Sections}
	\end{center}
\end{figure}


\clearpage
\bibliographystyle {plainnat}
\bibliography {../../ReferenceMaterial/References} 

%\clearpage 
%\begin{thebibliography}{9} 
%	\bibitem{latexcompanion} 
%	Spencer Smith and Lei Lai. 
%	A New Requirements Template for Scientific Computing. 
%	Proceedings of SREP'05, Paris, France 2005. 
	
%	\bibitem{latexcompanion} 
%	Spencer Smith, Lei Lai and Ridha Khedri. 
%	Requirements Analysis for Engineering computation: A Systematic Approach 
%	for Improving Reliability. 
%	Springer, 2007. 
	
%	\bibitem{latexcompanion} 
%	Dmitry E. Pelinovsky. 
%	Localization in Periodic Potentials. 
%	Cambridge University Press, 2011. 
	
%	\bibitem{latexcompanion} 
%	Bernard Deconinck and Benjamin L.Segal. 
%	The stability spectrum for elliptic solutions to the focusing NLS equation. 
%	PhysicaD, 2017.  
	
%	\bibitem{latexcompanion} 
%	J. Chen and D.E. Pelinovksy. 
%	Rogue periodic waves in the focusing nonlinear Schrodinger equation. 
%	Proceeding A of Roy.Soc. Lond., 2018. 
	
%	\bibitem{latexcompanion} 
% 	 Malomed, Boris  
% 	 "Nonlinear Schrödinger Equations"  
% 	 Scott, Alwyn, Encyclopedia of Nonlinear Science (2005) 
% 	 New York: Routledge, pp. 639–643
	
%\end{thebibliography} 

\newpage

\section{Appendix}

\subsection{Symbolic Parameters}

There are no symbolic parameters.  


\end{document}
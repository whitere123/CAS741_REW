\documentclass[12pt, titlepage]{article}

\usepackage{amsmath, mathtools}

\usepackage[round]{natbib}
\usepackage{amsfonts}
\usepackage{amssymb}
\usepackage{graphicx}
\usepackage{colortbl}
\usepackage{xr}
\usepackage{hyperref}
\usepackage{longtable}
\usepackage{xfrac}
\usepackage{tabularx}
\usepackage{float}
\usepackage{siunitx}
\usepackage{booktabs}
\usepackage{multirow}
\usepackage[section]{placeins}
\usepackage{caption}
\usepackage{fullpage} 
\usepackage{amsmath}
\hypersetup{
	colorlinks,
	citecolor=black,
	filecolor=black,
	linkcolor=red,
	urlcolor=blue
}
\usepackage[round]{natbib}
\usepackage{tikz}
\def\checkmark{\tikz\fill[scale=0.4](0,.35) -- (.25,0) -- (1,.7) -- (.25,.15)-- 
cycle;}

\input{../../Comments}
%% Common Parts

\newcommand{\progname}{SpecSearch} % PUT YOUR PROGRAM NAME HERE %Every program
                                % should have a name


\begin{document}

\title{\progname: System Verification and Validation Test Report} 
\author{Robert E. White}
\date{\today}
	
\maketitle

\pagenumbering{roman}

\section{Revision History}

\begin{tabularx}{\textwidth}{p{3cm}p{2cm}X}
\toprule {\bf Date} & {\bf Version} & {\bf Notes}\\
\midrule
2018-12-07 & 1.0 & First draft of SystVnV test report.\\
2018-12-09 & 1.1 & Creation of final draft for final documentation. \\
\bottomrule
\end{tabularx}

~\newpage

\section{Symbols, Abbreviations and Acronyms}

\renewcommand{\arraystretch}{1.2}
\begin{tabular}{l l} 
	\toprule		
	\textbf{symbol} & \textbf{description}\\
	\midrule 
	T & Test\\
	R & Requirement\\ 
	NFR & Non-functional Requirement\\
	\bottomrule
\end{tabular}\\

Refer to the SRS Symbols, Abbreviations and Acronyms for a more 
complete list (\cite{SRS}) \url{https://github.com/whitere123/CAS741_REW}. 

\newpage

\tableofcontents

\listoftables %if appropriate

\listoffigures %if appropriate

\newpage

\pagenumbering{arabic}

This document will briefly summarize the tests outlined in the System 
Verification and Validation plan, explain the automated testing set-up, 
summarize the results of the tests and list the changes made in response to the 
tests. After reading this document one should be able to determine whether or 
not \progname \ satisfied its functional and nonfunctional requirements with 
respect to the system tests. They 
should also be able to trace a test to a particular module and a module to a 
particular requirement. 

\section{Functional Requirements Evaluation} 

The three functional requirements for \progname \ are Rin, Rfind and 
Rplt. More details on these requirements can be found in the System 
Requirements Specification, \cite{SRS}. There are six tests in the System 
Verification and Validation plan 
(\url{https://github.com/whitere123/CAS741_REW}) that cover functional 
requirements: test-Rin1NonNumeric, test-Rin1kBounds, test-Rin1NBounds, 
test-Rin1PBounds, test-Rfind-Rin and test-Rplt. The first five aforementioned 
tests are 
automated tests that check for exceptions and count the spectrum size.  
Test-Rplt is a visual analysis of the six spectral plots by my thesis 
supervisor. A summary of this test is presented at the end of this section. \\ 
A binary decision will be used to evaluate the tests. Each of the automated 
tests loop through a list of input and return a pass or fail message for each 
input. These automated tests and a summary of their results are presented in 
\ref{automate}. A test will be considered a failure if it returns at least one 
fail 
message. In this case the code will have to be modified until non of the tests 
return a fail message. The evaluation of test-Rplt is dependant on the 
standards of my supervisor. The code will have to be modified if he does not 
consider the plots to be adequate enough for his research. The evaluation of 
test-Rplt is given in the table below. A list of passing input (inputs that do 
not throw exceptions) was given to my supervisor. He ran the software with 
these input and gave a check mark if the numerical plot was of the correct form 
(ie 
appropriate symmetry and clusting of points). A final column full of 
check marks will indicate a passing test for test-Rplt. \\

\clearpage
\begin{table}[h!]
	\centering
	\begin{tabular}{|c|c|c|c|c|c|c|c|c|c|c|c|c|c|c|c|c|c|c|c|c|c|c|c|}
		\hline        
		Input ID& k& N & P & Result \\
		\hline
		I1      &0.6 &100 &2 & \checkmark \\ \hline
		I2     &0.1 &120 &2 & \checkmark \\ \hline
		I3     &0.9 &500 &2 & \checkmark\\ \hline 
		I4     &0.88 &550 &2 & \checkmark \\ \hline 
		I5      &0.99 &200 &2 & \checkmark \\ \hline
		I6     &0.65 &700 & 2& \checkmark \\ \hline
		I7    &0.4 &100 & 2& \checkmark \\ \hline 
		I8     &0.8 &400 &4& \checkmark \\ \hline
		I9      &0.9 &500 &4 & \checkmark \\ \hline
		I10     &0.2 &700 &4 & \checkmark \\ \hline
		I11     &0.3 &200 &4 & \checkmark \\ \hline 
		I13    &0.8 &100 &4 & \checkmark  \\ \hline
		I14      &0.89 &150 &4 & \checkmark \\ \hline
		I15     &0.69 &500 &4 & \checkmark \\ \hline
		I16     &0.55 &300 &2 & \checkmark \\ \hline 
		I17    &0.9 &400 &4 & \checkmark  \\ 
		\hline
	\end{tabular}\\
	\caption{test-Rplt}
	\label{Table:D_13}
\end{table}  

\section{Nonfunctional Requirements Evaluation}

The two Nonfunctional requirements for \progname \ are NFR1, which is related 
to maintainability and manageability, and NFR2, which is related to accuracy. 
More details on these requirements can be found in the System 
Requirements Specification, \cite{SRS}. Although not explicitly mentioned in 
the SRS, usabilty will also be tested. It is related to maintainability as my 
supervisor should be able to use the code after we have completed the project.\\
The three Nonfunctional requirement tests are outlined in the System 
Verification and Validation plan 
(\url{https://github.com/whitere123/CAS741_REW}). They are test-NFR1, test-NFR2 
and test-UserPerformance. Usability will recieve a mark out of 15. Each 
question is weighted according to importance by my supervisor. A question will 
not recieve marks if I was unable to answer 
it or if the response was below satisfactory. A failing grade on the usability 
survey is less than or equal to 11. \\
test-NFR2 was a code inspection that checked for maintainability and 
manageability. My supervisor inspected the code to see that each equation was 
implemented successfully and to check if he was able understand the module 
structure. Upon inspection my supervisor deemed the code adequate, maintainable 
and manageable. 

\subsection{Usability} 

The following usability table was proposed in the System Verification and 
Validation Plan. I have filled out my responses under each question in the 
following list. Questions 
with poor responses were addressed in \ref{changes}. Due to time constraints 
and other commitments I was the only person to fill out this survey.

\label{UsabilitySurvey}
\begin{itemize}
	\item How long did it take before you could run the software? How many 
	attempts at running SpecSeach did it take before you understood how to 
	properly use it and interpret the output? (1 marks) \\
	This question is not applicable to me as I was the person who created the 
	software. The response would be biased. 
	\item Was this program useful for your research and were you able to 
	interpret the results? (4 marks) \\ 
	Yes, this program was helpful for my research. The spectral plot helped me 
	to foreshadow the next stage of my research and understand the context of 
	the spectrum.\\
	Yes, I was able to interpret the results. The range of the axes was 
	appropriate and matlab has a convienient interface for analyzing figures. 
	In particular it allows you to click on particular points and display their 
	values. 
	\item What aspects of this software do you feel need improvement? (1 
	mark)\\ 
	The system performs adequately with respect to the functional requirements
	and within reasonable time.
	\item How does this program compare with other software that finds this 
	particular spectrum? (2 mark)\\ 
	I could not find another software package that performs the same task as 
	\progname. 
	\item Was it clear how and where to input the variables? (3 marks) \\
	Yes, the command line prompt gave very clear instructions. 
	\item Were the plots clear? (4 marks) \\
	Yes, the choice of markers was appropriate and it was easy to differentiate 
	between the different plots.
\end{itemize} 
\noindent
Based on the previously described metric, usability gets $\frac{12}{15}$. This 
test is related to test-UserPerformance.
		
\subsection{Checklist}

The following checklist was proposed in the System Verification and 
Validation Plan. This checklist is related to accuracy and usability. The 
evaluation metric is simple: the total number of checks divided by 5, the total 
number of elements in the checklist. A pass is a perfect score.

\label{softwarevercheck}
\begin{itemize}
	\item Non of the results surprised you.  \checkmark   
	\item Were you able to identify which numerical algorithm and wave solution 
	each plot represented?  \checkmark 
	\item Were all of the plots legible?  \checkmark 
	\item Was the output useful for your research?  \checkmark 
	\item Was it clear how to input the variables?  \checkmark 
\end{itemize}  
\noindent
The evaluation for this test is 1. 
\subsection{Inspection by Supervisor} 

The code inspection by my supervisor is amongst the most 
simple and important nonfunctional test. The metric is binary. Either my 
supervisor says that the code satisfies the nonfunctional requirements or 
doesn't. The code passed the inspection by my supervisor. 
	
\section{Comparison to Existing Implementation}	

My supervisor and I could not find any software that performs the same tasks as 
\progname. 

\section{Unit Testing} 

This section is not appropriate for the System Verification and Validation 
Plan. For more details about unit testing please refer to the Unit VnV Plan and 
Unit VnV Report at (\url{https://github.com/whitere123/CAS741_REW}). 

\section{Changes Due to Testing} \label{changes}

All of the tests passed. No changes were made to the software package.

\section{Automated Testing} \label{automate} 

 Each 
test involved cycling through a list of intial data and comparing 
the outputed result with the expected output.The following tables summarize the 
results for test-Rin1NonNumeric, 
test-Rin1kBounds, test-Rin1NBounds, test-Rin1PBounds and test-Rfind-Rin. 
Details about the Input, expected result and test result are given 
in the tables. A check mark indicates that the test 
returned the expected 
result. Descriptions of each test can be found in the SystVnVPlan at 
\url{https://github.com/whitere123/CAS741_REW}.  A test is considered a failure 
if at least one expected value does not match the output for one intial data 
configuration (one of the rows).

\begin{table}[h!]
	\centering
	\begin{tabular}{|c|c|c|c|c|c|c|c|c|c|c|c|c|c|c|c|c|c|c|c|c|c|c|c|}
		\hline        
		Input ID& k& N & P & Expected Result& Test Result \\
		\hline
		I1    &X &X &X & Exception & \checkmark \\ \hline
		I2    &X &30 &2 & Exception& \checkmark \\ \hline 
		I3    &X &X &2 &Exception & \checkmark \\ \hline 
		I4     &X &100 &X &Exception & \checkmark \\ \hline
		I5     &0.5 &X & 2& Exception& \checkmark \\ \hline
		I6    &0.8 &X &X & Exception & \checkmark \\ \hline
		I7      &X &X &4 & Exception & \checkmark \\ \hline
		I8     &X &1000 &X & Exception & \checkmark \\ \hline
		I9     &0.99 &X &X & Exception & \checkmark \\ \hline 
		I10    &0.6 &250 &X & Exception & \checkmark \\ 
		\hline
	\end{tabular}\\
	\caption{test-Rin1NonNumeric}
\end{table} 
\clearpage
\begin{table}[h!]
	\centering
	\begin{tabular}{|c|c|c|c|c|c|c|c|c|c|c|c|c|c|c|c|c|c|c|c|c|c|c|c|}
		\hline        
		Input ID& k& N & P &Expected Result& Test Result \\
		\hline
		I1     &1 &100 &2 & Error& \checkmark \\ \hline
		I2     &0 &100 &4 & Error& \checkmark \\ \hline 
		I3     &-10 &50 &2 &Error& \checkmark \\ \hline 
		I4      &-55 &200 &4 &Error& \checkmark \\ \hline
		I5     &2 &150 & 2& Error& \checkmark\\ \hline
		I6     &1.0001 & 100 &2 & Error& \checkmark \\ \hline
		I7     &-0.001 & 200 &4 & Error& \checkmark\\ \hline 
		I8     &1.01 & 200 &2 &Error& \checkmark \\ \hline 
		I9      &-0.01 &100 &4 &Error& \checkmark \\ \hline
		I10     &0.7 & -10 &2 & Error& \checkmark \\ \hline
		I11     &0.9 & 20 &4 & Error& \checkmark \\ \hline 
		I12     &0.9 & 5 &2 &Error & \checkmark \\ \hline 
		I13      &0.7 &0 &4 &Error & \checkmark \\ \hline
		I14     &0.5 & -1 & 2& Error& \checkmark \\ \hline 
		I15    &0.9 & 100 &0 & Error& \checkmark \\ \hline 
		I16     &0.9 & 500 &-2 &Error& \checkmark \\ \hline 
		I17      &0.7 &150 &-4 &Error& \checkmark \\ \hline
		I18     &0.5 & 200 & 100& Error& \checkmark \\ \hline 
		I19     &0.7 & 0.5 & 100& Error& \checkmark \\
		\hline
	\end{tabular}\\
	\caption{test-RinkBounds, test-RinNBounds,test-Rin1PBounds}
\end{table}  
\clearpage
\begin{table}[h!]
	\centering
	\begin{tabular}{|c|c|c|c|c|c|c|c|c|c|c|c|c|c|c|c|c|c|c|c|c|c|c|c|}
		\hline        
		Input ID& k& N & P & Expected Result & Test result \\
		\hline
		I1      &0.6 &100 &2 & 4N & \checkmark \\ \hline
		I2     &0.1 &120 &2 & 4N & \checkmark \\ \hline
		I3     &0.9 &500 &2 & 4N & \checkmark \\ \hline 
		I4     &0.88 &550 &2 & 4N & \checkmark \\ \hline 
		I5      &0.99 &200 &2 & 4N & \checkmark \\ \hline
		I6     &0.65 &700 & 2& 4N & \checkmark \\ \hline
		I7    &0.4 &100 & 2& 4N & \checkmark \\ \hline 
		I8     &0.8 &400 &4& 4N & \checkmark \\ \hline
		I9      &0.9 &500 &4 & 4N & \checkmark \\ \hline
		I10     &0.2 &700 &4 & 4N & \checkmark \\ \hline
		I11     &0.3 &200 &4 & 4N & \checkmark \\ \hline 
		I13    &0.8 &100 &4 & 4N & \checkmark  \\ \hline
		I14      &0.89 &150 &4 & 4N & \checkmark \\ \hline
		I15     &0.69 &500 &4 & 4N & \checkmark \\ \hline
		I16     &0.55 &300 &2 & 4N & \checkmark \\ \hline 
		I17    &0.9 &400 &4 & 4N & \checkmark  \\ 
		\hline
	\end{tabular}\\
	\caption{test-Rin-Rfind}
\end{table} 

\newpage		
\section{Trace to Requirements}

\begin{table}[h]
	\centering
	\begin{tabular}{|c|c|c|c|c|c|c|c|c|c|c|c|c|c|c|c|c|c|c|c|c|c|c|c|}
		\hline        
		& Rin& Rfind & Rplt & NFR1 & NFR2 \\
		\hline
		test-Rin1NonNumeric     &X & & & &  \\ \hline
		test-Rin1kBounds    & X& & & &   \\ \hline
		test-Rin1NBounds    &X & & & &  \\ \hline 
		test-Rin1PBounds    &X & & & &    \\ \hline 
		test-RFind-Rin    &X &X & & &   \\ \hline 
		test-Rplt    & & &X & &    \\ \hline 
		test-NFR1  & & & &X &   \\ \hline 
		test-NFR2  & & & & &X    \\ \hline 
		test-UserPerformance  & & & & &X  \\
		\hline
	\end{tabular}\\
	\caption{Traceability Between Test }
	\label{Table:D_1}
\end{table} 

\section{Trace to Modules}	
\clearpage
\begin{table}
	\centering
	\begin{tabular}{p{0.2\textwidth} p{0.6\textwidth}}
		\toprule
		\textbf{Test Case} & \textbf{Modules}\\
		\midrule
		test-Rin1NonNumeric & M1, M2, M13\\
		test-Rin1kBounds & M1, M2, M13\\
		test-Rin1NBounds & M1, M2, M13\\
		test-Rin1PBounds & M1, M2, M13\\
		test-RFind-Rin & M1, M2, M13, M4, M6, M7, M8, M9, M10,M11,M14\\
		Rfind & M4, M6, M7, M8, M9, M10,M11, M13, M14\\
		test-Rplt & M3, M11, M13 \\
		test-NFR1 & M1-M14 \\
		test-NFR2 & M1-M14 \\
		test-UserPerformance & M1-M14 \\
		\bottomrule
	\end{tabular}
	\caption{Trace Between Test Cases and Modules}
	\label{TblRT}
\end{table}	

\section{Code Coverage Metrics} 

No code covering metrics were used in the creation of \progname.

\clearpage
\bibliographystyle {plainnat}
\bibliography {../../../ReferenceMaterial/References} 

\end{document}
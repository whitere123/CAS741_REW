\documentclass[12pt, titlepage]{article}

\usepackage{booktabs}
\usepackage{tabularx}
\usepackage{hyperref}
\hypersetup{
    colorlinks,
    citecolor=black,
    filecolor=black,
    linkcolor=red,
    urlcolor=blue
}
\usepackage[round]{natbib}

%% Comments

\usepackage{color}

\newif\ifcomments\commentstrue

\ifcomments
\newcommand{\authornote}[3]{\textcolor{#1}{[#3 ---#2]}}
\newcommand{\todo}[1]{\textcolor{red}{[TODO: #1]}}
\else
\newcommand{\authornote}[3]{}
\newcommand{\todo}[1]{}
\fi

\newcommand{\wss}[1]{\authornote{blue}{SS}{#1}}
\newcommand{\an}[1]{\authornote{magenta}{Author}{#1}}


%% Common Parts

\newcommand{\progname}{SpecSearch} % PUT YOUR PROGRAM NAME HERE %Every program
                                % should have a name


\usepackage{tikz}
\def\checkmark{\tikz\fill[scale=0.4](0,.35) -- (.25,0) -- (1,.7) -- (.25,.15)-- 
	cycle;}

\begin{document}

\title{\progname: Unit Verification and Validation Test Report} 
\author{Robert E. White}
\date{\today}
	
\maketitle

\pagenumbering{roman}

\section{Revision History}

\begin{tabularx}{\textwidth}{p{3cm}p{2cm}X}
\toprule {\bf Date} & {\bf Version} & {\bf Notes}\\
\midrule
2018-12-07 & 1.0 & Creation of first draft.\\
2018-12-09 & 1.1 & Creation of final draft for final documentation. \\
\bottomrule
\end{tabularx}

~\newpage

\section{Symbols, Abbreviations and Acronyms}

\renewcommand{\arraystretch}{1.2}
\begin{tabular}{l l} 
	\toprule		
	\textbf{symbol} & \textbf{description}\\
	\midrule 
	T & Test\\
	R & Requirement\\ 
	NFR & Non-functional Requirement\\
	\bottomrule
\end{tabular}\\

Refer to the SRS Symbols, Abbreviations and Acronyms for a more 
complete list (\cite{SRS}) \url{https://github.com/whitere123/CAS741_REW}. 

\newpage

\tableofcontents

\listoftables %if appropriate

\listoffigures %if appropriate

\newpage

\pagenumbering{arabic}

This document will briefly summarize the unit tests outlined in the Unit 
Verification and Validation plan, explain the automated testing set-up, 
summarize the results of the tests and list the changes made in response to the 
tests. After reading this document one should be able to determine whether or 
not \progname \ satisfied its functional and nonfunctional requirements with 
respect to the unit tests. They should also be able to trace a unit test to a 
particular module and that module to a particular requirement. 

\section{Functional Requirements Evaluation} \label{funct}

The three functional requirements for \progname \ are Rin, Rfind and 
Rplt. More details on these requirements can be found in the System 
Requirements Specification, \cite{SRS}. There are nine tests in the Unit 
Verification and Validation plan 
(\url{https://github.com/whitere123/CAS741_REW}) that cover functional 
requirements: test-InParams-N, test-InParams-P, test-InParams-k, 
test-NumParams-Dom, test-NumParams-Dom-Bound, test-NumParams-EllipMat, 
test-SpecMAT-Tr1, test-Plotting-Inspect and test-Plotting-Inspect-Bound. The 
first seven tests are automated. These tests inspect the InParam, NumParam 
and SpecMat modules. Tests with ``-Bound" in their names cycle through the 
boundary analysis input table and verify that an assert statement returns true. 
The other tests cycle through a table of common inputs within their constraints 
and check to see that an assert statement returns true. A summary of these 
tests will be covered in \ref{automation}. \\
Test-Plotting-Inspect and test-Plotting-Inspect-Bound are similar to Test-Rplt 
in the SystVnVReport. The exception is that we are testing the output module 
and plotting module. We are more concerned with checking that theoretical 
eigenvalues overlap with the appropriate numerical eigenvalues instead of basic 
aspects of the plot like symmetry and correct number of eigenvalues. These 
tests involve a visual analysis of the six spectral plots by my thesis 
supervisor using the common input and the boundary input. A summary of these 
tests are presented at the end of this section. The common input refers to 
input data from the UnitVnVPlan (table \ref{Table:D_13}) and the boundary data 
refers to the boundary table from the unitVnVPlan (table \ref{Table:D_14}) \\ 
A binary decision will be used to evaluate the tests. Each of the automated 
tests loop through a list of input and return a pass or fail message for each 
input. A test will be considered a failure if it returns at least one 
fail 
message. In this case the code will have to be modified until non of the tests 
return a fail message. The evaluation of the plotting tests is dependant on the 
standards of my supervisor. The code will have to be modified if he does not 
consider the plots to be adequate enough for his research. The evaluation of 
test-Plotting-Inspect and test-Plotting-Inspect-Bound are given in the tables 
below. A list of passing input (inputs that do 
not throw exceptions) and passing boundary inputs was given to my supervisor. 
He ran the software with 
these input tables and gave a check mark if the theoretical eigenvalues 
overlapped 
with the appropriate ones. A check mark in the furtherst right column will 
indicate that my supervisor approved of the plot for that particular input.  

\begin{table}[h!]
	\centering
	\begin{tabular}{|c|c|c|c|c|c|c|c|c|c|c|c|c|c|c|c|c|c|c|c|c|c|c|c|}
		\hline        
		Input ID& k& N & P & Result \\
		\hline
		I1      &0.6 &100 &2 & \checkmark \\ \hline
		I2     &0.1 &120 &2 & \checkmark \\ \hline
		I3     &0.9 &500 &2 & \checkmark\\ \hline 
		I4     &0.88 &550 &2 & \checkmark \\ \hline 
		I5      &0.99 &200 &2 & \checkmark \\ \hline
		I6     &0.65 &700 & 2& \checkmark \\ \hline
		I7    &0.4 &100 & 2& \checkmark \\ \hline 
		I8     &0.8 &400 &4& \checkmark \\ \hline
		I9      &0.9 &500 &4 & \checkmark \\ \hline
		I10     &0.2 &700 &4 & \checkmark \\ \hline
		I11     &0.3 &200 &4 & \checkmark \\ \hline 
		I13    &0.8 &100 &4 & \checkmark  \\ \hline
		I14      &0.89 &150 &4 & \checkmark \\ \hline
		I15     &0.69 &500 &4 & \checkmark \\ \hline
		I16     &0.55 &300 &2 & \checkmark \\ \hline 
		I17    &0.9 &400 &4 & \checkmark  \\ 
		\hline
	\end{tabular}\\
	\caption{Test-Plotting-Inspect}
	\label{Table:D_13}
\end{table}  

\clearpage
\begin{table}[h!]
	\centering
	\begin{tabular}{|c|c|c|c|c|c|c|c|c|c|c|c|c|c|c|c|c|c|c|c|c|c|c|c|}
		\hline        
		Input ID& k& N & P & Result \\
		\hline
		I1      &0.9 &100 &2 & \checkmark \\ \hline
		I2     &0.99 &120 &2 & \checkmark \\ \hline
		I3     &0.9999 &500 &2 & \checkmark \\ \hline 
		I4     &0.99999 &550 &2 & \checkmark \\ \hline 
		I5      &0.999999 &2 &4 & \checkmark \\ \hline
		I6     &0.01 &700 & 2& \checkmark \\ \hline
		I7    &0.0001 &100 & 2& \checkmark \\ \hline 
		I8     &0.000001 &400 &4& \checkmark \\ \hline
		I9      &0.0000001 &500 &4 & \checkmark \\ 
		\hline
	\end{tabular}\\
	\caption{test-Plotting-Inspect-Bound}
	\label{Table:D_14}
\end{table} 

\section{Nonfunctional Requirements Evaluation}

Nonfunctional tests of units will not be relevant to \progname. 
For system tests related to Nonfunctional requirements please see the System 
Verification and Validation Plan at 
\url{https://github.com/whitere123/CAS741_REW}. 
	
\section{Comparison to Existing Implementation}	

My supervisor and I could not find any software that performs the same tasks as 
\progname. 

\section{Unit Testing} \label{unitest}

This section will provide a brief summary of each unit test. It will list the 
data table used, assert statement and expected value of assert statement. 
Traces from tests to requirements and tests to modules are presented in 
\ref{trace1} and \ref{trace2}. 

\begin{enumerate} 
	\item{test-InParams-N\\} \label{inparamsN} 
	
	Input: Table \ref{Table:D_13}
	
	Assert Statement: $assert(InParams.N,N)$
	
	Output: assert=True 
	
	\item{test-InParams-P\\} \label{inparamsP} 
	
	Input: Table \ref{Table:D_13}
	
	Assert Statement: $assert(InParams.P,P)$
	
	Output: assert=True 
	
	\item{test-InParams-k\\} \label{inparamsK} 
	
	Input: Table \ref{Table:D_13}
	
	Assert Statement: $assert(InParams.k,k)$
	
	Output: assert=True 
	
	\item{test-NumParams-Dom\\} \label{numparamsdom}
	
	Input: Table \ref{Table:D_13}
	
	Assert Statement: $assert(NumParams.Dom(1),-NumParams.xend)  \\
	 \& \ 
	assert(NumParams.Dom(end),NumParams.xend)$
	
	Output: assert=True 
	
	\item{test-NumParams-Dom-Bound\\} \label{numparamsdombound}
	
	Input: Table \ref{Table:D_14}
	
	Assert Statement: $assert(NumParams.Dom(1),-NumParams.xend)  \\
	\& \ 
	assert(NumParams.Dom(end),NumParams.xend)$ 
	
	Output: assert=True 
	
	\item{test-NumParams-EllipMat\\} \label{numparamsellipmat}
	
	Input: Table \ref{Table:D_13} 
	
	Assert Statement: assert($Numpars.ellipjdn(2) , Numpars.EllipMat(2,2)$)
	
	Output: assert = True 
	
	\item{test-SpecMAT-Tr1\\} \label{specmattr1}

	Input: Table \ref{Table:D_13}
	
	Assert Statement: assert($SpecMat.NUM_{x} , -SpecMat.NUM_{x}^{T}$) 
	
	Output: assert=True 
	
\end{enumerate}

\noindent
Test-Plotting-Inspect and test-Plotting-Inspect-Bound are explained in 
\ref{funct}. 

\section{Changes Due to Testing} 

All of the tests passed. No changes were made to the software package.

\section{Automated Testing} \label{automation} 

The tests in \ref{unitest} were automated. Data tables were organized into 
MATLAB matrices, with input as rows. A for loop was constructed to cycle 
through the rows and input them into arguments in \progname. The aforementioned 
assert statements \ref{unitest} were called in each iteration of the loop and 
the result of the test was printed to the screen (either pass or fail). 
The results of 
these automated tests are summarized in the following tables. Details about the 
intial state, Input, Output, expected result and result of running the
test are given in the following tables. A check mark indicates that the test 
returned the expected result. A link to a description of the unit test is 
provided in the table caption.

\clearpage
\begin{table}[h!]
	\centering
	\begin{tabular}{|c|c|c|c|c|c|c|c|c|c|c|c|c|c|c|c|c|c|c|c|c|c|c|c|}
		\hline        
		Input ID& k& N & P &Expected Result& Test Result \\
		\hline
		I1     &1 &100 &2 & True & \checkmark \\ \hline
		I2     &0 &100 &4 & True & \checkmark \\ \hline 
		I3     &-10 &50 &2 &True& \checkmark \\ \hline 
		I4      &-55 &200 &4 &True& \checkmark \\ \hline
		I5     &2 &150 & 2& True& \checkmark \\ \hline
		I6     &1.0001 & 100 &2 & True& \checkmark \\ \hline
		I7     &-0.001 & 200 &4 & True& \checkmark\\ \hline 
		I8     &1.01 & 200 &2 &True& \checkmark \\ \hline 
		I9      &-0.01 &100 &4 &True& \checkmark \\ \hline
		I10     &0.7 & -10 &2 & True& \checkmark \\ \hline
		I11     &0.9 & 20 &4 & True& \checkmark \\ \hline 
		I12     &0.9 & 5 &2 &True & \checkmark \\ \hline 
		I13      &0.7 &0 &4 &True & \checkmark \\ \hline
		I14     &0.5 & -1 & 2& True& \checkmark \\ \hline 
		I15    &0.9 & 100 &0 & True& \checkmark \\ \hline 
		I16     &0.9 & 500 &-2 &True& \checkmark \\ \hline 
		I17      &0.7 &150 &-4 &True& \checkmark \\ \hline
		I18     &0.5 & 200 & 100& True& \checkmark \\ \hline 
		I19     &0.7 & 0.5 & 100& True& \checkmark \\
		\hline
	\end{tabular}\\
	\caption{test-InParams-N \ref{inparamsN}, 
	test-InParams-P \ref{inparamsP}, test-InParams-k \ref{inparamsK}, 
	test-NumParams-Dom \ref{numparamsdom}} 
\end{table} 

\clearpage
\begin{table}[h!]
	\centering
	\begin{tabular}{|c|c|c|c|c|c|c|c|c|c|c|c|c|c|c|c|c|c|c|c|c|c|c|c|}
		\hline        
		Input ID& k& N & P & Expected Result& Result \\
		\hline
		I1      &0.9 &100 &2 & True& \checkmark \\ \hline
		I2     &0.99 &120 &2 & True & \checkmark \\ \hline
		I3     &0.9999 &500 &2 &True & \checkmark \\ \hline 
		I4     &0.99999 &550 &2 & True& \checkmark \\ \hline 
		I5      &0.999999 &2 &4 & True& \checkmark \\ \hline
		I6     &0.01 &700 & 2& True& \checkmark \\ \hline
		I7    &0.0001 &100 & 2& True& \checkmark \\ \hline 
		I8     &0.000001 &400 &4&True & \checkmark \\ \hline
		I9      &0.0000001 &500 &4&True & \checkmark \\ 
		\hline
	\end{tabular}\\
	\caption{test-NumParams-Dom-Bound \ref{numparamsdombound}}
\end{table} 

\clearpage
\begin{table}[h!]
	\centering
	\begin{tabular}{|c|c|c|c|c|c|c|c|c|c|c|c|c|c|c|c|c|c|c|c|c|c|c|c|}
		\hline        
		Input ID& k& N & P &Expected Result& Test Result \\
		\hline
		I1     &1 &100 &2 & True & \checkmark \\ \hline
		I2     &0 &100 &4 & True & \checkmark \\ \hline 
		I3     &-10 &50 &2 &True& \checkmark \\ \hline 
		I4      &-55 &200 &4 &True& \checkmark \\ \hline
		I5     &2 &150 & 2& True& \checkmark \\ \hline
		I6     &1.0001 & 100 &2 & True& \checkmark \\ \hline
		I7     &-0.001 & 200 &4 & True& \checkmark\\ \hline 
		I8     &1.01 & 200 &2 &True& \checkmark \\ \hline 
		I9      &-0.01 &100 &4 &True& \checkmark \\ \hline
		I10     &0.7 & -10 &2 & True& \checkmark \\ \hline
		I11     &0.9 & 20 &4 & True& \checkmark \\ \hline 
		I12     &0.9 & 5 &2 &True & \checkmark \\ \hline 
		I13      &0.7 &0 &4 &True & \checkmark \\ \hline
		I14     &0.5 & -1 & 2& True& \checkmark \\ \hline 
		I15    &0.9 & 100 &0 & True& \checkmark \\ \hline 
		I16     &0.9 & 500 &-2 &True& \checkmark \\ \hline 
		I17      &0.7 &150 &-4 &True& \checkmark \\ \hline
		I18     &0.5 & 200 & 100& True& \checkmark \\ \hline 
		I19     &0.7 & 0.5 & 100& True& \checkmark \\
		\hline
	\end{tabular}\\
	\caption{test-NumParams-EllipMat \ref{numparamsellipmat}} 
\end{table} 

\clearpage
\begin{table}[h!]
	\centering
	\begin{tabular}{|c|c|c|c|c|c|c|c|c|c|c|c|c|c|c|c|c|c|c|c|c|c|c|c|}
		\hline        
		Input ID& k& N & P &Expected Result& Test Result \\
		\hline
		I1     &1 &100 &2 & True & \checkmark \\ \hline
		I2     &0 &100 &4 & True & \checkmark \\ \hline 
		I3     &-10 &50 &2 &True& \checkmark \\ \hline 
		I4      &-55 &200 &4 &True& \checkmark \\ \hline
		I5     &2 &150 & 2& True& \checkmark \\ \hline
		I6     &1.0001 & 100 &2 & True& \checkmark \\ \hline
		I7     &-0.001 & 200 &4 & True& \checkmark\\ \hline 
		I8     &1.01 & 200 &2 &True& \checkmark \\ \hline 
		I9      &-0.01 &100 &4 &True& \checkmark \\ \hline
		I10     &0.7 & -10 &2 & True& \checkmark \\ \hline
		I11     &0.9 & 20 &4 & True& \checkmark \\ \hline 
		I12     &0.9 & 5 &2 &True & \checkmark \\ \hline 
		I13      &0.7 &0 &4 &True & \checkmark \\ \hline
		I14     &0.5 & -1 & 2& True& \checkmark \\ \hline 
		I15    &0.9 & 100 &0 & True& \checkmark \\ \hline 
		I16     &0.9 & 500 &-2 &True& \checkmark \\ \hline 
		I17      &0.7 &150 &-4 &True& \checkmark \\ \hline
		I18     &0.5 & 200 & 100& True& \checkmark \\ \hline 
		I19     &0.7 & 0.5 & 100& True& \checkmark \\
		\hline
	\end{tabular}\\
	\caption{test-SpecMAT-Tr1 \ref{specmattr1}} 
\end{table} 

\newpage
\section{Trace to Requirements}
		
This section shows the traceability between the modules and the test cases.

\label{trace1}
\begin{table}[h]
	\centering
	\begin{tabular}{|c|c|c|c|c|c|c|c|c|c|c|c|c|c|c|c|c|c|c|c|c|c|c|c|}
		\hline        
		Tests & Rin& Rfind & Rplt  \\
		\hline
		test-InParams-N     &X & &   \\ \hline
		test-InParams-P    &X& &    \\ \hline
		test-InParams-k    &X & &   \\ \hline 
		test-NumParams-Dom    & &X &     \\ \hline 
		test-NumParams-Dom-Bound    & &X &    \\ \hline 
		test-NumParams-EllipMat    & &X &     \\ \hline 
		test-SpecMAT-Tr1  & &X &    \\ \hline 
		test-Plotting-Inspect & & &X     \\ \hline 
		test-Plotting-Inspect-Bound  & & &X   \\
		\hline
	\end{tabular}\\
	\caption{Traceability Between Test }
	\label{Table:D_1}
\end{table} 		

	
\section{Trace to Modules} \label{trace2}
\clearpage
\begin{table}
	\centering
	\begin{tabular}{p{0.3\textwidth} p{0.6\textwidth}}
		\toprule
		\textbf{Test Case} & \textbf{Modules}\\
		\midrule
		test-InParams-N & M1, M2, M13\\
		test-InParams-P & M1, M2, M13\\
		test-InParams-k & M1, M2, M13\\
		test-NumParams-Dom & M1, M6, M13\\
		test-NumParams-Dom-Bound & M1, M6, M13\\
		test-NumParams-EllipMat & M1,M4, M13\\
		test-SpecMAT-Tr1 &M1,M4, M13 \\
		test-Plotting-Inspect & M1,M11,M3 \\
		test-Plotting-Inspect-Bound & M1,M11,M3 \\
		\bottomrule
	\end{tabular}
	\caption{Trace Between Test Cases and Modules}
	\label{TblRT}
\end{table}	

\section{Code Coverage Metrics} 

No code covering metrics were used in the creation of \progname.

\bibliographystyle{plainnat}

\bibliography{../../../ReferenceMaterial/References}

\end{document}